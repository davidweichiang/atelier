\documentclass[11pt,twoside]{book}

%%Page Size (rev. 08/19/2016)
%\usepackage[inner=0.5in, outer=0.5in, top=0.5in, bottom=0.5in, papersize={6in,9in}, head=12pt, headheight=30pt, headsep=5pt]{geometry}
\usepackage[inner=0.5in, outer=0.5in, top=0.5in, bottom=0.3in, papersize={5.5in,8.5in}, head=12pt, headheight=30pt, headsep=5pt]{geometry}
%% width of textblock = 324 pt / 4.5in
%% A5 = 5.8 x 8.3 inches -- if papersize is A5, then margins should be [inner=0.75in, outer=0.55in, top=0.4in, bottom=0.4in]


%%Header (rev. 4/11/2011)
\usepackage{fancyhdr}
 \pagestyle{fancy}
\renewcommand{\chaptermark}[1]{\markboth{#1}{}}
\renewcommand{\sectionmark}[1]{\markright{\thesection\ #1}}
 \fancyhf{}
\fancyhead[LE,RO]{\thepage}
\fancyhead[CE]{Graduale O.P.}
\fancyhead[CO]{\leftmark}
 \fancypagestyle{plain}{ %
\fancyhf{} % remove everything
\renewcommand{\headrulewidth}{0pt} % remove lines as well
\renewcommand{\footrulewidth}{0pt}}



\usepackage[autocompile,allowdeprecated=false]{gregoriotex}
\usepackage{gregoriosyms}
\gresetgregoriofont[op]{greciliae}




%%Titles (rev. 9/4/2011) -- TOCLESS --- lets you have sections that don't appear in the table of contents

\setcounter{secnumdepth}{-1}

\usepackage[compact,nobottomtitles*]{titlesec}
\titlespacing*{\chapter}{0pt}{*0}{*1}
\titlespacing*{\section}{0pt}{*0}{*1}
\titlespacing*{\subsection}{0pt}{*0}{*0}
\titlespacing*{\subsubsection}{0pt}{10pt}{*0}
\titleformat{\chapter} {\normalfont\LARGE\sc\center}{\thechapter}{1em}{}
\titleformat{\section} {\normalfont\Large\sc\center}{\thesection}{1em}{}
\titleformat{\subsection} {\normalfont\large\sc\center}{\thesubsection}{1em}{}
\titleformat{\subsubsection}{\normalfont\normalsize\it}{\thesubsubsection}{1em}{}

\newcommand{\nocontentsline}[3]{}
\newcommand{\tocless}[2]{\bgroup\let\addcontentsline=\nocontentsline#1{#2}\egroup} %% lets you have sections that don't appear in the table of contents


%%%

%%Index (rev. December 11, 2013)
\usepackage[noautomatic,nonewpage]{imakeidx}


\makeindex[name=incipit,title=Index]
\indexsetup{level=\section,toclevel=section,noclearpage}

\usepackage[indentunit=8pt,rule=.5pt,columns=2]{idxlayout}


%%Table of Contents (rev. May 16, 2011)

%\usepackage{multicol}
%\usepackage{ifthen}
%\usepackage[toc]{multitoc}

%% General settings (rev. January 19, 2015)

\usepackage{ulem}

\usepackage[latin,english]{babel}
\usepackage{lettrine}


\usepackage{fontspec}

\setmainfont[Ligatures=TeX,BoldFont=MinionPro-Bold,ItalicFont=MinionPro-It, BoldItalicFont=MinionPro-BoldIt]{MinionPro-Regular-Modified.otf}



%% Style for translation line
\grechangestyle{translation}{\textnormal\selectfont}
\grechangestyle{annotation}{\fontsize{10}{10}\selectfont}
\grechangestyle{commentary}{\textnormal\selectfont}
\gresetcustosalteration{invisible}



%\grechangedim{annotationseparation}{0.1cm}{scalable}

%\GreLoadSpaceConf{smith-four}

\frenchspacing

\usepackage{indentfirst} %%%indents first line after a section

\usepackage{graphicx}
%\usepackage{tocloft}

%%Hyperref (rev. August 20, 2011)
%\usepackage[colorlinks=false,hyperindex=true,bookmarks=true]{hyperref}
\usepackage{hyperref}
\hypersetup{pdftitle={Graduale O.P. 2016}}
\hypersetup{pdfauthor={Order of Preachers (\today)}}
\hypersetup{pdfsubject={Liturgy}}
\hypersetup{pdfkeywords={Dominican, Liturgy, Order of Preachers, Dominican Rite, Liturgia Horarum, Divine Office}}

\newlength{\drop}



\begin{document}


%%%Initial Matter within Body (20 May 2011)
\raggedbottom
\grechangedim{maxbaroffsettextleft}{0 cm}{scalable}


\chapter[3{rd} Sunday of Advent]{Graduale O.P.}
\section{3{rd} Sunday of Advent}



\newpage

%%Combination

\subsection{Officium}  \greannotation{I} \index[Officium]{Gaudete in Domino} \label{Gaudete in Domino (Officium)} \grecommentary[0pt]{Phil 4:4, 5; Ps 84} \gresetinitiallines{1}  \gregorioscore{graduale-chants/in--gaudete_in_domino--dominican--id_4287}  \vspace{5pt} \par{Rejoice in the Lord always; again, I say, rejoice. Let your moderation be known to all men. The Lord is near. Have no anxiety, but in every prayer let your petitions be made known to God. \Vbar. And the peace of God, which surpasses all understanding, keep your hearts and minds.} \newpage

\subsection{Responsorium}  \greannotation{VII} \index[Responsorium]{Qui sedes, Domine} \label{Qui sedes, Domine (Responsorium)} \grecommentary[0pt]{Ps 79:2, 3; \Vbar. 2} \gresetinitiallines{1}  \gregorioscore{graduale-chants/gr--qui_sedes_domine--dominican--id_6651}  \vspace{5pt} \par{You, O Lord, who sit upon the cherubim, stir up Your might, and come. \Vbar. Take heed, O you who rule Israel, you who lead Joseph like a sheep.} \newpage
\subsection{Responsorium}  \greannotation{V} \index[Responsorium]{Fuit homo} \label{Fuit homo (Responsorium)} \grecommentary[0pt]{Io 1:6; \Vbar. 7} \gresetinitiallines{1}  \gregorioscore{graduale-chants/gr--fuit_homo--dominican}  \vspace{5pt} \par{There was a man sent from God, whose name was John. \Vbar. This man came to bear witness concerning the light, to prepare a perfect people for the Lord.} \newpage

\subsection{Alleluia}  \greannotation{IV} \index[Alleluia]{Excita Domine} \label{Excita Domine (Alleluia)} \grecommentary[0pt]{Ps 79:3} \gresetinitiallines{1}  \gregorioscore{graduale-chants/al--excita_domine--dominican--id_5097}  \vspace{5pt} \par{Stir up Your might, O Lord, and come, that You may save us.} \newpage
\subsection{Offertorium}  \greannotation{IV} \index[Offertorium]{Benedixisti Domine} \label{Benedixisti Domine (Offertorium)} \grecommentary[0pt]{Ps 84:2} \gresetinitiallines{1}  \gregorioscore{graduale-chants/of--benedixisti_domine--dominican--id_4304}  \vspace{5pt} \par{You have blessed Your land, O Lord; You have turned away the captivity of Jacob; You have forgiven the iniquity of Your people.} \newpage
\subsection{Communio}  \greannotation{VII} \index[Communio]{Dicite: Pusillanimes} \label{Dicite: Pusillanimes (Communio)} \grecommentary[0pt]{Cf. Is 35:4} \gresetinitiallines{1}  \gregorioscore{graduale-chants/co--dicite_pusillanimes--dominican--id_5719}  \vspace{5pt} \par{Say to the faint-hearted: "Take courage and fear not; behold, our God will come and save us."} \newpage


\end{document}
