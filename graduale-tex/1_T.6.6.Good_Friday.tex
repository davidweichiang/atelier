\documentclass[11pt,twoside]{book}

%%Page Size (rev. 08/19/2016)
%\usepackage[inner=0.5in, outer=0.5in, top=0.5in, bottom=0.5in, papersize={6in,9in}, head=12pt, headheight=30pt, headsep=5pt]{geometry}
\usepackage[inner=0.5in, outer=0.5in, top=0.5in, bottom=0.3in, papersize={5.5in,8.5in}, head=12pt, headheight=30pt, headsep=5pt]{geometry}
%% width of textblock = 324 pt / 4.5in
%% A5 = 5.8 x 8.3 inches -- if papersize is A5, then margins should be [inner=0.75in, outer=0.55in, top=0.4in, bottom=0.4in]


%%Header (rev. 4/11/2011)
\usepackage{fancyhdr}
 \pagestyle{fancy}
\renewcommand{\chaptermark}[1]{\markboth{#1}{}}
\renewcommand{\sectionmark}[1]{\markright{\thesection\ #1}}
 \fancyhf{}
\fancyhead[LE,RO]{\thepage}
\fancyhead[CE]{Graduale O.P.}
\fancyhead[CO]{\leftmark}
 \fancypagestyle{plain}{ %
\fancyhf{} % remove everything
\renewcommand{\headrulewidth}{0pt} % remove lines as well
\renewcommand{\footrulewidth}{0pt}}



\usepackage[autocompile,allowdeprecated=false]{gregoriotex}
\usepackage{gregoriosyms}
\gresetgregoriofont[op]{greciliae}




%%Titles (rev. 9/4/2011) -- TOCLESS --- lets you have sections that don't appear in the table of contents

\setcounter{secnumdepth}{-1}

\usepackage[compact,nobottomtitles*]{titlesec}
\titlespacing*{\chapter}{0pt}{*0}{*1}
\titlespacing*{\section}{0pt}{*0}{*1}
\titlespacing*{\subsection}{0pt}{*0}{*0}
\titlespacing*{\subsubsection}{0pt}{10pt}{*0}
\titleformat{\chapter} {\normalfont\LARGE\sc\center}{\thechapter}{1em}{}
\titleformat{\section} {\normalfont\Large\sc\center}{\thesection}{1em}{}
\titleformat{\subsection} {\normalfont\large\sc\center}{\thesubsection}{1em}{}
\titleformat{\subsubsection}{\normalfont\normalsize\it}{\thesubsubsection}{1em}{}

\newcommand{\nocontentsline}[3]{}
\newcommand{\tocless}[2]{\bgroup\let\addcontentsline=\nocontentsline#1{#2}\egroup} %% lets you have sections that don't appear in the table of contents


%%%

%%Index (rev. December 11, 2013)
\usepackage[noautomatic,nonewpage]{imakeidx}


\makeindex[name=incipit,title=Index]
\indexsetup{level=\section,toclevel=section,noclearpage}

\usepackage[indentunit=8pt,rule=.5pt,columns=2]{idxlayout}


%%Table of Contents (rev. May 16, 2011)

%\usepackage{multicol}
%\usepackage{ifthen}
%\usepackage[toc]{multitoc}

%% General settings (rev. January 19, 2015)

\usepackage{ulem}

\usepackage[latin,english]{babel}
\usepackage{lettrine}


\usepackage{fontspec}

\setmainfont[Ligatures=TeX,BoldFont=MinionPro-Bold,ItalicFont=MinionPro-It, BoldItalicFont=MinionPro-BoldIt]{MinionPro-Regular-Modified.otf}



%% Style for translation line
\grechangestyle{translation}{\textnormal\selectfont}
\grechangestyle{annotation}{\fontsize{10}{10}\selectfont}
\grechangestyle{commentary}{\textnormal\selectfont}
\gresetcustosalteration{invisible}



%\grechangedim{annotationseparation}{0.1cm}{scalable}

%\GreLoadSpaceConf{smith-four}

\frenchspacing

\usepackage{indentfirst} %%%indents first line after a section

\usepackage{graphicx}
%\usepackage{tocloft}

%%Hyperref (rev. August 20, 2011)
%\usepackage[colorlinks=false,hyperindex=true,bookmarks=true]{hyperref}
\usepackage{hyperref}
\hypersetup{pdftitle={Graduale O.P. 2017}}
\hypersetup{pdfauthor={Order of Preachers (\today)}}
\hypersetup{pdfsubject={Liturgy}}
\hypersetup{pdfkeywords={Dominican, Liturgy, Order of Preachers, Dominican Rite, Liturgia Horarum, Divine Office}}

\newlength{\drop}



\begin{document}


%%%Initial Matter within Body (20 May 2011)
\raggedbottom
\grechangedim{maxbaroffsettextleft}{0 cm}{scalable}


\chapter[Friday of the Passion of the Lord]{Graduale O.P.}
\section{Friday of the Passion of the Lord}



\newpage

\subsection{Improperia}  \grechangedim{spaceabovelines}{0.6cm}{scalable}

%\emph{A veiled cross is carried to the steps of the altar. Then two priests standing to the right of the altar hold the cross by its arms and turning to the community or the people sing the verse:}

Cantors:
\vspace{3pt}
\greannotation{VIII}
\gabcsnippet{(c4)Pó(c)pu(df)le(efED) me(cd)us,(d) (;) quid(e!fg) fe(gf)ci(e) ti(fe)bi,(de) (;) aut(c) in(e) quo(ixgv_//hih/ghg) (,4) con(fe)tris(fg)tá(fe)vi(fg) te?(efEDd) (;) Re(c)spón(efeeV>)de(dc) mi(ef)hi:(d) (:) Qui(deDC)a(c) e(e!fg)dú(gf)xi(ef) te(d) (;) de(ce) ter(g)ra(ixgv_//hih/ghg) Æ(fe)gýp(fg)ti,(e) (;) pa(ce)rás(ixgv_//hiHGhg)ti(fe) (,4) Cru(fg)cem(ef/gffe) (;) Sal(fV>)va(f)tó(ghg)ri(f) tu(gf)o.(e!fg) (::)}

\vspace{10pt}

%\emph{When the verse is finished, the two deacons stand at the steps of the sanctuary and sing the ``Hagios." Each time ``Hagios" is sung, both kneel; having finished the word ``Hagios," they immediately rise.}
Priest Cantors:
\vspace{10pt}
\gresetinitiallines{0}
\gabcsnippet{(c4)A<alt>Kneel</alt>(g)gi(fe)os(fgffe) (;) o<alt>Rise</alt>(c) The(d)ós,(fg!hvhg) (:) A<alt>Kneel</alt>(g)gi(fe)os(fgffe) (;) is<alt>Rise</alt>(c)chy(d)rós,(fg!hvhg) (:) (z) A<alt>Kneel</alt>(gh)gi(g)os(ixhg/hiHGg) (;) a<alt>Rise</alt>(f)thá(g)na(h)tos,(iyhv_//jjvIHih) (:) e(g)lé(h!ijIH)i(g)son(hghg) (,4) i(fvED)más.(fg/hhg) (::)}
\vspace{10pt}

Choir:
\vspace{10pt}
%\emph{Then the choir and, if possible, all those assembled sing the following in response. Each time ``Sanctus" is sung, all kneel; having finished the word ``Sanctus," all immediately rise. (The priests who hold the cross do not kneel, nor do the two deacons kneel when the community kneels, nor the community when the two deacons kneel.)}
\gabcsnippet{(c4)Sanc<alt>Kneel</alt>(g)tus(fgffe) (;) De<alt>Rise</alt>(cd)us,(fg!hvhg) (:) Sanc<alt>Kneel</alt>(g)tus(fgffe) (;) for<alt>Rise</alt>(cd)tis,(fg!hvhg) (:) Sanc<alt>Kneel</alt>(ghg~)tus(ixhg/hiHGg) (;) et<alt>Rise</alt>(f) im(g)mor(h)tá(hj/jvIH)lis,(ih) (:) mi(g)se(h)ré(h!ijIHG)re(hghg) (,4) no(fvED)bis.(fg/hhg) (::)}

\newpage

Cantors:
%\emph{When the choir has finished the response, the two priests move more towards the middle of the altar and sing the second verse:}}
\gabcsnippet{(c4) <sp>V/</sp>. Qui(deDC)a(c) e(e!fg)dú(gf)xi(ef) te(d) (;) per(c) de(e)sér(ixgv_//hih/ghg)tum(fe) (;) qua(fe)dra(fg)gín(fV>)ta(fg) an(f)nis,(e) (;) et(fV>) man(de)na(d) ci(dc)bá(df)vi(de) te,(eed) (;) et(e) in(g)tro(gh)dú(hhg)xi(e) (,3) in(gV>) ter(ef)ram(e) sa(fe)tis(dc) óp(df)ti(de)mam,(ed) (:) pa(ce)rás(ixgv_//hih/ghg)ti(fe) (,4) Cru(fg)cem(ef/gffe) (;) Sal(fV>)va(f)tó(ghg)ri(f) tu(gf)o.(e!fg) (::)}
\vspace{10pt}

\emph{The ``Hagios" and ``Sanctus" are repeated.}
\vspace{10pt}
%\emph{Then the two deacons sing the ``Hagios" again and the choir again responds with ``Sanctus." They kneel and rise as before.}

Cantors:
%\emph{Then the priests move to the middle of the altar and sing the third verse:}
\gabcsnippet{(c4) <sp>V/</sp>. Quid(c) ul(de)tra(d) dé(e!fg)bu(fe)i(e) (,1) fá(e)ce(e)re(e) ti(e)bi,(dc) (;) et(de) non(de) fe(ed)ci?(cd) (:) E(deDC)go(c) qui(ef)dem(ed) (,1) plan(e!fg)tá(gf)vi(ef) te(d) (;) ví(e)ne(c)am(e) me(gh)am(h) (,3) spe(h)ci(g)o(fe)sí(de)si(dc)mam,(c) (:) et(cd) tu(fg) fac(gf)ta(gh) es(h) (,3) mi(h)hi(g) ni(fe)mis(d) a(fe)má(ddv)ra:(c) (:) a(d)cé(c)to(df) nam(fV>)que(d) (,1) mix(ffe)to(c) cum(df) fel(fe)le(d) (;) si(e)tim(f) me(d)am(e) po(f)tás(ddv)ti,(c) (:) et(cd) lán(ffg)ce(f)a(f) (,3) per(f)fo(gh)rás(h)ti(g) la(fe)tus(df) (;) Sal(fV>)va(f)tó(ghg)ri(f) tu(gf)o.(e!fg) (::)}

\vspace{5pt}

\emph{The ``Hagios" and ``Sanctus" are repeated.}

\vspace{10pt}
\par{O my people, what have I done to you, or in what way have I afflicted you? Answer me: Because I led you out of the land of Egypt, you have prepared a cross from your Savior. \par O holy God, holy and mighty, holy immortal one, have mercy on us. \par \Vbar. Because I led you through the desert for forty years, and fed you with manna, and brought you into an excellent land, you have prepared a cross for your Savior. \par \Vbar. What more could I have done for you, that I have not done? I have planted you, my most beautiful vine, and you have proved exceeding bitter to me; for vinegar mixed with gall in my thirst you gave me to drink, and with a lance you pierced the side of your Savior.}

\grechangedim{spaceabovelines}{0cm}{scalable}
         \newpage


\subsection{Antiphona}  \greannotation{VI} \index[Antiphona]{Ecce lignum Crucis} \label{Ecce lignum Crucis (Antiphona)} \grecommentary[0pt]{} \gresetinitiallines{1} \grechangedim{maxbaroffsettextleft@nobar}{0 cm}{scalable} \gregorioscore{graduale-chants/an--ecce_lignum--dominican--id_6369} \grechangedim{maxbaroffsettextleft@nobar}{12 cm}{scalable} \vspace{5pt} \par{Behold the wood of the Cross, on which the savior of the world hung; come, let us adore.}
\subsection{Antiphona}  \greannotation{IV} \index[Antiphona]{Tuam Crucem} \label{Tuam Crucem (Antiphona)} \grecommentary[0pt]{} \gresetinitiallines{1} \grechangedim{maxbaroffsettextleft@nobar}{0 cm}{scalable} \gregorioscore{graduale-chants/an--tuam_crucem--dominican--id_6093} \grechangedim{maxbaroffsettextleft@nobar}{12 cm}{scalable} \vspace{5pt} \par{We adore your Christ, O Lord, we recall your glorious passion; have mercy on us, you who suffered for us.} \newpage
\subsection{Antiphona}  \greannotation{IV} \index[Antiphona]{Crucem tuam} \label{Crucem tuam (Antiphona)} \grecommentary[0pt]{Ps 66:2} \gresetinitiallines{1} \grechangedim{maxbaroffsettextleft@nobar}{0 cm}{scalable} \gregorioscore{graduale-chants/an--crucem_tuam--dominican--id_5371} \grechangedim{maxbaroffsettextleft@nobar}{12 cm}{scalable} \vspace{5pt} \par{We adore your Cross, O Lord, and we praise and glorify your holy Resurrection; for behold, on account of the Cross joy has come to the whole world.}
\subsection{Antiphona}  \greannotation{IV} \index[Antiphona]{Adoremus Crucis} \label{Adoremus Crucis (Antiphona)} \grecommentary[0pt]{} \gresetinitiallines{1} \grechangedim{maxbaroffsettextleft@nobar}{0 cm}{scalable} \gregorioscore{graduale-chants/an--adoremus--dominican--id_6069} \grechangedim{maxbaroffsettextleft@nobar}{12 cm}{scalable} \vspace{5pt} \par{We adore the sign of the Cross, through which we have received the sacrament of salvation.} \newpage
\subsection{Hymnus}  \greannotation{I} \index[Hymnus]{Crux fidelis} \label{Crux fidelis (Hymnus)} \grecommentary[10pt]{Venantius Fortunatus (sæc. VI)} \gresetinitiallines{1} \gresetlyriccentering{syllable} \grechangedim{maxbaroffsettextleft@nobar}{0 cm}{scalable} \gregorioscore{graduale-chants/hy--crux_fidelis--dominican--english} \gresetlyriccentering{vowel} \grechangedim{maxbaroffsettextleft@nobar}{12 cm}{scalable} \vspace{5pt} \par{} \newpage


\subsection{Communio}  \greannotation{I} \index[Communio]{Adoramus te} \label{Adoramus te (Communio)} \grecommentary[0pt]{} \gresetinitiallines{1} \grechangedim{maxbaroffsettextleft@nobar}{0 cm}{scalable} \gregorioscore{graduale-chants/co--adoramus_te--dominican} \grechangedim{maxbaroffsettextleft@nobar}{12 cm}{scalable} \vspace{5pt} \par{We adore you, O Christ, and we bless you, because by your Cross you have redeemed the world.}
\subsection{Communio}  \greannotation{VIII} \index[Communio]{Per lignum} \label{Per lignum (Communio)} \grecommentary[0pt]{} \gresetinitiallines{1} \grechangedim{maxbaroffsettextleft@nobar}{0 cm}{scalable} \gregorioscore{graduale-chants/co--per_lignum--dominican--id_4990--good-friday} \grechangedim{maxbaroffsettextleft@nobar}{12 cm}{scalable} \vspace{5pt} \par{By the tree we became slaves, and by the holy cross we have been set free; the fruit of the tree enslaved us, the Son of God redeemed us. Alleluia!} \newpage
\subsection{Communio}  \greannotation{VII} \index[Communio]{Salvator mundi} \label{Salvator mundi (Communio)} \grecommentary[0pt]{} \gresetinitiallines{1} \grechangedim{maxbaroffsettextleft@nobar}{0 cm}{scalable} \gregorioscore{graduale-chants/co--salvator_mundi--dominican} \grechangedim{maxbaroffsettextleft@nobar}{12 cm}{scalable} \vspace{5pt} \par{Savior of the world, save us, you have redeemed us by your Cross and Blood; help us, we beseech you, our God.}

\end{document}
