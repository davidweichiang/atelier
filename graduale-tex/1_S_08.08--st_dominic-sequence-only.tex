\documentclass[11pt,twoside]{book}

%%Page Size (rev. 08/19/2016)
%\usepackage[inner=0.5in, outer=0.5in, top=0.5in, bottom=0.5in, papersize={6in,9in}, head=12pt, headheight=30pt, headsep=5pt]{geometry}
\usepackage[inner=0.5in, outer=0.5in, top=0.5in, bottom=0.3in, papersize={5.5in,8.5in}, head=12pt, headheight=30pt, headsep=5pt]{geometry}
%% width of textblock = 324 pt / 4.5in
%% A5 = 5.8 x 8.3 inches -- if papersize is A5, then margins should be [inner=0.75in, outer=0.55in, top=0.4in, bottom=0.4in]


%%Header (rev. 4/11/2011)
\usepackage{fancyhdr}
 \pagestyle{fancy}
\renewcommand{\chaptermark}[1]{\markboth{#1}{}}
\renewcommand{\sectionmark}[1]{\markright{\thesection\ #1}}
 \fancyhf{}
\fancyhead[LE,RO]{\thepage}
\fancyhead[CE]{Graduale O.P.}
\fancyhead[CO]{\leftmark}
 \fancypagestyle{plain}{ %
\fancyhf{} % remove everything
\renewcommand{\headrulewidth}{0pt} % remove lines as well
\renewcommand{\footrulewidth}{0pt}}



\usepackage[autocompile,allowdeprecated=false]{gregoriotex}
\usepackage{gregoriosyms}
\gresetgregoriofont[op]{greciliae}




%%Titles (rev. 9/4/2011) -- TOCLESS --- lets you have sections that don't appear in the table of contents

\setcounter{secnumdepth}{-2}

\usepackage[compact,nobottomtitles*]{titlesec}
\titlespacing*{\part}{0pt}{*0}{*1}
\titlespacing*{\chapter}{0pt}{*0}{*1}
\titlespacing*{\section}{0pt}{*0}{*1}
\titlespacing*{\subsubsection}{0pt}{*0}{*0}
\titlespacing*{\subsubsubsection}{0pt}{10pt}{*0}
\titleformat{\part} {\normalfont\Huge\sc\center}{\thepart}{1em}{}
\titleformat{\chapter} {\normalfont\huge\sc\center}{\thechapter}{1em}{}
\titleformat{\section} {\normalfont\LARGE\sc\center}{\thesection}{1em}{}
\titleformat{\subsection} {\normalfont\Large\sc\center}{\thesubsubsection}{1em}{}
\titleformat{\subsubsection}{\normalfont\large\sc\center}{\thesubsubsubsection}{1em}{}

\newcommand{\nocontentsline}[3]{}
\newcommand{\tocless}[2]{\bgroup\let\addcontentsline=\nocontentsline#1{#2}\egroup} %% lets you have sections that don't appear in the table of contents


%%%

%%Index (rev. December 11, 2013)
\usepackage[noautomatic,nonewpage]{imakeidx}


\makeindex[name=incipit,title=Index]
\indexsetup{level=\section,toclevel=section,noclearpage}

\usepackage[indentunit=8pt,rule=.5pt,columns=2]{idxlayout}


%%Table of Contents (rev. May 16, 2011)

%\usepackage{multicol}
%\usepackage{ifthen}
%\usepackage[toc]{multitoc}

%% General settings (rev. January 19, 2015)

\usepackage{ulem}

\usepackage[latin,english]{babel}
\usepackage{lettrine}


\usepackage{fontspec}

\setmainfont[Ligatures=TeX,BoldFont=MinionPro-Bold,ItalicFont=MinionPro-It, BoldItalicFont=MinionPro-BoldIt]{MinionPro-Regular-Modified.otf}


%% Style for translation line
\grechangestyle{translation}{\textnormal\selectfont}
\grechangestyle{annotation}{\fontsize{10}{10}\selectfont}
\grechangestyle{commentary}{\textnormal\selectfont}
\gresetcustosalteration{invisible}



%\grechangedim{annotationseparation}{0.1cm}{scalable}

%\GreLoadSpaceConf{smith-four}

\frenchspacing

\usepackage{indentfirst} %%%indents first line after a section

\usepackage{graphicx}
%\usepackage{tocloft}

%%Hyperref (rev. August 20, 2011)
%\usepackage[colorlinks=false,hyperindex=true,bookmarks=true]{hyperref}
\usepackage{hyperref}
\hypersetup{pdftitle={Graduale O.P. 2016}}
\hypersetup{pdfauthor={Order of Preachers (\today)}}
\hypersetup{pdfsubject={Liturgy}}
\hypersetup{pdfkeywords={Dominican, Liturgy, Order of Preachers, Dominican Rite, Liturgia Horarum, Divine Office}}

\newlength{\drop}



\begin{document}

\pagestyle{plain}


%%%Initial Matter within Body (20 May 2011)
\raggedbottom
\grechangedim{maxbaroffsettextleft}{0 cm}{scalable}


%%Combination

 \subsubsection{Sequentia}  \greannotation{VI} \index[Sequentia]{In caelesti hierarchia} \label{In caelesti hierarchia (Sequentia)} \grecommentary[0pt]{} \gresetinitiallines{1} \grechangedim{maxbaroffsettextleft@nobar}{12 cm}{scalable} \grechangedim{spaceabovelines}{0.5cm}{scalable} \gresetlyriccentering{vowel}  \gregorioscore{graduale-chants/se--in_caelesti_hierarchia--dominican--id_5838}  \vspace{5pt} \par{In the heavenly hierarchy, let there sound a new harmony, produced in a new canticle; and let the melody of our choir on this earth agree therewith, rejoicing with Dominic. \par From the waste of Egypt the creator of the world called the man of his decree. On the ark of poverty he fords the stream of vanity for the salvation of souls. \par Under the figure of a hound the preacher of the world is shown beforehand to his mother. Bearing in his mouth a torch, he exhorts all to the law of love. \par He is a new law-giver, he is the imitator of Elias and a detester of crimes. He scatters the foxes of Samson and with the trumpet of Gideon he puts to flight the hosts of the enemy. \par 5. While yet alive in body he restores to a mother her son recalled from the dead. A storm bows to his sign of the cross. The company of the brethren eats bread sent as a gift of God. \par The blessed man, in whom all the Church now takes joy, is exalted. He fills the world with his seed, and at last is located in the army of heaven. \par The grain of wheat lies hidden, the star lurks concealed; but the maker of all things commands the bones of Joseph to germinate and the star to radiate for the salvation of souls. \par O how the fragrance of his tomb surpassing every aroma, proves the bloom of his flesh! The sick come with haste and are cured, the blind and the lame are made whole by the repetition of his power. \par Wherefore let us sound forth with full voice our praises to the great Saint Dominic. O people of need, as you follow in his footsteps, call on him to intercede. \par But do you, O loving father, good Shepherd and defender of your flock, commend at the court of the great king by your constant prayer the concerns of your forsaken sheep. Amen. Alleluia.} \newpage
\end{document}
