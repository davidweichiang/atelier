\documentclass[11pt,twoside]{book}

%%Page Size (rev. 08/19/2016)
%\usepackage[inner=0.5in, outer=0.5in, top=0.5in, bottom=0.5in, papersize={6in,9in}, head=12pt, headheight=30pt, headsep=5pt]{geometry}
\usepackage[inner=0.5in, outer=0.5in, top=0.5in, bottom=0.3in, papersize={5.5in,8.5in}, head=12pt, headheight=30pt, headsep=5pt]{geometry}
%% width of textblock = 324 pt / 4.5in
%% A5 = 5.8 x 8.3 inches -- if papersize is A5, then margins should be [inner=0.75in, outer=0.55in, top=0.4in, bottom=0.4in]


%%Header (rev. 4/11/2011)
\usepackage{fancyhdr}
 \pagestyle{fancy}
\renewcommand{\chaptermark}[1]{\markboth{#1}{}}
\renewcommand{\sectionmark}[1]{\markright{\thesection\ #1}}
 \fancyhf{}
\fancyhead[LE,RO]{\thepage}
\fancyhead[CE]{Graduale O.P.}
\fancyhead[CO]{\leftmark}
 \fancypagestyle{plain}{ %
\fancyhf{} % remove everything
\renewcommand{\headrulewidth}{0pt} % remove lines as well
\renewcommand{\footrulewidth}{0pt}}



\usepackage[autocompile,allowdeprecated=false]{gregoriotex}
\usepackage{gregoriosyms}
\gresetgregoriofont[op]{greciliae}




%%Titles (rev. 9/4/2011) -- TOCLESS --- lets you have sections that don't appear in the table of contents

\setcounter{secnumdepth}{-1}

\usepackage[compact,nobottomtitles*]{titlesec}
\titlespacing*{\chapter}{0pt}{*0}{*1}
\titlespacing*{\section}{0pt}{*0}{*1}
\titlespacing*{\subsection}{0pt}{*0}{*0}
\titlespacing*{\subsubsection}{0pt}{10pt}{*0}
\titleformat{\chapter} {\normalfont\LARGE\sc\center}{\thechapter}{1em}{}
\titleformat{\section} {\normalfont\Large\sc\center}{\thesection}{1em}{}
\titleformat{\subsection} {\normalfont\large\sc\center}{\thesubsection}{1em}{}
\titleformat{\subsubsection}{\normalfont\normalsize\it}{\thesubsubsection}{1em}{}

\newcommand{\nocontentsline}[3]{}
\newcommand{\tocless}[2]{\bgroup\let\addcontentsline=\nocontentsline#1{#2}\egroup} %% lets you have sections that don't appear in the table of contents


%%%

%%Index (rev. December 11, 2013)
\usepackage[noautomatic,nonewpage]{imakeidx}


\makeindex[name=incipit,title=Index]
\indexsetup{level=\section,toclevel=section,noclearpage}

\usepackage[indentunit=8pt,rule=.5pt,columns=2]{idxlayout}


%%Table of Contents (rev. May 16, 2011)

%\usepackage{multicol}
%\usepackage{ifthen}
%\usepackage[toc]{multitoc}

%% General settings (rev. January 19, 2015)

\usepackage{ulem}

\usepackage[latin,english]{babel}
\usepackage{lettrine}


\usepackage{fontspec}

\setmainfont[Ligatures=TeX,BoldFont=MinionPro-Bold,ItalicFont=MinionPro-It, BoldItalicFont=MinionPro-BoldIt]{MinionPro-Regular-Modified.otf}



%% Style for translation line
\grechangestyle{translation}{\textnormal\selectfont}
\grechangestyle{annotation}{\fontsize{10}{10}\selectfont}
\grechangestyle{commentary}{\textnormal\selectfont}
\gresetcustosalteration{invisible}



%\grechangedim{annotationseparation}{0.1cm}{scalable}

%\GreLoadSpaceConf{smith-four}

\frenchspacing

\usepackage{indentfirst} %%%indents first line after a section

\usepackage{graphicx}
%\usepackage{tocloft}

%%Hyperref (rev. August 20, 2011)
%\usepackage[colorlinks=false,hyperindex=true,bookmarks=true]{hyperref}
\usepackage{hyperref}
\hypersetup{pdftitle={Graduale O.P. 2017}}
\hypersetup{pdfauthor={Order of Preachers (\today)}}
\hypersetup{pdfsubject={Liturgy}}
\hypersetup{pdfkeywords={Dominican, Liturgy, Order of Preachers, Dominican Rite, Liturgia Horarum, Divine Office}}

\newlength{\drop}



\begin{document}


%%%Initial Matter within Body (20 May 2011)
\raggedbottom
\grechangedim{maxbaroffsettextleft}{0 cm}{scalable}


\chapter[Fourth Sunday of Lent]{Graduale O.P.}
\section{Fourth Sunday of Lent}



\newpage

%%Combination
\subsection{Offertorium}  \greannotation{II} \index[Offertorium]{Laudate Dominum} \label{Laudate Dominum (Offertorium)} \grecommentary[0pt]{Ps 134:3, 6} \gresetinitiallines{1}  \gregorioscore{graduale-chants/of--laudate_dominum--dominican--id_5285}  \vspace{5pt} \par{Praise the Lord, for He is good; sing to His name, for He is sweet; whatsoever He pleased He has done in heaven and in earth.} \newpage

\subsection{Offertorium} Anno A (\emph{ad libitum}): \greannotation{IV} \index[Offertorium]{Illumina oculos meos} \label{Illumina oculos meos (Offertorium)} \grecommentary[5pt]{Ps 12:4, 5} \gresetinitiallines{1}  \gregorioscore{graduale-chants/of--illumina--dominican--id_5998}  \vspace{5pt} \par{Enlighten my eyes lest I fall into the sleep of death; lest my enemy say: ``I have prevailed against him."} \newpage

\subsection{Communio} Anno A (Quando legitur Evangelium de cæco nato): \greannotation{VI} \index[Communio]{Lutum fecit} \label{Lutum fecit (Communio)} \grecommentary[5pt]{Cf. Io 9:6, 11, 38} \gresetinitiallines{1}  \gregorioscore{graduale-chants/co--lutum_fecit--dominican--id_6341}  \vspace{5pt} \par{The Lord made clay of spittle, and anointed my eyes, and I went, and washed, and I saw; and I believed in God.} \newpage
\subsection{Communio} Anno B: \greannotation{IV} \index[Communio]{Ierusalem quæ ædificatur} \label{Ierusalem quæ ædificatur (Communio)} \grecommentary[0pt]{Ps 121:3, 4} \gresetinitiallines{1}  \gregorioscore{graduale-chants/co--ierusalem_quae--dominican--id_5394}  \vspace{5pt} \par{Jerusalem, which is built as a city, which is compact together; for there did the tribes go up, the tribes of the Lord, to praise Your name, O Lord.} \newpage

\subsection{Communio} Anno C (Quando legitur Evangelium de filio prodigo): \greannotation{VIII} \index[Communio]{Oportet te} \label{Oportet te (Communio)} \grecommentary[3pt]{Lc 15:32} \gresetinitiallines{1}  \gregorioscore{graduale-chants/co--oportet_te_fili--dominican--id_6842}  \vspace{5pt} \par{You ought to rejoice, my son, because Your brother was dead, and has come to life; he was lost, and is found.} \newpage

\end{document}
