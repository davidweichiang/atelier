\documentclass[11pt,twoside]{book}

%%Page Size (rev. 08/19/2016)
%\usepackage[inner=0.5in, outer=0.5in, top=0.5in, bottom=0.5in, papersize={6in,9in}, head=12pt, headheight=30pt, headsep=5pt]{geometry}
\usepackage[inner=0.5in, outer=0.5in, top=0.5in, bottom=0.3in, papersize={5.5in,8.5in}, head=12pt, headheight=30pt, headsep=5pt]{geometry}
%% width of textblock = 324 pt / 4.5in
%% A5 = 5.8 x 8.3 inches -- if papersize is A5, then margins should be [inner=0.75in, outer=0.55in, top=0.4in, bottom=0.4in]


%%Header (rev. 4/11/2011)
\usepackage{fancyhdr}
 \pagestyle{fancy}
\renewcommand{\chaptermark}[1]{\markboth{#1}{}}
\renewcommand{\sectionmark}[1]{\markright{\thesection\ #1}}
 \fancyhf{}
\fancyhead[LE,RO]{\thepage}
\fancyhead[CE]{Graduale O.P.}
\fancyhead[CO]{\leftmark}
 \fancypagestyle{plain}{ %
\fancyhf{} % remove everything
\renewcommand{\headrulewidth}{0pt} % remove lines as well
\renewcommand{\footrulewidth}{0pt}}



\usepackage[autocompile,allowdeprecated=false]{gregoriotex}
\usepackage{gregoriosyms}
\gresetgregoriofont[op]{greciliae}




%%Titles (rev. 9/4/2011) -- TOCLESS --- lets you have sections that don't appear in the table of contents

\setcounter{secnumdepth}{-1}

\usepackage[compact,nobottomtitles*]{titlesec}
\titlespacing*{\chapter}{0pt}{*0}{*1}
\titlespacing*{\section}{0pt}{*0}{*1}
\titlespacing*{\subsection}{0pt}{*0}{*0}
\titlespacing*{\subsubsection}{0pt}{10pt}{*0}
\titleformat{\chapter} {\normalfont\LARGE\sc\center}{\thechapter}{1em}{}
\titleformat{\section} {\normalfont\Large\sc\center}{\thesection}{1em}{}
\titleformat{\subsection} {\normalfont\large\sc\center}{\thesubsection}{1em}{}
\titleformat{\subsubsection}{\normalfont\normalsize\it}{\thesubsubsection}{1em}{}

\newcommand{\nocontentsline}[3]{}
\newcommand{\tocless}[2]{\bgroup\let\addcontentsline=\nocontentsline#1{#2}\egroup} %% lets you have sections that don't appear in the table of contents


%%%

%%Index (rev. December 11, 2013)
\usepackage[noautomatic,nonewpage]{imakeidx}


\makeindex[name=incipit,title=Index]
\indexsetup{level=\section,toclevel=section,noclearpage}

\usepackage[indentunit=8pt,rule=.5pt,columns=2]{idxlayout}


%%Table of Contents (rev. May 16, 2011)

%\usepackage{multicol}
%\usepackage{ifthen}
%\usepackage[toc]{multitoc}

%% General settings (rev. January 19, 2015)

\usepackage{ulem}

\usepackage[latin,english]{babel}
\usepackage{lettrine}


\usepackage{fontspec}

\setmainfont[Ligatures=TeX,BoldFont=MinionPro-Bold,ItalicFont=MinionPro-It, BoldItalicFont=MinionPro-BoldIt]{MinionPro-Regular-Modified.otf}



%% Style for translation line
\grechangestyle{translation}{\textnormal\selectfont}
\grechangestyle{annotation}{\fontsize{10}{10}\selectfont}
\grechangestyle{commentary}{\textnormal\selectfont}
\gresetcustosalteration{invisible}



%\grechangedim{annotationseparation}{0.1cm}{scalable}

%\GreLoadSpaceConf{smith-four}

\frenchspacing

\usepackage{indentfirst} %%%indents first line after a section

\usepackage{graphicx}
%\usepackage{tocloft}

%%Hyperref (rev. August 20, 2011)
%\usepackage[colorlinks=false,hyperindex=true,bookmarks=true]{hyperref}
\usepackage{hyperref}
\hypersetup{pdftitle={Graduale O.P. 2017}}
\hypersetup{pdfauthor={Order of Preachers (\today)}}
\hypersetup{pdfsubject={Liturgy}}
\hypersetup{pdfkeywords={Dominican, Liturgy, Order of Preachers, Dominican Rite, Liturgia Horarum, Divine Office}}

\newlength{\drop}



\begin{document}


%%%Initial Matter within Body (20 May 2011)
\raggedbottom
\grechangedim{maxbaroffsettextleft}{0 cm}{scalable}


\chapter[Third Sunday of Lent]{Graduale O.P.}
\section{Third Sunday of Lent}



\newpage

%%Combination
\subsection{Officium}  \greannotation{VII} \index[Officium]{Oculi mei} \label{Oculi mei (Officium)} \grecommentary[0pt]{Ps 24:15, 16; \Vbar. 1-2} \gresetinitiallines{1}  \gregorioscore{graduale-chants/in--oculi_mei--dominican--id_7217}  \vspace{5pt} \par{My eyes are towards the Lord, for He shall pluck my feet out of the snare. Look upon me, and have mercy on me, for I am alone and poor. \Vbar. To You, O Lord, have I lifted up my soul; in You, O my God, I put my trust, let me not be ashamed.} \newpage
\subsection{Officium} Vel: \greannotation{III} \index[Officium]{Dum sanctificatus fuero} \label{Dum sanctificatus fuero (Officium)} \grecommentary[7pt]{Ezech. 36:23, 24, 25, 26; \Vbar. Ps 33:2 [GR]} \gresetinitiallines{1}  \gregorioscore{graduale-chants/in--dum_sanctificatus--vatican-op}  \vspace{5pt} \par{When I vindicate my holiness through you, I will gather you from all lands, and I will sprinkle clean water upon you, and you shall be cleansed from all your filthiness; and I will give you a new Spirit. \Vbar. I will bless the Lord at all times; his praise shall continually be in my mouth.} \newpage


\subsection{Offertorium}  \greannotation{IV} \index[Offertorium]{Iustitiæ Domini} \label{Iustitiæ Domini (Offertorium)} \grecommentary[0pt]{Ps 18:9, 11, 12} \gresetinitiallines{1}  \gregorioscore{graduale-chants/of--iustitiae_domini--dominican--id_6109}  \vspace{5pt} \par{The precepts of the Lord are right, rejoicing hearts, and His judgments are sweeter than honey and the honeycomb, for Your servant keeps them.} \newpage

\subsection{Communio} Quando legitur Evangelium de Samaritana: \greannotation{VII} \index[Communio]{Qui biberit aquam} \label{Qui biberit aquam (Communio)} \grecommentary[0pt]{Io 4:13, 14} \gresetinitiallines{1}  \gregorioscore{graduale-chants/co--qui_biberit--dominican--id_4868}  \vspace{5pt} \par{He who drinks of the water that I will give him, says the Lord, "it shall become in him a fountain of water, springing up unto life everlasting."} \newpage


\subsection{Communio} Quando legitur aliud Evangelium: \greannotation{I} \index[Communio]{Passer invenit} \label{Passer invenit (Communio)} \grecommentary[0pt]{Ps 83:4, 5} \gresetinitiallines{1}  \gregorioscore{graduale-chants/co--passer_invenit--dominican--id_6674}  \vspace{5pt} \par{The sparrow has found herself a house, and the turtle a nest where she may lay her young ones: Your altars, O Lord of Hosts, my King, and my God. Blessed are they who dwell in Your house; they shall praise You forever and ever.} \newpage


\end{document}
