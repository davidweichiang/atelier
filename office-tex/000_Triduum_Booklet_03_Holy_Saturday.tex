\documentclass[11pt,twoside]{book}

%%Page Size (rev. 08/19/2016)
%\usepackage[inner=0.5in, outer=0.5in, top=0.5in, bottom=0.5in, papersize={6in,9in}, head=12pt, headheight=30pt, headsep=5pt]{geometry}
\usepackage[inner=0.5in, outer=0.5in, top=0.5in, bottom=0.5in, papersize={5.5in,8.5in}, head=12pt, headheight=30pt, headsep=5pt]{geometry}
%% width of textblock = 324 pt / 4.5in
%% A5 = 5.8 x 8.3 inches -- if papersize is A5, then margins should be [inner=0.75in, outer=0.55in, top=0.4in, bottom=0.4in]


%%Header (rev. 4/11/2011)
\usepackage{fancyhdr}
 \pagestyle{fancy}
\renewcommand{\chaptermark}[1]{\markboth{#1}{}}
\renewcommand{\sectionmark}[1]{\markright{#1}}
 \fancyhf{}
\fancyhead[LE,RO]{\thepage}
\fancyhead[CE]{\leftmark}
\fancyhead[CO]{\rightmark}
 \fancypagestyle{plain}{ %
\fancyhf{} % remove everything
\renewcommand{\headrulewidth}{0pt} % remove lines as well
\renewcommand{\footrulewidth}{0pt}}



\usepackage[autocompile,allowdeprecated=false]{gregoriotex}
\usepackage{gregoriosyms}
\gresetgregoriofont[op]{greciliae}




%%Titles (rev. 9/4/2011) -- TOCLESS --- lets you have sections that don't appear in the table of contents

\setcounter{secnumdepth}{-1}

\usepackage[compact,nobottomtitles*]{titlesec}
\titlespacing*{\chapter}{0pt}{-30pt}{0pt}
\titlespacing*{\section}{0pt}{*0}{*1}
\titlespacing*{\subsubsection}{0pt}{*0}{*0}
\titlespacing*{\subsubsubsection}{0pt}{10pt}{*0}
\titleformat{\part} {\normalfont\Huge\sc\center}{\thechapter}{1em}{}
\titleformat{\chapter} {\normalfont\LARGE\sc\center}{\thechapter}{1em}{}
\titleformat{\section} {\normalfont\Large\sc\center}{\thesection}{1em}{}
\titleformat{\subsection} {\normalfont\Large\sc\center}{\thesubsubsection}{1em}{}
\titleformat{\subsubsection}{\normalfont\large\sc\center}{\thesubsubsubsection}{1em}{}
\titleformat{\paragraph}{\normalfont\normalsize\sc\center}{\thesubsubsubsection}{1em}{}

\newcommand{\nocontentsline}[3]{}
\newcommand{\tocless}[2]{\bgroup\let\addcontentsline=\nocontentsline#1{#2}\egroup} %% lets you have sections that don't appear in the table of contents


%%%

%%Index (rev. December 11, 2013)
\usepackage[noautomatic,nonewpage]{imakeidx}


\makeindex[name=incipit,title=Index]
\indexsetup{level=\section,toclevel=section,noclearpage}

\usepackage[indentunit=8pt,rule=.5pt,columns=2]{idxlayout}


%%Table of Contents (rev. May 16, 2011)

%\usepackage{multicol}
%\usepackage{ifthen}
%\usepackage[toc]{multitoc}

%% General settings (rev. January 19, 2015)

\usepackage[normalem]{ulem}

\usepackage[latin,english]{babel}
\usepackage{lettrine}

\usepackage{paracol}

\usepackage{fontspec}

\setmainfont[Ligatures=TeX,BoldFont=MinionPro-Bold,ItalicFont=MinionPro-It, BoldItalicFont=MinionPro-BoldIt]{MinionPro-Regular-Modified.otf}


%% Style for translation line
\grechangestyle{translation}{\fontsize{10}{10}\it\selectfont}
\grechangestyle{annotation}{\fontsize{10}{10}\selectfont}
\grechangestyle{commentary}{\textnormal\selectfont}
\gresetcustosalteration{invisible}

%\grechangedim{annotationseparation}{0.1cm}{scalable}

%\GreLoadSpaceConf{smith-four}

\frenchspacing

\usepackage{indentfirst} %%%indents first line after a section

\usepackage{graphicx}
%\usepackage{tocloft}

%%Hyperref (rev. August 20, 2011)
%\usepackage[colorlinks=false,hyperindex=true,bookmarks=true]{hyperref}
\usepackage{hyperref}
\hypersetup{pdftitle={Vesperale O.P. 2016}}
\hypersetup{pdfauthor={Order of Preachers}}
\hypersetup{pdfsubject={Liturgy}}
\hypersetup{pdfkeywords={Dominican, Liturgy, Order of Preachers, Dominican Rite, Liturgia Horarum, Divine Office}}

\newlength{\drop}



\begin{document}


\raggedbottom

\newcommand{\lectio}[3]{%
  \makebox[0pt][l]{#1}%
  \makebox[\textwidth][c]{#2}%
  \makebox[0pt][r]{\normalsize{\textnormal{#3}}}}


%%Combination
\chapter{Holy Saturday}
\section{Office of Readings}
              \par \emph{At the Prior’s signal all stand, face the high altar and silently make the Sign of the Cross.}
 \subsubsection{Invitatory Psalm}   \index[Psalmus]{Invitatory Psalm} \label{Invitatory Psalm (Psalmus)}         \vspace{5pt} \par \emph{Ant.} Come, let us worship Christ, who for our sake suffered death and was buried.

\subsubsection{Psalm 95}	\paragraph{A call to praise God}




\vspace{5pt}
\emph{Encourage each other daily while it is still today (Hebrews 3:13).}
\vspace{5pt}

Come, let us sing to the Lord 

 \noindent  and shout with joy to the Rock who saves us.

\noindent Let us approach him with praise and thanksgiving 

\noindent   and sing joyful songs to the Lord.

\vspace{5pt}
\emph{Ant.} Come, let us worship Christ, who for our sake suffered death and was buried.
\vspace{5pt}

The Lord is God, the mighty God, 

\noindent   the great king over all the gods.

\noindent He holds in his hands the depths of the earth 

\noindent   and the highest mountains as well.

\noindent He made the sea; it belongs to him, 

\noindent   the dry land, too, for it was formed by his hands.

\vspace{5pt}
\emph{Ant.} Come, let us worship Christ, who for our sake suffered death and was buried.
\vspace{5pt}

Come, then, let us bow down and worship, 

\noindent   bending the knee before the Lord, our maker.

\noindent For he is our God and we are his people, 

\noindent   the flock he shepherds.

\vspace{5pt}
\emph{Ant.} Come, let us worship Christ, who for our sake suffered death and was buried.
\vspace{5pt}

Today, listen to the voice of the Lord:

\noindent Do not grow stubborn, as your fathers did in the wilderness, 

\noindent when at Meriba and Massah they challenged me and provoked me, 

\noindent Although they had seen all of my works.

\vspace{5pt}
\emph{Ant.} Come, let us worship Christ, who for our sake suffered death and was buried.
\vspace{5pt}

Forty years I endured that generation. 

\noindent I said, “They are a people whose hearts go astray

\noindent   and they do not know my ways.”

\noindent So I swore in my anger, 

\noindent   “They shall not enter into my rest.”

\vspace{5pt}
\emph{Ant.} Come, let us worship Christ, who for our sake suffered death and was buried.
\vspace{5pt}

Glory to the Father, and to the Son, 

\noindent and to the Holy Spirit:

\noindent as it was in the beginning, is now, 

\noindent and will be for ever. Amen.

\vspace{5pt}
\emph{Ant.} Come, let us worship Christ, who for our sake suffered death and was buried.
 \subsubsection{Hymn}   \index[Hymn]{Hymn} \label{Hymn (Hymn)}         Thirty years among us dwelling,
His appointed time fulfilled,
Born for this he meets his passion,
This indeed he freely willed;
On the cross the Lamb is lifted
Where his lifeblood shall be spilled.

2. He endured the nails, the spitting,
Vinegar and spear and reed;
From that holy body broken,
Blood and water forth proceed:
Earth and stars and sky and ocean,
By that flood from stain are freed.

3. Faithful Cross! Above all others,
One and only noble tree!
None in foliage, none in blossom,
None in fruit, your peer may be;
Precious wood and precious fast'ning,
Precious weight upheld in plea.

4. Bend your boughs, O Tree of glory!
Your relaxing sinews bend;
For a while the ancient rigor,
That your birth bestowed, suspend;
And the King of heav'nly beauty
On your bosom gently tend.

5. You alone were counted worthy
This world's ransom to sustain,
That a shipwrecked race forever
Might the port of refuge gain;
With the sacred blood anointed
Of the Lamb for sinners slain.

6. Praise and honor to the Father;
Praise and honor to the Son,
Praise and honor to the Spirit,
Ever three and ever one;
Consubstantial, coeternal,
While unending ages run.
Amen.
 \subsubsection{Antiphon 1}   \index[Antiphon]{Antiphon 1} \label{Antiphon 1 (Antiphon)}         In peace, I will lie down and sleep.
 \subsubsection{Psalm 4} \paragraph{Thanksgiving}  \index[Psalmus]{Psalm 4} \label{Psalm 4 (Psalmus)} \emph{The resurrection of Christ was God’s supreme and wholly marvelous work (Saint Augustine).}        \par \vspace{5pt} \noindent  When I call, answer me, O God of justice;~\GreStar{}~\nopagebreak

from anguish you released me, have mercy and hear me!

\noindent O men, how long will your hearts be closed,~\GreStar{}~\nopagebreak

will you love what is futile and seek what is false?

\noindent It is the Lord who grants favors to those whom he loves;~\GreStar{}~\nopagebreak

the Lord hears me whenever I call him.

\noindent Fear him; do not sin: ponder on your bed and be still.~\GreStar{}~\nopagebreak

Make justice your sacrifice, and trust in the Lord.

\noindent ``What can bring us happiness?" many say.~\GreStar{}~\nopagebreak

Let the light of your face shine on us, O Lord.

\noindent You have put into my heart a greater joy~\GreStar{}~\nopagebreak

than they have from abundance of corn and new wine.

\noindent I will lie down in peace and sleep comes at once~\GreStar{}~\nopagebreak

for you alone, Lord, make me dwell in safety.

\noindent Glory to the Father, and to the Son,~\GreStar{}~\nopagebreak

and to the Holy Spirit:

\noindent as it was in the beginning, is now,~\GreStar{}~\nopagebreak

and will be for ever. Amen.

 \subsubsection{Antiphon 2}   \index[Antiphon]{Antiphon 2} \label{Antiphon 2 (Antiphon)}         My body shall rest in hope.
 \subsubsection{Psalm 16} \paragraph{God is my portion, my heritage}  \index[Psalmus]{Psalm 16} \label{Psalm 16 (Psalmus)} \emph{The Father raised up Jesus from the dead and broke the bonds of death (Acts 2:24).}        \par \vspace{5pt} \noindent Preserve me, God, I take refuge in you.~†~\nopagebreak

I say to the Lord: “You are my God. ~\GreStar{}~\nopagebreak

My happiness lies in you alone.”

\noindent He has put into my heart a marvelous love ~\GreStar{}~\nopagebreak

for the faithful ones who dwell in his land.

\noindent Those who choose other gods increase their sorrows.~†~\nopagebreak

Never will I offer their offerings of blood. ~\GreStar{}~\nopagebreak

Never will I take their name upon my lips.

\noindent O Lord, it is you who are my portion and cup; ~\GreStar{}~\nopagebreak

it is you yourself who are my prize.

\noindent The lot marked out for me is my delight: ~\GreStar{}~\nopagebreak

welcome indeed the heritage that falls to me!

\noindent I will bless the Lord who gives me counsel, ~\GreStar{}~\nopagebreak

who even at night directs my heart.

\noindent I keep the Lord ever in my sight: ~\GreStar{}~\nopagebreak

since he is at my right hand, I shall stand firm.

\noindent And so my heart rejoices, my soul is glad; ~\GreStar{}~\nopagebreak

even my body shall rest in safety.

\noindent For you will not leave my soul among the dead, ~\GreStar{}~\nopagebreak

nor let your beloved know decay.

\noindent You will show me the path of life,~†~\nopagebreak

the fullness of joy in your presence, ~\GreStar{}~\nopagebreak

at your right hand happiness for ever.

\noindent Glory to the Father, and to the Son,~\GreStar{}~\nopagebreak

and to the Holy Spirit:

\noindent as it was in the beginning, is now,~\GreStar{}~\nopagebreak

and will be for ever. Amen.

 \subsubsection{Antiphon 3}   \index[Antiphon]{Antiphon 3} \label{Antiphon 3 (Antiphon)}         Lift high the ancient portals. The King of glory enters.
 \subsubsection{Psalm 24} \paragraph{The Lord’s entry into his temple}  \index[Psalmus]{Psalm 24} \label{Psalm 24 (Psalmus)} \emph{Christ opened heaven for us in the humanity he assumed (Saint Irenaeus).}        \par \vspace{5pt} \noindent The Lord’s is the earth and its fullness,~$\star$~\nopagebreak

the world and all its peoples.

\noindent It is he who set it on the seas;~$\star$~\nopagebreak

on the waters he made it firm.

\noindent Who shall climb the mountain of the Lord?~$\star$~\nopagebreak

Who shall stand in his holy place?

\noindent The man with clean hands and pure heart,~†~\nopagebreak

who desires not worthless things,~$\star$~\nopagebreak

who has not sworn so as to deceive his neighbor.

\noindent He shall receive blessings from the Lord~$\star$~\nopagebreak

and reward from the God who saves him.

\noindent Such are the men who seek him,~$\star$~\nopagebreak

seek the face of the God of Jacob.

\noindent O gates, lift high your heads;~†~\nopagebreak

grow higher, ancient doors.~$\star$~\nopagebreak

Let him enter, the king of glory!

\noindent Who is the king of glory?~†~\nopagebreak

The Lord, the mighty, the valiant,~$\star$~\nopagebreak

the Lord, the valiant in war.

\noindent O gates, lift high your heads;~†~\nopagebreak

grow higher, ancient doors.~$\star$~\nopagebreak

Let him enter, the king of glory!

\noindent Who is he, the king of glory?~†~\nopagebreak

He, the Lord of armies,~$\star$~\nopagebreak

he is the king of glory.

\noindent Glory to the Father, and to the Son,~$\star$~\nopagebreak

and to the Holy Spirit:

\noindent as it was in the beginning, is now,~$\star$~\nopagebreak

and will be for ever. Amen.
                 \par \vspace{5pt} \Vbar. Take up my cause and rescue me. \par \Rbar. Be true to your word, give me life. \newpage
 \subsubsection{First Reading}             From the letter to the Hebrews \hfill 4:1-13

\paragraph{Let us strive to enter the Lord’s rest}

\lettrine[lines=3]{W}{}hile the promise of entrance into his rest still holds, we ought to be fearful of disobeying lest any one of you be judged to have lost his chance of entering. We have indeed heard the good news, as they did. But the word which they heard did not profit them, for they did not receive it in faith.

It is we who have believed who enter into that rest, just as God said:

\vspace{5pt}
 “Then I swore in my anger,\par
    ‘They shall never enter into my rest.’”
\vspace{5pt}
Yet God’s work was finished when he created the world, for in reference to the seventh day Scripture somewhere says, “And God rested from all his work on the seventh day”; and again, in the place we have referred to, God says, “They shall never enter into my rest.”

Therefore, since it remains for some to enter, and those to whom it was first announced did not enter because of unbelief, God once more set a day, “today,” when long afterward he spoke through David the words we have quoted:

\vspace{5pt}
   “Today if you should hear his voice,\par
      harden not your hearts.”
\vspace{5pt}

Now if Joshua had led them into the place of rest, God would not have spoken afterward of another day. Therefore, a sabbath rest still remains for the people of God. And he who enters into God’s rest, rests from his own work as God did from his. Let us strive to enter into that rest, so that no one may fall, in imitation of the example of Israel’s unbelief.

Indeed, God’s word is living and effective, sharper than any two-edged sword. It penetrates and divides soul and spirit, joints and marrow; it judges the reflections and thoughts of the heart. Nothing is concealed from him; all lies bare and exposed to the eyes of him to whom we must render an account.


\subsubsection{Responsory}
\hfill See Matthew 27:66, 60, 62

They buried the Lord and sealed the tomb\par
by rolling a large stone in front of it.\par
– They stationed soldiers there to guard it.\par
\vspace{5pt}
The chief priests asked Pilate for a guard.\par
– They stationed soldiers there to guard it.

 \subsubsection{Second Reading}             From an ancient homily on Holy Saturday

\par \hfill(PG 43, 439, 451, 462-463)

\paragraph{The Lord descends into hell}

\lettrine[lines=3]{S}{}omething strange is happening—there is a great silence on earth today, a great silence and stillness. The whole earth keeps silence because the King is asleep. The earth trembled and is still because God has fallen asleep in the flesh and he has raised up all who have slept ever since the world began. God has died in the flesh and hell trembles with fear.

He has gone to search for our first parent, as for a lost sheep. Greatly desiring to visit those who live in darkness and in the shadow of death, he has gone to free from sorrow the captives Adam and Eve, he who is both God and the son of Eve. The Lord approached them bearing the cross, the weapon that had won him the victory. At the sight of him Adam, the first man he had created, struck his breast in terror and cried out to everyone: “My Lord be with you all.” Christ answered him: “And with your spirit.” He took him by the hand and raised him up, saying: “Awake, O sleeper, and rise from the dead, and Christ will give you light.”

I am your God, who for your sake have become your son. Out of love for you and for your descendants I now by my own authority command all who are held in bondage to come forth, all who are in darkness to be enlightened, all who are sleeping to arise. I order you, O sleeper, to awake. I did not create you to be held a prisoner in hell. Rise from the dead, for I am the life of the dead. Rise up, work of my hands, you who were created in my image. Rise, let us leave this place, for you are in me and I am in you; together we form only one person and we cannot be separated.

For your sake I, your God, became your son; I, the Lord, took the form of a slave; I, whose home is above the heavens, descended to the earth and beneath the earth. For your sake, for the sake of man, I became like a man without help, free among the dead. For the sake of you, who left a garden, I was betrayed to the Jews in a garden, and I was crucified in a garden.

See on my face the spittle I received in order to restore to you the life I once breathed into you. See there the marks of the blows I received in order to refashion your warped nature in my image. On my back see the marks of the scourging I endured to remove the burden of sin that weighs upon your back. See my hands, nailed firmly to a tree, for you who once wickedly stretched out your hand to a tree.

I slept on the cross and a sword pierced my side for you who slept in paradise and brought forth Eve from your side. My side has healed the pain in yours. My sleep will rouse you from your sleep in hell. The sword that pierced me has sheathed the sword that was turned against you.

Rise, let us leave this place. The enemy led you out of the earthly paradise. I will not restore you to that paradise, but I will enthrone you in heaven. I forbade you the tree that was only a symbol of life, but see, I who am life itself am now one with you. I appointed cherubim to guard you as slaves are guarded, but now I make them worship you as God. The throne formed by cherubim awaits you, its bearers swift and eager. The bridal chamber is adorned, the banquet is ready, the eternal dwelling places are prepared, the treasure houses of all good things lie open. The kingdom of heaven has been prepared for you from all eternity.

\subsubsection{Responsory}

\hfill 

Our shepherd, the source of the water of life, has died.\par
The sun was darkened when he passed away.\par
But now man’s captor is made captive.\par
– This is the day when our Savior broke through the gates of death.

\vspace{5pt}

He has destroyed the barricades of hell,\par
overthrown the sovereignty of the devil.\par
– This is the day when our Savior broke through the gates of death.
    \newpage
 \subsubsection{Oratio Ieremiæ}   \index[Oratio Ieremiæ]{Oratio Ieremiæ} \label{Oratio Ieremiæ (Oratio Ieremiæ)}         Remember, Lord, what has happened to us,
    pay attention, and see our disgrace:
Our heritage is turned over to strangers,
    our homes, to foreigners.
We have become orphans, without fathers;
    our mothers are like widows.
We pay money to drink our own water,
    our own wood comes at a price.
With a yoke on our necks, we are driven;
    we are worn out, but allowed no rest.

We extended a hand to Egypt and Assyria,
    to satisfy our need of bread.
Our ancestors, who sinned, are no more;
    but now we bear their guilt.
Servants rule over us,
    with no one to tear us from their hands.
We risk our lives just to get bread,
    exposed to the desert heat;
Our skin heats up like an oven,
    from the searing blasts of famine.

Women are raped in Zion,
    young women in the cities of Judah;
Princes have been hanged by them,
    elders shown no respect.
Young men carry millstones,
    boys stagger under loads of wood;
The elders have abandoned the gate,
    the young men their music.

The joy of our hearts has ceased,
    dancing has turned into mourning;
The crown has fallen from our head:
    woe to us that we sinned!
Because of this our hearts grow sick,
    at this our eyes grow dim:
Because of Mount Zion, lying desolate,
    and the jackals roaming there!

But you, Lord, are enthroned forever;
    your throne stands from age to age.
Why have you utterly forgotten us,
    forsaken us for so long?
Bring us back to you, Lord, that we may return:
    renew our days as of old.
For now you have indeed rejected us
    and utterly turned your wrath against us.
\section{Morning Prayer}
 \subsubsection{Antiphon 1}   \index[Antiphon]{Antiphon 1} \label{Antiphon 1 (Antiphon)}         Though sinless, the Lord has been put to death. The world is in mourning as for an only son.
 \subsubsection{Psalm 64} \paragraph{Prayer for help against enemies}  \index[Psalmus]{Psalm 64} \label{Psalm 64 (Psalmus)} \emph{This psalm commemorates most particularly our Lord’s passion (Saint Augustine).}        \par \vspace{5pt} \noindent Hear my voice, O God, as I complain,~\GreStar{}~\nopagebreak

guard my life from dread of the foe.

\noindent Hide me from the band of the wicked,~\GreStar{}~\nopagebreak

from the throng of those who do evil.

\noindent They sharpen their tongues like swords;~\GreStar{}~\nopagebreak

they aim bitter words like arrows

\noindent to shoot at the innocent from ambush,~\GreStar{}~\nopagebreak

shooting suddenly and recklessly.

\noindent They scheme their evil course;~\GreStar{}~\nopagebreak

they conspire to lay secret snares.

\noindent They say: “Who will see us?~\GreStar{}~\nopagebreak

Who can search out our crimes?”

\noindent He will search who searches the mind~\GreStar{}~\nopagebreak

and knows the depth of the heart.

\noindent God has shot them with his arrow~\GreStar{}~\nopagebreak

and dealt them sudden wounds.

\noindent Their own tongue has brought them to ruin~\GreStar{}~\nopagebreak

and all who see them mock.

\noindent Then will all men fear;~†~\nopagebreak

they will tell what God has done.~\GreStar{}~\nopagebreak

They will understand God’s deeds.

\noindent The just will rejoice in the Lord~†~\nopagebreak

and fly to him for refuge.~\GreStar{}~\nopagebreak

All the upright hearts will glory.

\noindent Glory to the Father, and to the Son,~\GreStar{}~\nopagebreak

and to the Holy Spirit:

\noindent as it was in the beginning, is now,~\GreStar{}~\nopagebreak

and will be for ever. Amen.

 \subsubsection{Antiphon 2}   \index[Antiphon]{Antiphon 2} \label{Antiphon 2 (Antiphon)}         From the jaws of hell, Lord, rescue my soul.
 \subsubsection{Canticle: Isaiah 38:10--14, 17b--20} \paragraph{Anguish of a dying man and joy in his restoration}  \index[Psalmus]{Canticle: Isaiah 38:10--14, 17b--20} \label{Canticle: Isaiah 38:10--14, 17b--20 (Psalmus)} \emph{I was living, I was dead . . . and I hold the keys of death (Revelation 1:17--18).}        \par \vspace{5pt} \noindent Once I said,~$\star$~\nopagebreak

“In the noontime of life I must depart!

\noindent To the gates of the nether world I shall be consigned~$\star$~\nopagebreak

for the rest of my years.”

\noindent I said, “I shall see the Lord no more~$\star$~\nopagebreak

in the land of the living.

\noindent No longer shall I behold my fellow men~$\star$~\nopagebreak

among those who dwell in the world.”

\noindent My dwelling, like a shepherd’s tent,~$\star$~\nopagebreak

is struck down and borne away from me;

\noindent you have folded up my life, like a weaver~$\star$~\nopagebreak

who severs the last thread.

\noindent Day and night you give me over to torment;~$\star$~\nopagebreak

I cry out until the dawn.

\noindent Like a lion he breaks all my bones;~$\star$~\nopagebreak

day and night you give me over to torment.

\noindent Like a swallow I utter shrill cries;~$\star$~\nopagebreak

I moan like a dove.

\noindent My eyes grow weak, gazing heaven-ward:~$\star$~\nopagebreak

O Lord, I am in straits; be my surety!

\noindent You have preserved my life~$\star$~\nopagebreak

from the pit of destruction,

\noindent When you cast behind your back~$\star$~\nopagebreak

all my sins.

\noindent For it is not the nether world that gives you thanks,~$\star$~\nopagebreak

nor death that praises you;

\noindent Neither do those who go down into the pit~$\star$~\nopagebreak

await your kindness.

\noindent The living, the living give you thanks,~$\star$~\nopagebreak

as I do today.

\noindent Fathers declare to their sons,~$\star$~\nopagebreak

O God, your faithfulness.

\noindent The Lord is our savior;~$\star$~\nopagebreak

we shall sing to stringed instruments

\noindent In the house of the Lord~$\star$~\nopagebreak

all the days of our life.

\noindent Glory to the Father, and to the Son,~$\star$~\nopagebreak

and to the Holy Spirit:

\noindent as it was in the beginning, is now,~$\star$~\nopagebreak

and will be for ever. Amen.
 \subsubsection{Antiphon 3}   \index[Antiphon]{Antiphon 3} \label{Antiphon 3 (Antiphon)}         I was dead, but now I live for ever, and I hold the keys of death and of hell.
 \subsubsection{Psalm 150} \paragraph{Praise the Lord}  \index[Psalmus]{Psalm 150} \label{Psalm 150 (Psalmus)} \emph{Let mind and heart be in your song: this is to glorify God with your whole self (Hesychius).}        \par \vspace{5pt} \noindent Praise God in his holy place,~\GreStar{}~\nopagebreak

praise him in his mighty heavens.

\noindent Praise him for his powerful deeds,~\GreStar{}~\nopagebreak

praise his surpassing greatness.

\noindent O praise him with sound of trumpet,~\GreStar{}~\nopagebreak

praise him with lute and harp.

\noindent Praise him with timbrel and dance,~\GreStar{}~\nopagebreak

praise him with strings and pipes.

\noindent O praise him with resounding cymbals,~\GreStar{}~\nopagebreak

praise him with clashing of cymbals.

\noindent Let everything that lives and that breathes~\GreStar{}~\nopagebreak

give praise to the Lord.

\noindent Glory to the Father, and to the Son,~\GreStar{}~\nopagebreak

and to the Holy Spirit:

\noindent as it was in the beginning, is now,~\GreStar{}~\nopagebreak

and will be for ever. Amen.

 \subsubsection{Reading}     \hfill Hos 5:15b--16:2        \lettrine[lines=2]{T}{}hus says the Lord, \\In their affliction, they shall look for me:\\
   “Come, let us return to the Lord,\\
For it is he who has rent, but he will heal us;\\
   he has struck us, but he will bind our wounds.\\
He will revive us after two days;\\
   on the third day he will raise us up,\\
   to live in his presence.”

 \subsubsection{Responsory}  \greannotation{VI} \index[Responsory]{For our sake} \label{For our sake (Responsory)} \grecommentary[0pt]{Ph 2:8; \Vbar. 9} \gresetinitiallines{1} \grechangestyle{initial}{\fontsize{36}{36}\selectfont} \grechangedim{maxbaroffsettextleft@nobar}{12 cm}{scalable} \grechangedim{spaceabovelines}{0.5cm}{scalable} \gresetlyriccentering{syllable}   \gregorioscore{chants/misc.christus_factus--english--holy_saturday}
 \subsubsection{Gospel Canticle}   \index[Antiphon]{Gospel Canticle} \label{Gospel Canticle (Antiphon)}         Save us, O Savior of the world. On the cross you redeemed us by the shedding of your blood; we cry out for your help, O God.
 \subsubsection{Canticle of Zechariah (Luke 1:68--79)} \paragraph{The Messiah and his forerunner}  \index[Psalmus]{Canticle of Zechariah (Luke 1:68--79)} \label{Canticle of Zechariah (Luke 1:68--79) (Psalmus)}         \par \vspace{5pt} \lettrine[loversize=0.15,lines=2]{B}{}lessed be the Lord, the God of Israel; ~$\star$~\nopagebreak

\hspace{2pt} he has come to his people and set them free.

\noindent He has raised up for us a mighty savior,

born of the house of his servant David.

\noindent Through his holy prophets he promised of old

  that he would save us from our enemies,~$\star$~\nopagebreak

  from the hands of all who hate us.

\noindent He promised to show mercy to our fathers~$\star$~\nopagebreak


and to remember his holy covenant.

\noindent This was the oath he swore to our father Abraham:~$\star$~\nopagebreak

to set us free from the hands of our enemies,

\noindent free to worship him without fear,~$\star$~\nopagebreak

holy and righteous in his sight

   all the days of our life.

\noindent You, my child, shall be called the prophet of the Most High;~$\star$~\nopagebreak

for you will go before the Lord to prepare his way,

\noindent to give his people knowledge of salvation~$\star$~\nopagebreak


by the forgiveness of their sins.

\noindent In the tender compassion of our God~$\star$~\nopagebreak


the dawn from on high shall break upon us,

\noindent to shine on those who dwell in darkness and the shadow of death,~$\star$~\nopagebreak

and to guide our feet into the way of peace.

\noindent Glory to the Father, and to the Son,~$\star$~\nopagebreak

and to the Holy Spirit:

\noindent as it was in the beginning, is now,~$\star$~\nopagebreak

and will be for ever. Amen.
    
 \subsubsection{Preces} \emph{For the \emph{Preces} all stand facing the altar. Two friars (A) stand by the foot of the high altar, and two friars (B) stand in the middle of the choir.} \par \vspace{5pt} \greannotation{IV} \index[Preces]{Holy Saturday Preces} \label{Holy Saturday Preces (Preces)} \grecommentary[0pt]{} \gresetinitiallines{1} \grechangestyle{initial}{\fontsize{36}{36}\selectfont} \grechangedim{maxbaroffsettextleft@nobar}{12 cm}{scalable} \grechangedim{spaceabovelines}{0.5cm}{scalable} \gresetlyriccentering{syllable}   \gregorioscore{chants/misc.versus_litanici_in_cantu_sabbato_sancto--dominican--english--simple--rubrics}
 \subsubsection{The Lord's Prayer} \emph{When the \emph{Preces} are concluded all remain standing in silence until the Prior begins the Our Father.} \par \vspace{5pt}  \index[Varia]{The Lord's Prayer} \label{The Lord's Prayer (Varia)}         Our Father, who art in heaven,

hallowed be thy name;

thy kingdom come;

thy will be done on earth

as it is in heaven.

Give us this day our daily bread;

and forgive us our trespasses

as we forgive those who trespass against us;

and lead us not into temptation,

but deliver us from evil.
 \subsubsection{Concluding Prayer} \emph{When the Our Father is concluded the choir turns and kneels for the concluding collect.} \par \vspace{5pt}  \index[Concluding Prayer]{Good Friday} \label{Good Friday (Concluding Prayer)}         \lettrine[lines=3]{A}{}ll-powerful and ever-living God,
your only Son went down among the dead
and rose again in glory.
In your goodness
raise up your faithful people,
buried with him in baptism,
to be one with him
in the everlasting life of heaven,
where he lives and reigns with you and the Holy Spirit,
one God, for ever and ever. \par \Rbar. Amen.
              \par \vspace{10pt} \emph{All depart in silence.}

 \end{document}
