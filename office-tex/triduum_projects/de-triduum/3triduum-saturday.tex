\documentclass[11pt,twoside]{book}

%%Page Size (rev. 08/19/2016)
%\usepackage[inner=0.5in, outer=0.5in, top=0.5in, bottom=0.5in, papersize={6in,9in}, head=12pt, headheight=30pt, headsep=5pt]{geometry}
\usepackage[inner=0.5in, outer=0.5in, top=0.5in, bottom=0.5in, papersize={5.5in,8.5in}, head=12pt, headheight=30pt, headsep=5pt]{geometry}
%% width of textblock = 324 pt / 4.5in
%% A5 = 5.8 x 8.3 inches -- if papersize is A5, then margins should be [inner=0.75in, outer=0.55in, top=0.4in, bottom=0.4in]


%%Header (rev. 4/11/2011)
\usepackage{fancyhdr}
 \pagestyle{fancy}
\renewcommand{\chaptermark}[1]{\markboth{#1}{}}
\renewcommand{\sectionmark}[1]{\markright{#1}}
 \fancyhf{}
\fancyhead[LE,RO]{\thepage}
\fancyhead[CE]{\leftmark}
\fancyhead[CO]{\rightmark}
 \fancypagestyle{plain}{ %
\fancyhf{} % remove everything
\renewcommand{\headrulewidth}{0pt} % remove lines as well
\renewcommand{\footrulewidth}{0pt}}



\usepackage[autocompile,allowdeprecated=false]{gregoriotex}
\usepackage{gregoriosyms}
\gresetgregoriofont[op]{greciliae}

%%Titles (rev. 9/4/2011) -- TOCLESS --- lets you have sections that don't appear in the table of contents

\setcounter{secnumdepth}{-1}

\usepackage[compact,nobottomtitles*]{titlesec}
\titlespacing*{\chapter}{0pt}{-30pt}{0pt}
\titlespacing*{\section}{0pt}{*0}{*1}
\titlespacing*{\subsubsection}{0pt}{*0}{*0}
\titlespacing*{\subsubsubsection}{0pt}{10pt}{*0}
\titleformat{\part} {\normalfont\Huge\sc\center}{\thechapter}{1em}{}
\titleformat{\chapter} {\normalfont\LARGE\sc\center}{\thechapter}{1em}{}
\titleformat{\section} {\normalfont\Large\sc\center}{\thesection}{1em}{}
\titleformat{\subsection} {\normalfont\Large\sc\center}{\thesubsubsection}{1em}{}
\titleformat{\subsubsection}{\normalfont\large\sc\center}{\thesubsubsubsection}{1em}{}
\titleformat{\paragraph}{\normalfont\normalsize\sc\center}{\thesubsubsubsection}{1em}{}

\newcommand{\nocontentsline}[3]{}
\newcommand{\tocless}[2]{\bgroup\let\addcontentsline=\nocontentsline#1{#2}\egroup} %% lets you have sections that don't appear in the table of contents


%%%

%%Index (rev. December 11, 2013)
\usepackage[noautomatic,nonewpage]{imakeidx}


\makeindex[name=incipit,title=Index]
\indexsetup{level=\section,toclevel=section,noclearpage}

\usepackage[indentunit=8pt,rule=.5pt,columns=2]{idxlayout}


%%Table of Contents (rev. May 16, 2011)

%\usepackage{multicol}
%\usepackage{ifthen}
%\usepackage[toc]{multitoc}

%% General settings (rev. January 19, 2015)

\usepackage{ulem}

\usepackage[latin,german]{babel}
\usepackage{lettrine}

\usepackage{paracol}

\usepackage{fontspec}

\setmainfont[Ligatures=TeX,BoldFont=MinionPro-Bold,ItalicFont=MinionPro-It, BoldItalicFont=MinionPro-BoldIt]{MinionPro-Regular-Modified.otf}

%% Style for translation line
\grechangestyle{translation}{\fontsize{10}{10}\it\selectfont}
\grechangestyle{annotation}{\fontsize{10}{10}\selectfont}
\grechangestyle{commentary}{\textnormal\selectfont}
\gresetcustosalteration{invisible}

%\grechangedim{annotationseparation}{0.1cm}{scalable}

%\GreLoadSpaceConf{smith-four}

\frenchspacing

\usepackage{indentfirst} %%%indents first line after a section

\usepackage{graphicx}
%\usepackage{tocloft}

%%Hyperref (rev. August 20, 2011)
%\usepackage[colorlinks=false,hyperindex=true,bookmarks=true]{hyperref}
\usepackage{hyperref}
\hypersetup{pdftitle={Vesperale O.P. 2017}}
\hypersetup{pdfauthor={Order of Preachers}}
\hypersetup{pdfsubject={Liturgy}}
\hypersetup{pdfkeywords={Dominican, Liturgy, Order of Preachers, Dominican Rite, Liturgia Horarum, Divine Office}}

\newlength{\drop}



\begin{document}


\raggedbottom

\newcommand{\lectio}[3]{%
  \makebox[0pt][l]{#1}%
  \makebox[\textwidth][c]{#2}%
  \makebox[0pt][r]{\normalsize{\textnormal{#3}}}}


%%Combination

\chapter{Trauermette am Karsamstag}
\section{Officium Lectionis}
   \index[Varia]{Herr offne} \label{Herr offne (Varia)} \grecommentary[0pt]{} \gresetinitiallines{1} \grechangestyle{initial}{\fontsize{36}{36}\selectfont} \grechangedim{maxbaroffsettextleft@nobar}{12 cm}{scalable} \grechangedim{spaceabovelines}{0.5cm}{scalable} \gresetlyriccentering{syllable}  \grechangedim{maxbaroffsettextleft}{0 cm}{scalable} \gregorioscore{chants/herr_offne}
\subsubsection{Invitatorium}   \index[Invitatorium]{Invitatorium} \label{Invitatorium (Invitatorium)}         \vspace{5pt} \par \gresetlyriccentering{syllable}

\emph{Cantor:}~
\nopagebreak
\greannotation{I}
\gabcsnippet{(c4) Im(d) Kreuz(h_) Je(g)su(h) Chri(j_)sti(h_) fin(g)den(f) wir(e_) Heil.(d_) (::)} \vspace{5pt}

\emph{Alle:}~
\nopagebreak
\gabcsnippet{(c4) Im(d) Kreuz(h_) Je(g)su(h) Chri(j_)sti(h_) fin(g)den(f) wir(e_) Heil.(d_) (::)} \vspace{5pt}

\emph{Cantor:}~
\nopagebreak
\gabcsnippet{Kommt(dh_), lasst(h) uns(h) ju(i)beln(h) vor(h) dem(g) Herrn(h) (,) und(j) zu(h)jauch(h)zen(g) dem(f) Fels(g) uns(g)res(h) Hei(f)les! (f) (:) 
Lasst(f) uns(f) mit(g) Lob(ixi_) sei(h)-nem(h) An(h)ge(h)sicht(f) na(gh)hen, (g) (,) vor(e) ihm(e) jau(g)chzen(g) mit(h) Lie(fe)dern! (d) <sp>R/</sp> (::)} \vspace{5pt}


\emph{Cantor:}~
\nopagebreak
\gabcsnippet{(c4)Denn(d) der(d) Herr(h_) ist(h) ein(h) gro(i)ßer(g) Gott,(h) (,) ein(h) gro(h)ßer(j) Kö(hi)nig(h) üb(h)er(h) all(g)en(h) Gö(f)ttern.(f) (:)  In(d) sei(f)ner(f) Hand(g_) sind(g) die(g) Tie(h)fen(g) der(h) Er(f)de,(g) (,) sein(f) sind(f) die(f) Gi(g)pfel(g) der(f) Ber(gh)ge.(h) (:) Sein(h) ist(g) das(h) Meer,(ixi_) das(h) er(h) ge(f)macht(g) hat,(g) (,) das(g) tro(h)cke(g)ne(f) Land,(g_) das(g) sei(g)ne(e) Hän(g)de(g)  ge(h)bil(fe)det.(d) (::)} \vspace{5pt}

\emph{Alle:}~
\nopagebreak
\gabcsnippet{(c4) Im(d) Kreuz(h_) Je(g)su(h) Chri(j_)sti(h_) fin(g)den(f) wir(e_) Heil.(d_) (::)} \vspace{5pt}

\emph{Cantor:}~
\nopagebreak

\gabcsnippet{(c4)Kommt(d_), lasst(d) uns(d) nie(h)der(h)fal(h)len(h), uns(h) vor(h) ihm(i) ver(g)nei(hi)gen,(h) (,) lasst(h) uns(j) nie(h)der(h)knien(h) vor(g) dem(f) Herrn,(g) un(g)serm(h) Schö(f)pfer!(f) (:) Denn(f) er(g) ist(g) un(f)ser(g) Gott,(h) (,) wir(h) sind(h) das(h) Volk(ixi_) sei(h)ner(f) Wei(gh)de,(g) (,) die(g) Her(fg)de,(g) von(e) sei(g)ner(h) Hand(f) ge(e)führt.(d) 
(::)}

 \vspace{5pt}


\emph{Alle:}~
\nopagebreak
\gabcsnippet{(c4) Im(d) Kreuz(h_) Je(g)su(h) Chri(j_)sti(h_) fin(g)den(f) wir(e_) Heil.(d_) (::)}

 \vspace{5pt}

\emph{Cantor:}~
\nopagebreak
\gabcsnippet{(c4) Ach,(d_) wür(d)det(d) ihr(d) doch(d) heu(dh)te(h) auf(h) sei(h)ne(h) Sti(i)mme(g) hö(hi)ren!(h) (;) „Ver(h)här(h)tet(h) eu(h)er(h) Herz(g) nicht(h) wie(h) in(h) Me(f)rí(g)ba,(g) (,) wie(h) in(h) der(j) Wü(hi)ste(h) am(h) Tag(g) von(h) Ma(f)ssa!(f) (:) Dort(f) ha(f)ben(f) eu(f)re(f) Vä(g)ter(g) mich(f) ver(g)sucht,(h) (,) sie(h) ha(h)ben(h) mich(h) auf(h) die(h) Pro(ixi)be(h) ge(f)stellt(g) und(g) hat(g)ten(g) doch(g) mein(e) Tun(g) ge(h)se(fe)hen.(d) 
(::)}

 \vspace{5pt}


\emph{Alle:}~
\nopagebreak
\gabcsnippet{(c4) Im(d) Kreuz(h_) Je(g)su(h) Chri(j_)sti(h_) fin(g)den(f) wir(e_) Heil.(d_) (::)}

 \vspace{5pt}

\emph{Cantor:}~\nopagebreak

\gabcsnippet{(c4) Vier(d)zig(d) Jah(dh)re(h) war(h) mir(h) dies(h) Ge(h)schlecht(i) zu(g)wi(hi)der,(h) (;) und(h) ich(f) sa(gh)gte:(g) Sie(h) sind(h) ein(j) Volk(h), de(h)ssen(h) Herz(h) in(g) die(f) Ir(g)re(g) geht;(g) (,) denn(h) mei(h)ne(j) We(hi)ge(h) ken(h)nen(g) sie(h) nicht.(f) (:) Da(f)rum(f) ha(f)be(f) ich(f) in(f) mei(f)nem(f) Zorn(g) ge(f)schwo(gh)ren:(h) (,) Sie(h) sol(ixi)len(h) nicht(f) kom(gh)men(g) in(g) das(e) Land(g) mei(g)ner(h) Ru(fe)he.”(d) (::)}

 \vspace{5pt}


\emph{Alle:}~\nopagebreak

\gabcsnippet{(c4) Im(d) Kreuz(h_) Je(g)su(h) Chri(j_)sti(h_) fin(g)den(f) wir(e_) Heil.(d_) (::)}

 \vspace{5pt}

\emph{Cantor:}~
\nopagebreak

\gabcsnippet{(c4) Eh(d)re(d) dem(d) Va(dh)ter,(h) (,) Eh(i)re(h) dem(g) Sohn,(h) (,) Eh(h)re(h) dem(j) Hei(h)li(g)gen(h) Geist.(f) (:) Wie(f) im(g) An(ixi)fang,(h) (,) so(h) auch(h) jetzt(h) und(f) al(g)le(h) Zeit(g) und(g) in(e) E(g)wig(g)keit.(h) A(fvED)men.(d) (::)}

 \vspace{5pt}


\emph{Alle:}~
\nopagebreak
\gabcsnippet{(c4) Im(d) Kreuz(h_) Je(g)su(h) Chri(j_)sti(h_) fin(g)den(f) wir(e_) Heil.(d_) (::)}
\subsubsection{Hymnus}  \greannotation{I} \index[Hymnus]{Hymnus} \label{Hymnus (Hymnus)} \grecommentary[0pt]{} \gresetinitiallines{1} \grechangestyle{initial}{\fontsize{36}{36}\selectfont} \grechangedim{maxbaroffsettextleft@nobar}{12 cm}{scalable} \grechangedim{spaceabovelines}{0.5cm}{scalable} \gresetlyriccentering{syllable}   

\gabcsnippet{
(c4) Heil(d)ig(e) Kreuz,(gh) du(hg) Baum(h) der(j) Treu(ji)e,(g) (,)
ed(i)ler(j) Baum,(kl) dem(hg) kein(j)er(i) gleich,(h) (;)
kein(hj)er(i) so(g) an(e) Laub(f) und(dc) Blü(de)te,(e) (,)
kein(d)er(h) so(hg) an(ed) Früch(f)ten(e) reich:(d) (;)
Sü(hj)ßes(i) Holz,(g) o(e) sü(f)ße(dc) Nä(de)gel,(e) (,)
wel(d)che(h) sü(hg)ße(ed) Last(f) an(e) euch.(d) (::)}

\gresetinitiallines{0}

\vspace{7pt}

\gabcsnippet{2. Beu(d)ge,(e) ho(gh)her(hg) Baum,(h) die(j) Zwei(ji)ge,(g) (,)
wer(i)de(j) weich(kl) an(hg) Stamm(j) und(i) Ast,(h) (;)
denn(hj) dein(i) har(g)tes(e) Holz(f) muß(dc) tra(de)gen(e) (,)
ei(d)ne(h) kö(hg)nig(ed)li(f)che(e) Last,(d) (;)
gib(hj) den(i) Glie(g)dern(e) dei(f)nes(dc) Schöp(de)fers(e) (,)
an(d) dem(h) Stam(hg)me(ed) lin(f)de(e) Rast.(d) (::)}

\vspace{7pt}

\gabcsnippet{3. Du(d) al(e)lein(gh) warst(hg) wert,(h) zu(j) tra(ji)gen(g) (,)
al(i)ler(j) Sün(kl)den(hg) Lö(j)se(i)geld,(h) (;)
du,(hj) die(i) Plan(g)ke,(e) die(f) uns(dc) ret(de)tet(e) (,)
aus(d) dem(h) Schiff(hg)bruch(ed) die(f)ser(e) Welt.(d) (;)
Du,(hj) ge(i)salbt(g) vom(e) Blut(f) des(dc) Lam(de)mes,(e) (,)
Pfos(d)ten,(h) der(hg) den(ed) Tod(f) ab(e)hält.(d) (::)}

\vspace{7pt}

\gabcsnippet{4. Lob(d) und(e) Ruhm(gh) sei(hg) oh(h)ne(j) En(ji)de(g) (,)
Gott,(i) dem(j) höchs(kl)ten(hg) Herrn,(j) ge(i)weiht.(h) (;)
Preis(hj) dem(i) Va(g)ter(e) und(f) dem(dc) Soh(de)ne(e) (,)
und(d) dem(h) Geist(hg) der(ed) Hei(f)lig(e)keit.(d) (;)
Ei(hj)nen(i) Gott(g) in(e) drei(f) Per(dc)so(de)nen(e) (,)
lo(d)be(h) al(hg)le(ed) Welt(f) und(e) Zeit.(d) (::)
A(ded)men.(cd) (::)}                 \newpage
\subsection{Psalmodie} \noindent \textsc{1 Ant.} Ich lege mich nieder und ruhe in Frieden.

\subsubsection{Psalm 4, 2--9}

\noindent Wenn ich rufe, erhöre mich,~\GreStar{}~\nopagebreak

Gott, du mein Retter! 

\noindent Du hast mir Raum geschaffen, als mir angst war.~\GreStar{}~\nopagebreak

Sei mir gnädig und hör auf mein Flehen!

\noindent Ihr Mächtigen, wie lange noch schmäht ihr meine Ehre,~\GreStar{}~\nopagebreak

warum liebt ihr den Schein und sinnt auf Lügen?

\noindent Erkennt doch: Wunderbar handelt der Herr an den Frommen;~\GreStar{}~\nopagebreak

der Herr erhört mich, wenn ich zu ihm rufe.

\noindent Ereifert ihr euch, so sündigt nicht!~\GreStar{}~\nopagebreak

Bedenkt es auf eurem Lager und werdet stille!

\noindent Bringt rechte Opfer dar~\GreStar{}~\nopagebreak

und vertraut auf den Herrn!

\noindent Viele sagen: „Wer lässt uns Gutes erleben?“~\GreStar{}~\nopagebreak

Herr, lass dein Angesicht über uns leuchten!

\noindent Du legst mir größere Freude ins Herz,~\GreStar{}~\nopagebreak

als andere haben bei Korn und Wein in Fülle.

\noindent In Frieden leg’ ich mich nieder und schlafe ein;~\GreStar{}~\nopagebreak

denn du allein, Herr, lässt mich sorglos ruhen.

\noindent Ehre sei dem Vater und dem Sohn~\GreStar{}~\nopagebreak

und dem Heiligen Geist.

\noindent Wie im Anfang, so auch jetzt und alle Zeit~\GreStar{}~\nopagebreak

und in Ewigkeit. Amen.

\vspace{10pt}

\noindent \textsc{Ant.} Ich lege mich nieder und ruhe in Frieden. 

\vspace{10pt}

\noindent \textsc{2 Ant.} Mein Leib ruht in sicherer Hoffnung: Du gibst mich der Unterwelt nicht preis.

\subsubsection{Psalm 16, 1--11}

\noindent Behüte mich, Gott, denn ich vertraue dir.~†~\nopagebreak

Ich sage zum Herrn: „Du bist mein Herr;~\GreStar{}~\nopagebreak

mein ganzes Glück bist du allein.“

\noindent An den Heiligen im Lande, den Herrlichen,~\GreStar{}~\nopagebreak

an ihnen nur hab’ ich mein Gefallen.

\noindent Viele Schmerzen leidet, wer fremden Göttern folgt.~†~\nopagebreak

Ich will ihnen nicht opfern,~\GreStar{}~\nopagebreak

ich nehme ihre Namen nicht auf meine Lippen.

\noindent Du, Herr, gibst mir das Erbe und reichst mir den Becher;~\GreStar{}~\nopagebreak

du hältst mein Los in deinen Händen.

\noindent Auf schönem Land fiel mir mein Anteil zu.~\GreStar{}~\nopagebreak

Ja, mein Erbe gefällt mir gut.

\noindent Ich preise den Herrn, der mich beraten hat.~\GreStar{}~\nopagebreak

Auch mahnt mich mein Herz in der Nacht.

\noindent Ich habe den Herrn beständig vor Augen.~\GreStar{}~\nopagebreak

Er steht mir zur Rechten, ich wanke nicht.

\noindent Darum freut sich mein Herz und frohlockt meine Seele;~\GreStar{}~\nopagebreak

auch mein Leib wird wohnen in Sicherheit.

\noindent Denn du gibst mich nicht der Unterwelt preis;~\GreStar{}~\nopagebreak

du lässt deinen Frommen das Grab nicht schauen.

\noindent Du zeigst mir den Pfad zum Leben.~†~\nopagebreak

Vor deinem Angesicht herrscht Freude in Fülle,~\GreStar{}~\nopagebreak

zu deiner Rechten Wonne für alle Zeit.

\noindent Ehre sei dem Vater und dem Sohn~\GreStar{}~\nopagebreak

und dem Heiligen Geist.

\noindent Wie im Anfang, so auch jetzt und alle Zeit~\GreStar{}~\nopagebreak

und in Ewigkeit. Amen.

\vspace{10pt}

\noindent \textsc{Ant.} Mein Leib ruht in sicherer Hoffnung: Du gibst mich der Unterwelt nicht preis.

\vspace{10pt}

\noindent \textsc{3 Ant.} Hebt euch, ihr uralten Pforten! Es kommt der König der Herrlichkeit.

\subsubsection{Psalm 24, 1--10}

\noindent Dem Herrn gehört die Erde und was sie erfüllt,~\GreStar{}~\nopagebreak

der Erdkreis und seine Bewohner.

\noindent Denn er hat ihn auf Meere gegründet,~\GreStar{}~\nopagebreak

ihn über Strömen befestigt.

\noindent Wer darf hinaufziehn zum Berg des Herrn,~\GreStar{}~\nopagebreak

wer darf stehn an seiner heiligen Stätte?

\noindent Der reine Hände hat und ein lauteres Herz,~\GreStar{}~\nopagebreak

der nicht betrügt und keinen Meineid schwört.

\noindent Er wird Segen empfangen vom Herrn~\GreStar{}~\nopagebreak

und Heil von Gott, seinem Helfer.

\noindent Das sind die Menschen, die nach ihm fragen,~\GreStar{}~\nopagebreak

die dein Antlitz suchen, Gott Jakobs.

\noindent Ihr Tore, hebt euch nach oben,~†~\nopagebreak

hebt euch, ihr uralten Pforten;~\GreStar{}~\nopagebreak

denn es kommt der König der Herrlichkeit.

\noindent Wer ist der König der Herrlichkeit?~\GreStar{}~\nopagebreak

Der Herr, stark und gewaltig, der Herr, mächtig im Kampf.

\noindent Ihr Tore, hebt euch nach oben,~†~\nopagebreak

hebt euch, ihr uralten Pforten;~\GreStar{}~\nopagebreak

denn es kommt der König der Herrlichkeit.

\noindent Wer ist der König der Herrlichkeit?~\GreStar{}~\nopagebreak

Der Herr der Heerscharen, er ist der König der Herrlichkeit.

\noindent Ehre sei dem Vater und dem Sohn~\GreStar{}~\nopagebreak

und dem Heiligen Geist.

\noindent Wie im Anfang, so auch jetzt und alle Zeit~\GreStar{}~\nopagebreak

und in Ewigkeit. Amen.

\vspace{10pt}

\noindent \textsc{Ant.} Hebt euch, ihr uralten Pforten! Es kommt der König der Herrlichkeit.

\subsubsection{Versiculum}  \greannotation{} \index[Versiculum]{Versiculum} \label{Versiculum (Versiculum)} \grecommentary[0pt]{} \gresetinitiallines{0} \grechangestyle{initial}{\fontsize{36}{36}\selectfont} \grechangedim{maxbaroffsettextleft@nobar}{12 cm}{scalable} \grechangedim{spaceabovelines}{0.5cm}{scalable} \gresetlyriccentering{syllable}   \gregorioscore{chants/versiculum_thursday}    \newpage
\subsection{Lesungen}
\subsubsection{Erste Lesung} \vspace{5pt} \emph{Aus den Klageliedern des Propheten Jeremia. Aleph. Weh, wie glanzlos ist das Gold, gedunkelt das köstliche Feingold, hingeschüttet die heiligen Steine an den Ecken aller Straßen. Beth. Die edlen Kinder Zions, einst aufgewogen mit reinem Gold, weh, wie Krüge aus Ton sind sie geachtet, wie Werk von Töpferhand. Ghimel. Selbst Schakale reichen die Brust, säugen ihre Jungen. Die Töchter meines Volkes sind grausam wie Strauße in der Wüste. Daleth. Des Säuglings Zunge klebt an seinem Gaumen vor Durst. Kinder betteln um Brot; keiner bricht es ihnen. - Jerusalem, Jerusalem, bekehre dich zum Herrn, deinem Gott.} \vspace{5pt} \greannotation{} \index[Erste Lesung]{Erste Lesung} \label{Erste Lesung (Erste Lesung)} \grecommentary[3pt]{Klgl 4, 1--4} \gresetinitiallines{1} \grechangestyle{initial}{\fontsize{36}{36}\selectfont} \grechangedim{maxbaroffsettextleft@nobar}{12 cm}{scalable} \grechangedim{spaceabovelines}{0.5cm}{scalable} \gresetlyriccentering{vowel}   \gregorioscore{chants/va--lamentationes_71_de_lamentatione--dominican}
\subsubsection{II} \vspace{5pt} \emph{He. Die einst Leckerbissen schmausten, verschmachten auf den Straßen. Die einst auf Purpur lagen, wälzen sich jetzt im Unrat. Vau. Größer ist die Schuld der Tochter, meines Volkes, als die Sünde Sodoms, das plötzlich vernichtet wurde, ohne dass eine Hand sich rührte. Zain. Ihre jungen Männer waren reiner als Schnee, weißer als Milch, ihr Leib rosiger als Korallen, saphirblau ihre Adern. Heth. Schwärzer als Ruß sehen sie aus, man erkennt sie nicht auf den Straßen. Die Haut schrumpft ihnen am Leib, trocken wie Holz ist sie geworden. - Jerusalem, Jerusalem, bekehre dich zum Herrn, deinem Gott.} \vspace{5pt} \greannotation{} \index[II]{II} \label{II (II)} \grecommentary[3pt]{Klgl 4, 5--8} \gresetinitiallines{1} \grechangestyle{initial}{\fontsize{36}{36}\selectfont} \grechangedim{maxbaroffsettextleft@nobar}{12 cm}{scalable} \grechangedim{spaceabovelines}{0.5cm}{scalable} \gresetlyriccentering{vowel}   \gregorioscore{chants/va--lamentationes_72_he_qui_vescebantur--dominican}
\subsubsection{III} \vspace{5pt} \emph{Teth. Besser die vom Schwert Getöteten als die vom Hunger Getöteten; sie sind verschmachtet, vom Missertrag der Felder getroffen. Jod. Die Hände liebender Mütter kochten die eigenen Kinder. Sie dienten ihnen als Speise beim Zusammenbruch der Tochter, meines Volkes. Caph. Randvoll gemacht hat der Herr seinen Grimm, ausgegossen seinen glühenden Zorn. Er entfachte in Zion ein Feuer, das bis auf den Grund alles verzehrte. Lamed. Kein König eines Landes, kein Mensch auf der Erde hätte jemals geglaubt, dass ein Bedränger und Feind durchschritte die Tore Jerusalems. - Jerusalem, Jerusalem, bekehre dich zum Herrn, deinem Gott.}  \vspace{5pt} \greannotation{} \index[III]{III} \label{III (III)} \grecommentary[3pt]{Klgl 4, 9--12} \gresetinitiallines{1} \grechangestyle{initial}{\fontsize{36}{36}\selectfont} \grechangedim{maxbaroffsettextleft@nobar}{12 cm}{scalable} \grechangedim{spaceabovelines}{0.5cm}{scalable} \gresetlyriccentering{vowel}   \gregorioscore{chants/va--lamentationes_73_teth_melius_fuit--dominican}    \newpage
\subsubsection{Zweite Lesung}             \vspace{10pt}

Leo der Große († 461), Aus einer Predigt über die Passion des Herrn

\vspace{10pt}

%\par \hfill(Nn. 65071: SC 123, 95--101)

\lettrine[lines=3]{M}{}it allen Kräften unseres Geistes und unseres Leibes müssen wir darnach trachten, unzertrennlich mit dem Geheimnis des Leidens Christi verbunden zu bleiben; denn der Herr sagt: „Wer nicht sein Kreuz auf sich nimmt und mir nachfolgt, ist meiner nicht würdig.“ Und der Apostel spricht: „Wenn wir mit ihm leiden, werden wir auch mit ihm verherrlicht werden.“ Wer anders erweist also nach diesen Worten dem wahrhaft leidenden, sterbenden und auferstehenden Christus seine Verehrung, als wer mit ihm leidet, stirbt und aufersteht? Diese Teilnahme an dem Leiden des Herrn hat bei allen Kindern der Kirche schon mit ihrer wunderbaren Wiedergeburt begonnen: Durch die Tilgung der Sünde ersteht hier der Mensch zu neuem Leben, und durch das dreimalige Untertauchen wird der dreitägige Tod des Herrn versinnbildet. Bei der Taufe wird gleichsam die Erddecke von einem Grabe entfernt. Mit unserem alten Menschen steigen wir in den Taufquell hinab, und neugeboren kommen wir aus ihm hervor. Was aber durch dieses Sakrament mit uns begonnen wurde, das müssen wir durch Taten vollenden. Die ganze Lebenszeit, die den im Heiligen Geiste Wiedergeborenen noch übrigbleibt, muß ein beständiges Tragen des Kreuzes sein. Obgleich nämlich durch die Macht des Leidens Christi dem starken und grausamen Feinde unseres Geschlechtes die „Gefäße der alten Erbeutung“ entrissen wurden und „der Herrscher dieser Welt“ über die Herzen der Erlösten keine Gewalt mehr hat, verfolgt er doch die Menschen selbst nach ihrer Rechtfertigung immer noch mit seiner alten Bosheit. Auf mancherlei Art greift er die an, in denen er nicht mehr herrscht, um nachlässige und sorglose Seelen aufs neue mit noch grausameren Banden an sich zu ketten, um sie aus dem Paradies der Kirche zu vertreiben und sie zu Genossen seiner Verdammnis zu machen. Wenn darum jemand merkt, daß er die Grenzen der christlichen Gebote überschreitet und daß seine Begierden auf Dinge gerichtet sind, die ihn vom rechten Wege abbringen könnten, so nehme er seine Zuflucht zum Kreuze des Herrn und kreuzige sein sündhaftes Wollen und Wünschen auf dem Baume des Lebens.
    \newpage
\subsubsection{Oratio Ieremiae Prophetae} \vspace{5pt} \emph{Gebet des Propheten Jeremia. Herr, denk daran, was uns geschehen, blick her und sieh unsre Schmach! An Ausländer fiel unser Erbe, unsre Häuser kamen an Fremde. Wir wurden Waisen, Kinder ohne Vater, unsere Mütter wurden Witwen. Unser Wasser trinken wir für Geld, unser Holz müssen wir bezahlen. Wir werden getrieben, das Joch auf dem Nacken, wir sind müde, man versagt uns die Ruhe. Nach Ägypten streckten wir die Hand, nach Assur, um uns mit Brot zu sättigen. Unsere Väter haben gesündigt; sie sind nicht mehr. Wir müssen ihre Sünden tragen. Sklaven herrschen über uns, niemand entreißt uns ihren Händen. Unter Lebensgefahr holen wir unser Brot, bedroht vom Schwert der Wüste. Unsere Haut glüht wie ein Ofen von den Gluten des Hungers. Frauen hat man in Zion geschändet, Jungfrauen in den Städten von Juda. Fürsten wurden von Feindeshand gehängt, den ltesten nahm man die Ehre. Junge Männer mussten die Handmühlen schleppen, unter der Holzlast brachen Knaben zusammen. Die Alten blieben fern vom Tor, die Jungen vom Saitenspiel. Dahin ist unseres Herzens Freude, in Trauer gewandelt unser Reigen. Die Krone ist uns vom Haupt gefallen. Weh uns, wir haben gesündigt. Darum ist krank unser Herz, darum sind trüb unsere Augen über den Zionsberg, der verwüstet liegt; Füchse laufen dort umher. Du aber, Herr, bleibst ewig, dein Thron von Geschlecht zu Geschlecht. Warum willst du uns für immer vergessen, uns verlassen fürs ganze Leben? Kehre uns, Herr, dir zu, dann können wir uns zu dir bekehren. Erneuere unsere Tage, damit sie werden wie früher. Oder hast du uns denn ganz verworfen, zürnst du uns über alle Maßen? - Jerusalem, Jerusalem, bekehre dich zum Herrn, deinem Gott.}  \vspace{5pt} \greannotation{} \index[Oratio Ieremiae Prophetae]{Oratio Ieremiae Prophetae} \label{Oratio Ieremiae Prophetae (Oratio Ieremiae Prophetae)} \grecommentary[4pt]{Klgl 5, 1--22} \gresetinitiallines{1} \grechangestyle{initial}{\fontsize{36}{36}\selectfont} \grechangedim{maxbaroffsettextleft@nobar}{12 cm}{scalable} \grechangedim{spaceabovelines}{0.5cm}{scalable} \gresetlyriccentering{vowel}   \gregorioscore{chants/misc.oratio.ieremiae}    \newpage
\section{Laudes}
\subsubsection{Psalmodie} \noindent \textsc{1 Ant.} Sie klagen um ihn, wie man klagt um den einzigen Sohn; denn er wurde getötet - und war doch ohne Schuld.

\subsubsection{Psalm 64, 2-11}

\noindent Höre, o Gott, mein lautes Klagen,~\GreStar{}~\nopagebreak

schütze mein Leben vor dem Schrecken des Feindes!

\noindent Verbirg mich vor der Schar der Bösen,~\GreStar{}~\nopagebreak

vor dem Toben derer, die Unrecht tun.

\noindent Sie schärfen ihre Zunge wie ein Schwert,~\GreStar{}~\nopagebreak

schießen giftige Worte wie Pfeile,

\noindent um den Schuldlosen von ihrem Versteck aus zu treffen.~\GreStar{}~\nopagebreak

Sie schießen auf ihn, plötzlich und ohne Scheu.

\noindent Sie sind fest entschlossen zu bösem Tun.~\GreStar{}~\nopagebreak

Sie planen, Fallen zu stellen, und sagen: „Wer sieht uns schon?“

\noindent Sie haben Bosheit im Sinn,~\GreStar{}~\nopagebreak

doch halten sie ihre Pläne geheim. 

\noindent Ihr Inneres ist heillos verdorben,~\GreStar{}~\nopagebreak

ihr Herz ist ein Abgrund.

\noindent Da trifft sie Gott mit seinem Pfeil;~\GreStar{}~\nopagebreak

sie werden jählings verwundet.

\noindent Ihre eigene Zunge bringt sie zu Fall.~\GreStar{}~\nopagebreak

Alle, die es sehen, schütteln den Kopf.

\noindent Dann fürchten sich alle Menschen;~†~\nopagebreak

sie verkünden Gottes Taten~\GreStar{}~\nopagebreak

und bedenken sein Wirken.

\noindent Der Gerechte freut sich am Herrn und sucht bei ihm Zuflucht.~\GreStar{}~\nopagebreak

Und es rühmen sich alle Menschen mit redlichem Herzen.

\noindent Ehre sei dem Vater und dem Sohn~\GreStar{}~\nopagebreak

und dem Heiligen Geist.

\noindent Wie im Anfang, so auch jetzt und alle Zeit~\GreStar{}~\nopagebreak

und in Ewigkeit. Amen.

\vspace{10pt}

\noindent \textsc{Ant.} Sie klagen um ihn, wie man klagt um den einzigen Sohn; denn er wurde getötet - und war doch ohne Schuld.

%\vspace{10pt} 

\newpage

\noindent \textsc{2 Ant.} Vor den Pforten der Unterwelt rette, o Herr, mein Leben.

\subsubsection{Jes 38, 10-13a.14c-d.17-20}

\noindent Ich sagte: In der Mitte meiner Tage~†~\nopagebreak

muss ich hinab zu den Pforten der Unterwelt,~\GreStar{}~\nopagebreak

man raubt mir den Rest meiner Jahre.

\noindent Ich darf den Herrn nicht mehr schauen im Land der Lebenden,~\GreStar{}~\nopagebreak

keinen Menschen mehr sehen bei den Bewohnern der Erde.

\noindent Meine Hütte bricht man über mir ab,~\GreStar{}~\nopagebreak

man schafft sie weg wie das Zelt eines Hirten. 

\noindent Wie ein Weber hast du mein Leben zu Ende gewoben,~\GreStar{}~\nopagebreak

du schneidest mich ab wie ein fertig gewobenes Tuch. 

\noindent Vom Anbruch des Tages bis in die Nacht gibst du mich preis;~\GreStar{}~\nopagebreak

bis zum Morgen schreie ich um Hilfe.

\noindent Meine Augen blicken ermattet nach oben:~\GreStar{}~\nopagebreak

Ich bin in Not, Herr. Steh mir bei!

\noindent Du hast mich aus meiner bitteren Not gerettet,~†~\nopagebreak

du hast mich vor dem tödlichen Abgrund bewahrt;~\GreStar{}~\nopagebreak

denn all meine Sünden warfst du hinter deinen Rücken.

\noindent Ja, in der Unterwelt dankt man dir nicht,~†~\nopagebreak

die Toten loben dich nicht;~\GreStar{}~\nopagebreak

wer ins Grab gesunken ist, kann nichts mehr von deiner Güte erhoffen.

\noindent Nur die Lebenden danken dir, wie ich am heutigen Tag.~\GreStar{}~\nopagebreak

Von deiner Treue erzählt der Vater den Kindern.

\noindent Der Herr war bereit, mir zu helfen.~\GreStar{}~\nopagebreak

Wir wollen singen und spielen im Haus des Herrn, solange wir leben!

\noindent Ehre sei dem Vater und dem Sohn~\GreStar{}~\nopagebreak

und dem Heiligen Geist.

\noindent Wie im Anfang, so auch jetzt und alle Zeit~\GreStar{}~\nopagebreak

und in Ewigkeit. Amen.

\vspace{10pt}

\noindent \textsc{Ant.} Vor den Pforten der Unterwelt rette, o Herr, mein Leben.

%\vspace{10pt} 

\newpage

\noindent \textsc{3 Ant.} Ich war tot, doch ich lebe in Ewigkeit. Ich habe die Schlüssel des Todes und der Unterwelt.

\subsubsection{Psalm 150, 1-6}

\noindent Lobet Gott in seinem Heiligtum,~\GreStar{}~\nopagebreak

lobt ihn in seiner mächtigen Feste!

\noindent Lobt ihn für seine großen Taten,~\GreStar{}~\nopagebreak

lobt ihn in seiner gewaltigen Größe!

\noindent Lobt ihn mit dem Schall der Hörner,~\GreStar{}~\nopagebreak

lobt ihn mit Harfe und Zither!

\noindent Lobt ihn mit Pauken und Tanz,~\GreStar{}~\nopagebreak

lobt ihn mit Flöten und Saitenspiel!

\noindent Lobt ihn mit hellen Zimbeln,~\GreStar{}~\nopagebreak

lobt ihn mit klingenden Zimbeln!

\noindent Alles, was atmet,~\GreStar{}~\nopagebreak

lobe den Herrn!

\noindent Ehre sei dem Vater und dem Sohn~\GreStar{}~\nopagebreak

und dem Heiligen Geist.

\noindent Wie im Anfang, so auch jetzt und alle Zeit~\GreStar{}~\nopagebreak

und in Ewigkeit. Amen.

\vspace{10pt}

\noindent \textsc{Ant.} Ich war tot, doch ich lebe in Ewigkeit. Ich habe die Schlüssel des Todes und der Unterwelt.
                 \newpage
\subsubsection{Kurzlesung}     \hfill Hos 6, 1--2        \lettrine[lines=3]{K}{}ommt, wir kehren zum Herrn zurück! Denn er hat Wunden gerissen, er wird uns auch heilen; er hat verwundet, er wird auch verbinden. Nach zwei Tagen gibt er uns das Leben zurück, am dritten Tag richtet er uns wieder auf, und wir leben vor seinem Angesicht.
\subsubsection{Responsorium} \emph{Christus war für uns gehorsam bis zum Tod, bis zum Tod am Kreuze. \Vbar.~Darum auch hat Gott ihn erhöht und ihm den Namen gegeben, der über allen Namen steht.} \vspace{5pt}  \greannotation{V} \index[Responsorium]{Responsorium} \label{Responsorium (Responsorium)} \grecommentary[7pt]{Phil 2, 8; \Vbar.~9} \gresetinitiallines{1} \grechangestyle{initial}{\fontsize{36}{36}\selectfont} \grechangedim{maxbaroffsettextleft@nobar}{12 cm}{scalable} \grechangedim{spaceabovelines}{0.5cm}{scalable} \gresetlyriccentering{vowel}   \gregorioscore{chants/gr--christus_factus_est--dominican--id_6737}    \newpage
\subsubsection{Benedictus} \noindent \textsc{Ant.} Retter der Welt, errette uns! Du hast uns erlöst durch dein Kreuz und dein Blut. Hilf uns, Herr, unser Gott!

\paragraph{Lk 1, 68-79}

\gabcsnippet{(f3) (e f hr0 fR ,3) (hr0 iRr1 fR : hr0 e f gRr1 fR ::)}

\noindent Gepriesen sei der Herr, der Gott \uline{I}sraels!~\GreStar{}~\nopagebreak

Denn er hat sein Volk besucht und ihm Erlös\uline{u}ng geschaffen;

\noindent er hat uns einen starken Retter erw\uline{e}ckt~\GreStar{}~\nopagebreak

im Hause seines Kn\uline{e}chtes David.

\noindent So hat er verheißen von \uline{a}lters her~\GreStar{}~\nopagebreak

durch den Mund seiner heilig\uline{e}n Propheten.

\noindent Er hat uns errettet vor unseren F\uline{ei}nden~\GreStar{}~\nopagebreak

und aus der Hand aller, d\uline{ie} uns hassen;

\noindent er hat das Erbarmen mit den Vätern an uns vollend\uline{e}t~†~\nopagebreak

und an seinen heiligen B\uline{u}nd gedacht~\GreStar{}~\nopagebreak

an den Eid, den er unserm Vater Abrah\uline{a}m geschworen hat;

\noindent er hat uns geschenkt, dass wir, aus Feindeshand befr\uline{ei}t,~†~\nopagebreak

ihm furchtlos dienen in Heiligkeit und Ger\uline{e}chtigkeit~\GreStar{}~\nopagebreak

vor seinem Angesicht all \uline{u}nsre Tage.

\noindent Und du, Kind, wirst Prophet des Höchst\uline{en hei}ßen;~†~\nopagebreak

denn du wirst dem Herrn vor\uline{a}ngehn~\GreStar{}~\nopagebreak

und ihm den W\uline{e}g bereiten.

\noindent Du wirst sein Volk mit der Erfahrung des Heils besch\uline{e}nken~\GreStar{}~\nopagebreak

in der Vergeb\uline{u}ng der Sünden. 

\noindent Durch die barmherzige Liebe unseres G\uline{o}ttes~\GreStar{}~\nopagebreak

wird uns besuchen das aufstrahlende Licht \uline{au}s der Höhe,

\noindent um allen zu leuchten, die in Finsternis sitzen und im Schatten des T\uline{o}des,~\GreStar{}~\nopagebreak

und unsre Schritte zu lenken auf den W\uline{e}g des Friedens.

\noindent Ehre sei dem Vater und dem S\uline{o}hn~\GreStar{}~\nopagebreak

und dem H\uline{ei}ligen Geist.

\noindent Wie im Anfang, so auch jetzt und \uline{a}lle Zeit~\GreStar{}~\nopagebreak

und in Ew\uline{i}gkeit. Amen.

\vspace{5pt}

\noindent \textsc{Ant.} Retter der Welt, errette uns! Du hast uns erlöst durch dein Kreuz und dein Blut. Hilf uns, Herr, unser Gott!
                 \newpage
\subsubsection{Preces}  \greannotation{} \index[Preces]{Holy Thursday} \label{Holy Thursday (Preces)} \grecommentary[3pt]{} \gresetinitiallines{1} \grechangestyle{initial}{\fontsize{36}{36}\selectfont} \grechangedim{maxbaroffsettextleft@nobar}{12 cm}{scalable} \grechangedim{spaceabovelines}{0.7cm}{scalable} \gresetlyriccentering{vowel}   \gregorioscore{chants/misc.versus_litanici_in_cantu_sabbato_sancto-de-rubrics}
\subsubsection{Vater Unser}   \index[Vater Unser]{Vater Unser} \label{Vater Unser (Vater Unser)}             \vspace{10pt}
\subsubsection{Oration}   \index[Oration]{Oration} \label{Oration (Oration)}         \lettrine[lines=3]{A}{}llmächtiger, ewiger Gott, dein eingeborener Sohn ist in das Reich des Todes hinabgestiegen und von den Toten glorreich auferstanden. Gib, dass deine Gläubigen, die durch die Taufe mit ihm begraben wurden, durch seine Auferstehung zum ewigen Leben gelangen. Darum bitten wir durch ihn, Jesus Christus, deinen Sohn, unseren Herrn und Gott, der in der Einheit des Heiligen Geistes mit dir lebt und herrscht in alle Ewigkeit.
\par \Rbar.~Amen.

\subsubsection{Schlußsegen}   \index[Schlußsegen]{Schlußsegen} \label{Schlußsegen (Schlußsegen)}         Der Herr sei mit euch.

\Rbar. Und mit deinem Geiste.

Es segne euch der allmächtige Gott, + \par der Vater und der Sohn und der Heilige Geist.

\Rbar. Amen.

Gehet hin in Frieden.

\Rbar. Dank sei Gott, dem Herrn.

  \end{document}
