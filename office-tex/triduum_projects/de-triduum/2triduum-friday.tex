\documentclass[11pt,twoside]{book}

%%Page Size (rev. 08/19/2016)
%\usepackage[inner=0.5in, outer=0.5in, top=0.5in, bottom=0.5in, papersize={6in,9in}, head=12pt, headheight=30pt, headsep=5pt]{geometry}
\usepackage[inner=0.5in, outer=0.5in, top=0.5in, bottom=0.5in, papersize={5.5in,8.5in}, head=12pt, headheight=30pt, headsep=5pt]{geometry}
%% width of textblock = 324 pt / 4.5in
%% A5 = 5.8 x 8.3 inches -- if papersize is A5, then margins should be [inner=0.75in, outer=0.55in, top=0.4in, bottom=0.4in]


%%Header (rev. 4/11/2011)
\usepackage{fancyhdr}
 \pagestyle{fancy}
\renewcommand{\chaptermark}[1]{\markboth{#1}{}}
\renewcommand{\sectionmark}[1]{\markright{#1}}
 \fancyhf{}
\fancyhead[LE,RO]{\thepage}
\fancyhead[CE]{\leftmark}
\fancyhead[CO]{\rightmark}
 \fancypagestyle{plain}{ %
\fancyhf{} % remove everything
\renewcommand{\headrulewidth}{0pt} % remove lines as well
\renewcommand{\footrulewidth}{0pt}}



\usepackage[autocompile,allowdeprecated=false]{gregoriotex}
\usepackage{gregoriosyms}
\gresetgregoriofont[op]{greciliae}

%%Titles (rev. 9/4/2011) -- TOCLESS --- lets you have sections that don't appear in the table of contents

\setcounter{secnumdepth}{-1}

\usepackage[compact,nobottomtitles*]{titlesec}
\titlespacing*{\chapter}{0pt}{-30pt}{0pt}
\titlespacing*{\section}{0pt}{*0}{*1}
\titlespacing*{\subsubsection}{0pt}{*0}{*0}
\titlespacing*{\subsubsubsection}{0pt}{10pt}{*0}
\titleformat{\part} {\normalfont\Huge\sc\center}{\thechapter}{1em}{}
\titleformat{\chapter} {\normalfont\LARGE\sc\center}{\thechapter}{1em}{}
\titleformat{\section} {\normalfont\Large\sc\center}{\thesection}{1em}{}
\titleformat{\subsection} {\normalfont\Large\sc\center}{\thesubsubsection}{1em}{}
\titleformat{\subsubsection}{\normalfont\large\sc\center}{\thesubsubsubsection}{1em}{}
\titleformat{\paragraph}{\normalfont\normalsize\sc\center}{\thesubsubsubsection}{1em}{}

\newcommand{\nocontentsline}[3]{}
\newcommand{\tocless}[2]{\bgroup\let\addcontentsline=\nocontentsline#1{#2}\egroup} %% lets you have sections that don't appear in the table of contents


%%%

%%Index (rev. December 11, 2013)
\usepackage[noautomatic,nonewpage]{imakeidx}


\makeindex[name=incipit,title=Index]
\indexsetup{level=\section,toclevel=section,noclearpage}

\usepackage[indentunit=8pt,rule=.5pt,columns=2]{idxlayout}


%%Table of Contents (rev. May 16, 2011)

%\usepackage{multicol}
%\usepackage{ifthen}
%\usepackage[toc]{multitoc}

%% General settings (rev. January 19, 2015)

\usepackage{ulem}

\usepackage[latin,german]{babel}
\usepackage{lettrine}

\usepackage{paracol}

\usepackage{fontspec}

\setmainfont[Ligatures=TeX,BoldFont=MinionPro-Bold,ItalicFont=MinionPro-It, BoldItalicFont=MinionPro-BoldIt]{MinionPro-Regular-Modified.otf}

\usepackage{unicode-math}
\setmathfont{LatinModernMath-Regular}
\AtBeginDocument{\grelatexsimpledefbarredsymbol{V}{0.1em}{0.12em}{0.14em}{0.18em}}

%% Style for translation line
\grechangestyle{translation}{\fontsize{10}{10}\it\selectfont}
\grechangestyle{annotation}{\fontsize{10}{10}\selectfont}
\grechangestyle{commentary}{\textnormal\selectfont}
\gresetcustosalteration{invisible}

%\grechangedim{annotationseparation}{0.1cm}{scalable}

%\GreLoadSpaceConf{smith-four}

\frenchspacing

\usepackage{indentfirst} %%%indents first line after a section

\usepackage{graphicx}
%\usepackage{tocloft}

%%Hyperref (rev. August 20, 2011)
%\usepackage[colorlinks=false,hyperindex=true,bookmarks=true]{hyperref}
\usepackage{hyperref}
\hypersetup{pdftitle={Vesperale O.P. 2017}}
\hypersetup{pdfauthor={Order of Preachers}}
\hypersetup{pdfsubject={Liturgy}}
\hypersetup{pdfkeywords={Dominican, Liturgy, Order of Preachers, Dominican Rite, Liturgia Horarum, Divine Office}}

\newlength{\drop}



\begin{document}


\raggedbottom

\newcommand{\lectio}[3]{%
  \makebox[0pt][l]{#1}%
  \makebox[\textwidth][c]{#2}%
  \makebox[0pt][r]{\normalsize{\textnormal{#3}}}}


%%Combination
\chapter{Trauermette am Karfreitag}
\section{Officium Lectionis}
    \index[Varia]{Herr offne} \label{Herr offne (Varia)} \grecommentary[0pt]{} \gresetinitiallines{1} \grechangestyle{initial}{\fontsize{36}{36}\selectfont} \grechangedim{maxbaroffsettextleft@nobar}{12 cm}{scalable} \grechangedim{spaceabovelines}{0.5cm}{scalable} \gresetlyriccentering{syllable}  \grechangedim{maxbaroffsettextleft}{0 cm}{scalable} \gregorioscore{chants/herr_offne}
 \subsubsection{Invitatorium}   \index[Invitatorium]{Invitatorium} \label{Invitatorium (Invitatorium)}         \vspace{5pt} \par \gresetlyriccentering{syllable}

\emph{Cantor:}~
\nopagebreak
\greannotation{I}
\gabcsnippet{(c4) Im(d) Kreuz(h_) Je(g)su(h) Chri(j_)sti(h_) fin(g)den(f) wir(e_) Heil.(d_) (::)} \vspace{5pt}

\emph{Alle:}~
\nopagebreak
\gabcsnippet{(c4) Im(d) Kreuz(h_) Je(g)su(h) Chri(j_)sti(h_) fin(g)den(f) wir(e_) Heil.(d_) (::)} \vspace{5pt}

\emph{Cantor:}~
\nopagebreak
\gabcsnippet{Kommt(dh_), lasst(h) uns(h) ju(i)beln(h) vor(h) dem(g) Herrn(h) (,) und(j) zu(h)jauch(h)zen(g) dem(f) Fels(g) uns(g)res(h) Hei(f)les! (f) (:) 
Lasst(f) uns(f) mit(g) Lob(ixi_) sei(h)-nem(h) An(h)ge(h)sicht(f) na(gh)hen, (g) (,) vor(e) ihm(e) jau(g)chzen(g) mit(h) Lie(fe)dern! (d) <sp>R/</sp> (::)} \vspace{5pt}


\emph{Cantor:}~
\nopagebreak
\gabcsnippet{(c4)Denn(d) der(d) Herr(h_) ist(h) ein(h) gro(i)ßer(g) Gott,(h) (,) ein(h) gro(h)ßer(j) Kö(hi)nig(h) üb(h)er(h) all(g)en(h) Gö(f)ttern.(f) (:)  In(d) sei(f)ner(f) Hand(g_) sind(g) die(g) Tie(h)fen(g) der(h) Er(f)de,(g) (,) sein(f) sind(f) die(f) Gi(g)pfel(g) der(f) Ber(gh)ge.(h) (:) Sein(h) ist(g) das(h) Meer,(ixi_) das(h) er(h) ge(f)macht(g) hat,(g) (,) das(g) tro(h)cke(g)ne(f) Land,(g_) das(g) sei(g)ne(e) Hän(g)de(g)  ge(h)bil(fe)det.(d) (::)} \vspace{5pt}

\emph{Alle:}~
\nopagebreak
\gabcsnippet{(c4) Im(d) Kreuz(h_) Je(g)su(h) Chri(j_)sti(h_) fin(g)den(f) wir(e_) Heil.(d_) (::)} \vspace{5pt}

\emph{Cantor:}~
\nopagebreak

\gabcsnippet{(c4)Kommt(d_), lasst(d) uns(d) nie(h)der(h)fal(h)len(h), uns(h) vor(h) ihm(i) ver(g)nei(hi)gen,(h) (,) lasst(h) uns(j) nie(h)der(h)knien(h) vor(g) dem(f) Herrn,(g) un(g)serm(h) Schö(f)pfer!(f) (:) Denn(f) er(g) ist(g) un(f)ser(g) Gott,(h) (,) wir(h) sind(h) das(h) Volk(ixi_) sei(h)ner(f) Wei(gh)de,(g) (,) die(g) Her(fg)de,(g) von(e) sei(g)ner(h) Hand(f) ge(e)führt.(d) 
(::)}

 \vspace{5pt}


\emph{Alle:}~
\nopagebreak
\gabcsnippet{(c4) Im(d) Kreuz(h_) Je(g)su(h) Chri(j_)sti(h_) fin(g)den(f) wir(e_) Heil.(d_) (::)}

 \vspace{5pt}

\emph{Cantor:}~
\nopagebreak
\gabcsnippet{(c4) Ach,(d_) wür(d)det(d) ihr(d) doch(d) heu(dh)te(h) auf(h) sei(h)ne(h) Sti(i)mme(g) hö(hi)ren!(h) (;) „Ver(h)här(h)tet(h) eu(h)er(h) Herz(g) nicht(h) wie(h) in(h) Me(f)rí(g)ba,(g) (,) wie(h) in(h) der(j) Wü(hi)ste(h) am(h) Tag(g) von(h) Ma(f)ssa!(f) (:) Dort(f) ha(f)ben(f) eu(f)re(f) Vä(g)ter(g) mich(f) ver(g)sucht,(h) (,) sie(h) ha(h)ben(h) mich(h) auf(h) die(h) Pro(ixi)be(h) ge(f)stellt(g) und(g) hat(g)ten(g) doch(g) mein(e) Tun(g) ge(h)se(fe)hen.(d) 
(::)}

 \vspace{5pt}


\emph{Alle:}~
\nopagebreak
\gabcsnippet{(c4) Im(d) Kreuz(h_) Je(g)su(h) Chri(j_)sti(h_) fin(g)den(f) wir(e_) Heil.(d_) (::)}

 \vspace{5pt}

\emph{Cantor:}~\nopagebreak

\gabcsnippet{(c4) Vier(d)zig(d) Jah(dh)re(h) war(h) mir(h) dies(h) Ge(h)schlecht(i) zu(g)wi(hi)der,(h) (;) und(h) ich(f) sa(gh)gte:(g) Sie(h) sind(h) ein(j) Volk(h), de(h)ssen(h) Herz(h) in(g) die(f) Ir(g)re(g) geht;(g) (,) denn(h) mei(h)ne(j) We(hi)ge(h) ken(h)nen(g) sie(h) nicht.(f) (:) Da(f)rum(f) ha(f)be(f) ich(f) in(f) mei(f)nem(f) Zorn(g) ge(f)schwo(gh)ren:(h) (,) Sie(h) sol(ixi)len(h) nicht(f) kom(gh)men(g) in(g) das(e) Land(g) mei(g)ner(h) Ru(fe)he.”(d) (::)}

 \vspace{5pt}


\emph{Alle:}~\nopagebreak

\gabcsnippet{(c4) Im(d) Kreuz(h_) Je(g)su(h) Chri(j_)sti(h_) fin(g)den(f) wir(e_) Heil.(d_) (::)}

 \vspace{5pt}

\emph{Cantor:}~
\nopagebreak

\gabcsnippet{(c4) Eh(d)re(d) dem(d) Va(dh)ter,(h) (,) Eh(i)re(h) dem(g) Sohn,(h) (,) Eh(h)re(h) dem(j) Hei(h)li(g)gen(h) Geist.(f) (:) Wie(f) im(g) An(ixi)fang,(h) (,) so(h) auch(h) jetzt(h) und(f) al(g)le(h) Zeit(g) und(g) in(e) E(g)wig(g)keit.(h) A(fvED)men.(d) (::)}

 \vspace{5pt}


\emph{Alle:}~
\nopagebreak
\gabcsnippet{(c4) Im(d) Kreuz(h_) Je(g)su(h) Chri(j_)sti(h_) fin(g)den(f) wir(e_) Heil.(d_) (::)}
\subsubsection{Hymnus}  \greannotation{I} \index[Hymnus]{Hymnus} \label{Hymnus (Hymnus)} \grecommentary[0pt]{} \gresetinitiallines{1} \grechangestyle{initial}{\fontsize{36}{36}\selectfont} \grechangedim{maxbaroffsettextleft@nobar}{12 cm}{scalable} \grechangedim{spaceabovelines}{0.5cm}{scalable} \gresetlyriccentering{syllable}   

\gabcsnippet{
(c4) Heil(d)ig(e) Kreuz,(gh) du(hg) Baum(h) der(j) Treu(ji)e,(g) (,)
ed(i)ler(j) Baum,(kl) dem(hg) kein(j)er(i) gleich,(h) (;)
kein(hj)er(i) so(g) an(e) Laub(f) und(dc) Blü(de)te,(e) (,)
kein(d)er(h) so(hg) an(ed) Früch(f)ten(e) reich:(d) (;)
Sü(hj)ßes(i) Holz,(g) o(e) sü(f)ße(dc) Nä(de)gel,(e) (,)
wel(d)che(h) sü(hg)ße(ed) Last(f) an(e) euch.(d) (::)}

\gresetinitiallines{0}

\vspace{7pt}

\gabcsnippet{2. Beu(d)ge,(e) ho(gh)her(hg) Baum,(h) die(j) Zwei(ji)ge,(g) (,)
wer(i)de(j) weich(kl) an(hg) Stamm(j) und(i) Ast,(h) (;)
denn(hj) dein(i) har(g)tes(e) Holz(f) muß(dc) tra(de)gen(e) (,)
ei(d)ne(h) kö(hg)nig(ed)li(f)che(e) Last,(d) (;)
gib(hj) den(i) Glie(g)dern(e) dei(f)nes(dc) Schöp(de)fers(e) (,)
an(d) dem(h) Stam(hg)me(ed) lin(f)de(e) Rast.(d) (::)}

\vspace{7pt}

\gabcsnippet{3. Du(d) al(e)lein(gh) warst(hg) wert,(h) zu(j) tra(ji)gen(g) (,)
al(i)ler(j) Sün(kl)den(hg) Lö(j)se(i)geld,(h) (;)
du,(hj) die(i) Plan(g)ke,(e) die(f) uns(dc) ret(de)tet(e) (,)
aus(d) dem(h) Schiff(hg)bruch(ed) die(f)ser(e) Welt.(d) (;)
Du,(hj) ge(i)salbt(g) vom(e) Blut(f) des(dc) Lam(de)mes,(e) (,)
Pfos(d)ten,(h) der(hg) den(ed) Tod(f) ab(e)hält.(d) (::)}

\vspace{7pt}

\gabcsnippet{4. Lob(d) und(e) Ruhm(gh) sei(hg) oh(h)ne(j) En(ji)de(g) (,)
Gott,(i) dem(j) höchs(kl)ten(hg) Herrn,(j) ge(i)weiht.(h) (;)
Preis(hj) dem(i) Va(g)ter(e) und(f) dem(dc) Soh(de)ne(e) (,)
und(d) dem(h) Geist(hg) der(ed) Hei(f)lig(e)keit.(d) (;)
Ei(hj)nen(i) Gott(g) in(e) drei(f) Per(dc)so(de)nen(e) (,)
lo(d)be(h) al(hg)le(ed) Welt(f) und(e) Zeit.(d) (::)
A(ded)men.(cd) (::)}                 \newpage
\subsection{Psalmodie} \noindent \textsc{1 Ant.} Die Könige der Erde stehen auf, die Großen haben sich verbündet gegen den Herrn und seinen Gesalbten.

\subsubsection{Ps 2,1-12}

Warum toben die Völker,~$\star$~\nopagebreak

warum machen die Nationen vergebliche Pläne?
 
\noindent Die Könige der Erde stehen auf,~$\star$~\nopagebreak

die Großen haben sich verbündet gegen den Herrn und seinen Gesalbten.
 
\noindent «Lasst uns ihre Fesseln zerreißen~$\star$~\nopagebreak

und von uns werfen ihre Stricke!»
 
\noindent Doch er, der im Himmel thront, lacht,~$\star$~\nopagebreak

der Herr verspottet sie.
 
\noindent Dann aber spricht er zu ihnen im Zorn,~$\star$~\nopagebreak

in seinem Grimm wird er sie erschrecken:
 
\noindent «Ich selber habe meinen König eingesetzt~$\star$~\nopagebreak

auf Zion, meinem heiligen Berg.»
 
\noindent Den Beschluss des Herrn will ich kundtun.~†~\nopagebreak

Er sprach zu mir: «Mein Sohn bist du.~$\star$~\nopagebreak

Heute habe ich dich gezeugt.
 
\noindent Fordre von mir und ich gebe dir die Völker zum Erbe,~$\star$~\nopagebreak

die Enden der Erde zum Eigentum.
 
\noindent Du wirst sie zerschlagen mit eiserner Keule,~$\star$~\nopagebreak

wie Krüge aus Ton wirst du sie zertrümmern.»
 
\noindent Nun denn, ihr Könige, kommt zur Einsicht,~$\star$~\nopagebreak

lasst euch warnen, ihr Gebieter der Erde!
 
\noindent Dient dem Herrn in Furcht~$\star$~\nopagebreak

und küsst ihm mit Beben die Füße,
 
\noindent damit er nicht zürnt~$\star$~\nopagebreak

und euer Weg nicht in den Abgrund führt. 
 
\noindent Denn wenig nur und sein Zorn ist entbrannt.~$\star$~\nopagebreak

Wohl allen, die ihm vertrauen!

\noindent Ehre sei dem Vater und dem Sohn~$\star$~\nopagebreak

und dem Heiligen Geist.

\noindent Wie im Anfang, so auch jetzt und alle Zeit~$\star$~\nopagebreak

und in Ewigkeit. Amen.

\vspace{10pt}

\noindent \textsc{Ant.} Die Könige der Erde stehen auf, die Großen haben sich verbündet gegen den Herrn und seinen Gesalbten.

\vspace{10pt}


\noindent \textsc{2 Ant.} Sie verteilen unter sich meine Kleider und werfen das Los um mein Gewand.

\subsubsection{Ps 22 (21), 2-23}

Mein Gott, mein Gott, warum hast du mich verlassen,~$\star$~\nopagebreak

bist fern meinem Schreien, den Worten meiner Klage?
 
\noindent Mein Gott, ich rufe bei Tag, doch du gibst keine Antwort;~$\star$~\nopagebreak

ich rufe bei Nacht und finde doch keine Ruhe.
 
\noindent Aber du bist heilig,~$\star$~\nopagebreak

du thronst über dem Lobpreis Israels.
 
\noindent Dir haben unsre Väter vertraut,~$\star$~\nopagebreak

sie haben vertraut und du hast sie gerettet.
 
\noindent Zu dir riefen sie und wurden befreit,~$\star$~\nopagebreak

dir vertrauten sie und wurden nicht zuschanden.
 
\noindent Ich aber bin ein Wurm und kein Mensch,~$\star$~\nopagebreak

der Leute Spott, vom Volk verachtet.
 
\noindent Alle, die mich sehen, verlachen mich,~$\star$~\nopagebreak

verziehen die Lippen, schütteln den Kopf:
 
\noindent «Er wälze die Last auf den Herrn,~$\star$~\nopagebreak

der soll ihn befreien! 
 
\noindent Der reiße ihn heraus,~$\star$~\nopagebreak

wenn er an ihm Gefallen hat.»
 
\noindent Du bist es, der mich aus dem Schoß meiner Mutter zog,~$\star$~\nopagebreak

mich barg an der Brust der Mutter.
 
\noindent Von Geburt an bin ich geworfen auf dich,~$\star$~\nopagebreak

vom Mutterleib an bist du mein Gott.
 
\noindent Sei mir nicht fern, denn die Not ist nahe~$\star$~\nopagebreak

und niemand ist da, der hilft.
 
\noindent Viele Stiere umgeben mich,~$\star$~\nopagebreak

Büffel von Baschan umringen mich.
 
\noindent Sie sperren gegen mich ihren Rachen auf,~$\star$~\nopagebreak

reißende, brüllende Löwen.
 
\noindent Ich bin hingeschüttet wie Wasser,~†~\nopagebreak

gelöst haben sich all meine Glieder.~$\star$~\nopagebreak

Mein Herz ist in meinem Leib wie Wachs zerflossen.
 
\noindent Meine Kehle ist trocken wie eine Scherbe,~†~\nopagebreak

die Zunge klebt mir am Gaumen,~$\star$~\nopagebreak

du legst mich in den Staub des Todes.
 
\noindent Viele Hunde umlagern mich,~†~\nopagebreak

eine Rotte von Bösen umkreist mich.~$\star$~\nopagebreak

Sie durchbohren mir Hände und Füße.
 
\noindent Man kann all meine Knochen zählen;~$\star$~\nopagebreak

sie gaffen und weiden sich an mir.
 
\noindent Sie verteilen unter sich meine Kleider~$\star$~\nopagebreak

und werfen das Los um mein Gewand.
 
\noindent Du aber, Herr, halte dich nicht fern!~$\star$~\nopagebreak

Du, meine Stärke, eil mir zu Hilfe!
 
\noindent Entreiße mein Leben dem Schwert,~$\star$~\nopagebreak

mein einziges Gut aus der Gewalt der Hunde!
 
\noindent Rette mich vor dem Rachen des Löwen,~$\star$~\nopagebreak

vor den Hörnern der Büffel rette mich Armen!
 
\noindent Ich will deinen Namen meinen Brüdern verkünden,~$\star$~\nopagebreak

inmitten der Gemeinde dich preisen.

\noindent Ehre sei dem Vater und dem Sohn~$\star$~\nopagebreak

und dem Heiligen Geist.

\noindent Wie im Anfang, so auch jetzt und alle Zeit~$\star$~\nopagebreak

und in Ewigkeit. Amen.

\vspace{10pt}

\noindent \textsc{Ant.} Sie verteilen unter sich meine Kleider und werfen das Los um mein Gewand.

\vspace{10pt}

\noindent \textsc{3 Ant.} Die mir nach dem Leben trachten, legen mir Schlingen; die mein Unheil suchen, planen Verderben.

\subsubsection{Ps 38 (37)}

Herr, strafe mich nicht in deinem Zorn~$\star$~\nopagebreak

und züchtige mich nicht in deinem Grimm!
 
\noindent Denn deine Pfeile haben mich getroffen,~$\star$~\nopagebreak

deine Hand lastet schwer auf mir.
 
\noindent Nichts blieb gesund an meinem Leib, weil du mir grollst;~$\star$~\nopagebreak

weil ich gesündigt, blieb an meinen Gliedern nichts heil.
 
\noindent Denn meine Sünden schlagen mir über dem Kopf zusammen,~$\star$~\nopagebreak

sie erdrücken mich wie eine schwere Last.
 
\noindent Mir schwären, mir eitern die Wunden~$\star$~\nopagebreak

wegen meiner Torheit.
 
\noindent Ich bin gekrümmt und tief gebeugt,~$\star$~\nopagebreak

den ganzen Tag geh ich traurig einher.
 
\noindent Denn meine Lenden sind voller Brand,~$\star$~\nopagebreak

nichts blieb gesund an meinem Leib.
 
\noindent Kraftlos bin ich und ganz zerschlagen,~$\star$~\nopagebreak

ich schreie in der Qual meines Herzens.
 
\noindent All mein Sehnen, Herr, liegt offen vor dir,~$\star$~\nopagebreak

mein Seufzen ist dir nicht verborgen.
 
\noindent Mein Herz pocht heftig, mich hat die Kraft verlassen,~$\star$~\nopagebreak

geschwunden ist mir das Licht der Augen.
 
\noindent Freunde und Gefährten bleiben mir fern in meinem Unglück~$\star$~\nopagebreak

und meine Nächsten meiden mich.
 
\noindent Die mir nach dem Leben trachten, legen mir Schlingen;~†~\nopagebreak

die mein Unheil suchen, planen Verderben,~$\star$~\nopagebreak

den ganzen Tag haben sie Arglist im Sinn.
 
\noindent Ich bin wie ein Tauber, der nicht hört,~$\star$~\nopagebreak

wie ein Stummer, der den Mund nicht auftut.
 
\noindent Ich bin wie einer, der nicht mehr hören kann,~$\star$~\nopagebreak

aus dessen Mund keine Entgegnung kommt.
 
\noindent Doch auf dich, Herr, harre ich;~$\star$~\nopagebreak

du wirst mich erhören, Herr, mein Gott.
 
\noindent Denn ich sage: Über mich sollen die sich nicht freuen,~$\star$~\nopagebreak

die gegen mich prahlen, wenn meine Füße straucheln.
 
\noindent Ich bin dem Fallen nahe,~$\star$~\nopagebreak

mein Leid steht mir immer vor Augen.
 
\noindent Ja, ich bekenne meine Schuld,~$\star$~\nopagebreak

ich bin wegen meiner Sünde in Angst.
 
\noindent Die mich ohne Grund befehden, sind stark;~$\star$~\nopagebreak

viele hassen mich wegen nichts.
 
\noindent Sie vergelten mir Gutes mit Bösem,~$\star$~\nopagebreak

sie sind mir Feind; denn ich trachte nach dem Guten.
 
\noindent Herr, verlass mich nicht, bleib mir nicht fern, mein Gott!~$\star$~\nopagebreak

Eile mir zu Hilfe, Herr, du mein Heil!

\noindent Ehre sei dem Vater und dem Sohn~$\star$~\nopagebreak

und dem Heiligen Geist.

\noindent Wie im Anfang, so auch jetzt und alle Zeit~$\star$~\nopagebreak

und in Ewigkeit. Amen.

\vspace{10pt}

\noindent \textsc{Ant.} Die mir nach dem Leben trachten, legen mir Schlingen; die mein Unheil suchen, planen Verderben.


 \subsubsection{Versiculum}  \greannotation{} \index[Versiculum]{Versiculum} \label{Versiculum (Versiculum)} \grecommentary[0pt]{} \gresetinitiallines{0} \grechangestyle{initial}{\fontsize{36}{36}\selectfont} \grechangedim{maxbaroffsettextleft@nobar}{12 cm}{scalable} \grechangedim{spaceabovelines}{0.5cm}{scalable} \gresetlyriccentering{syllable}   \gregorioscore{chants/versiculum_thursday}
\subsection{Lesungen}
 \subsubsection{Erste Lesung} \vspace{5pt} \emph{Aus den Klageliedern des Propheten Jeremia. Aleph. Weh, mit seinem Zorn umwölkt der Herr die Tochter Zion. Er schleudert vom Himmel zur Erde die Pracht Israels. Nicht dachte er an den Schemel seiner Füße am Tag seines Zornes. Beth. Schonungslos hat der Herr vernichtet alle Fluren Jakobs, niedergerissen in seinem Grimm die Bollwerke der Tochter Juda, zu Boden gestreckt, entweiht das Königtum und seine Fürsten. Ghimel. Abgehauen hat er in Zornesglut jedes Horn in Israel. Er zog seine Rechte zurück angesichts des Feindes und brannte in Jakob wie flammendes Feuer, ringsum alles verzehrend. - Jerusalem, Jerusalem, bekehre dich zum Herrn, deinem Gott.} \vspace{5pt} \greannotation{} \index[Erste Lesung]{Erste Lesung} \label{Erste Lesung (Erste Lesung)} \grecommentary[0pt]{Klgl 2, 1-3} \gresetinitiallines{1} \grechangestyle{initial}{\fontsize{36}{36}\selectfont} \grechangedim{maxbaroffsettextleft@nobar}{12 cm}{scalable} \grechangedim{spaceabovelines}{0.5cm}{scalable} \gresetlyriccentering{syllable}   \gregorioscore{chants/va--lamentationes_61_de_lamentatione_ieremiae_prophetae--dominican}
 \subsubsection{II} \vspace{5pt} \emph{Daleth. Er spannte den Bogen wie ein Feind, stand da, erhoben die Rechte. Wie ein Gegner erschlug er alles, was das Auge erfreut. Im Zelt der Tochter Zion goss er seinen Zorn aus wie Feuer. He. Wie ein Feind ist geworden der Herr, Israel hat er vernichtet. Vernichtet hat er alle Paläste, zerstört seine Burgen. Auf die Tochter Juda hat er gehäuft Jammer über Jammer. Vau. Er zertrat wie einen Garten seine Wohnstatt, zerstörte seinen Festort. Vergessen ließ der Herr auf Zion Festtag und Sabbat. In glühendem Zorn verwarf er König und Priester. - Jerusalem, Jerusalem, bekehre dich zum Herrn, deinem Gott.} \vspace{5pt} \greannotation{} \index[II]{II} \label{II (II)} \grecommentary[0pt]{Klgl 2, 4-6} \gresetinitiallines{1} \grechangestyle{initial}{\fontsize{36}{36}\selectfont} \grechangedim{maxbaroffsettextleft@nobar}{12 cm}{scalable} \grechangedim{spaceabovelines}{0.5cm}{scalable} \gresetlyriccentering{syllable}   \gregorioscore{chants/va--lamentationes_62_daleth_tetendit--dominican}
 \subsubsection{III} \vspace{5pt} \emph{Zain. Seinen Altar hat der Herr verschmäht, entweiht sein Heiligtum, überliefert in die Hand des Feindes die Mauern von Zions Palästen. Man lärmte im Haus des Herrn wie an einem Festtag Zu schleifen plante der Herr die Mauer der Tochter Zion. Heth. Er spannte die Messschnur und zog nicht zurück die Hand vom Vertilgen. Trauern ließ er Wall und Mauer; miteinander sanken sie nieder. Teth. In den Boden sanken ihre Tore, ihre Riegel hat er zerstört und zerbrochen. Ihr König und ihre Fürsten sind unter den Völkern, keine Weisung ist da, auch keine Offenbarung schenkt der Herr ihren Propheten. - Jerusalem, Jerusalem, bekehre dich zum Herrn, deinem Gott.}  \vspace{5pt} \greannotation{} \index[III]{III} \label{III (III)} \grecommentary[0pt]{Klgl 2, 7-9} \gresetinitiallines{1} \grechangestyle{initial}{\fontsize{36}{36}\selectfont} \grechangedim{maxbaroffsettextleft@nobar}{12 cm}{scalable} \grechangedim{spaceabovelines}{0.5cm}{scalable} \gresetlyriccentering{syllable}   \gregorioscore{chants/va--lamentationes_63_zain_repulit--dominican}    \newpage
 \subsubsection{Zweite Lesung}             \vspace{10pt}

Johannes Chrysostomus († 407), Aus einer Katechese


\vspace{10pt}



\lettrine[lines=3]{W}{}illst du erfahren, welche Kraft das Blut Christi besitzt? Dann laß uns zurückgehen bis zu dem Vorausbild. Auf das frühe Vorausbild wollen wir uns besinnen und die Niederschrift aus der Vergangenheit erzählen. Mose sagt: „Tötet ein einjähriges Lamm und bestreicht mit seinem Blut die Tür.“ Was sagst du da, Mose? Kann denn das Blut eines Lammes den vernunftbegabten Menschen befreien? Gewiß, sagt er, weil es auf das Blut des Herrn verweist. Wenn der Feind nicht das Blut des Vorbildes an Pfosten, sondern auf den Lippen der Glaubenden das kostbare Blut der Wahrheit leuchten sieht, mit dem der Tempel Christi geweiht ist, dann weicht er viel weiter zurück. 

Willst du der Kraft dieses Blutes noch weiter nachforschen? Dann schau bitte, woher es kommt und aus welcher Quelle es entspringt. Vom Kreuz Christi kam es zuerst, aus der Seite Christi nahm es den Anfang. Denn das Evangelium berichtet: Als Jesus tot war und noch am Kreuz hing, kam ein Soldat herbei und stieß die Seite auf. Da floß Wasser und Blut heraus: Symbol der Taufe das eine, Symbol des Mysteriums (der Eucharistie) das andere. Der Soldat hat die Seite geöffnet und die Wand des Tempels aufgetan. Ich habe den herrlichen Schatz gefunden und bin glücklich, den glanzvollen Reichtum entdeckt zu haben. So war es auch mit dem Lamm: Die Juden haben es geschlachtet, und ich erfahre die Frucht des Opfers. 

Blut und Wasser aus der Seite. Lieber Hörer, bitte geh nicht eilig an dem verborgenen Mysterium vorbei. Denn ich muß noch mystische und geheime Dinge aussprechen: Ich sagte, dieses Wasser und Blut seien Sinnzeichen für die Taufe und das Mysterium. Daraus ist die heilige Kirche aufgebaut, durch die Wiedergeburt aus dem Wasser und die Erneuerung des Heiligen Geistes, ich sage euch: durch die Taufe und das Mysterium, das aus seiner Seite hervorging. Aus seiner Seite nämlich baute Christus die Kirche, wie aus der Seite Adams Eva, die Gattin, kam. 

Dafür ist auch Paulus Zeuge, wenn er sagt: „Wir sind Glieder seines Leibes“, von seinem Gebein genommen, womit er die Seite meint. Denn wie Gott aus der Seite des Adam die Frau schuf, so gab uns Christus aus seiner Seite Wasser und Blut, wodurch die Kirche erbaut werden sollte. Wie Gott die Seite öffnete, während Adam im Schlaf ruhte, so schenkte er uns jetzt nach dem Tode Christi aus seiner Seite das Wasser und das Blut.
    \newpage
\section{Laudes}
\subsubsection{Psalmodie} \noindent \textsc{1 Ant.} Seinen eigenen Sohn hat Gott nicht verschont: Er hat ihn hingegeben für uns alle.

\subsubsection{Psalm 51 (50), 3--21}

\noindent Gott, sei mir gnädig nach deiner Huld,~\GreStar{}~\nopagebreak

tilge meine Frevel nach deinem reichen Erbarmen!

\noindent Wasch meine Schuld von mir ab~\GreStar{}~\nopagebreak

und mach mich rein von meiner Sünde!

\noindent Denn ich erkenne meine bösen Taten,~\GreStar{}~\nopagebreak

meine Sünde steht mir immer vor Augen.

\noindent Gegen dich allein habe ich gesündigt,~\GreStar{}~\nopagebreak

ich habe getan, was dir missfällt. 

\noindent So behältst du recht mit deinem Urteil,~\GreStar{}~\nopagebreak

rein stehst du da als Richter.

\noindent Denn ich bin in Schuld geboren;~\GreStar{}~\nopagebreak

in Sünde hat mich meine Mutter empfangen.

\noindent Lauterer Sinn im Verborgenen gefällt dir,~\GreStar{}~\nopagebreak

im Geheimen lehrst du mich Weisheit.

\noindent Entsündige mich mit Ysop, dann werde ich rein;~\GreStar{}~\nopagebreak

wasche mich, dann werde ich weißer als Schnee.

\noindent Sättige mich mit Entzücken und Freude!~\GreStar{}~\nopagebreak

Jubeln sollen die Glieder, die du zerschlagen hast.

\noindent Verbirg dein Gesicht vor meinen Sünden;~\GreStar{}~\nopagebreak

tilge all meine Frevel!

\noindent Erschaffe mir, Gott, ein reines Herz~\GreStar{}~\nopagebreak

und gib mir einen neuen, beständigen Geist!

\noindent Verwirf mich nicht von deinem Angesicht~\GreStar{}~\nopagebreak

und nimm deinen heiligen Geist nicht von mir!

\noindent Mach mich wieder froh mit deinem Heil;~\GreStar{}~\nopagebreak

mit einem willigen Geist rüste mich aus!

\noindent Dann lehre ich Abtrünnige deine Wege,~\GreStar{}~\nopagebreak

und die Sünder kehren um zu dir.

\noindent Befrei mich von Blutschuld, Herr, du Gott meines Heiles,~\GreStar{}~\nopagebreak

dann wird meine Zunge jubeln über deine Gerechtigkeit.

\noindent Herr, öffne mir die Lippen,~\GreStar{}~\nopagebreak

und mein Mund wird deinen Ruhm verkünden.

\noindent Schlachtopfer willst du nicht, ich würde sie dir geben;~\GreStar{}~\nopagebreak

an Brandopfern hast du kein Gefallen.

\noindent Das Opfer, das Gott gefällt, ist ein zerknirschter Geist,~\GreStar{}~\nopagebreak

ein zerbrochenes und zerschlagenes Herz 

wirst du, Gott, nicht verschmähen.

\noindent In deiner Huld tu Gutes an Zion;~\GreStar{}~\nopagebreak

bau die Mauern Jerusalems wieder auf!

\noindent Dann hast du Freude an rechten Opfern,~†~\nopagebreak

an Brandopfern und Ganzopfern,~\GreStar{}~\nopagebreak

dann opfert man Stiere auf deinem Altar.

\noindent Ehre sei dem Vater und dem Sohn~\GreStar{}~\nopagebreak

und dem Heiligen Geist.

\noindent Wie im Anfang, so auch jetzt und alle Zeit~\GreStar{}~\nopagebreak

und in Ewigkeit. Amen.

\vspace{10pt}

\noindent \textsc{Ant.} Seinen eigenen Sohn hat Gott nicht verschont: Er hat ihn hingegeben für uns alle.

\vspace{10pt}

\noindent \textsc{2 Ant.} Jesus Christus hat uns geliebt und durch sein Blut von unseren Sünden befreit. \nopagebreak

\subsubsection{Canticum}

\paragraph{Hab 3, 2--4.13a.15--19}

\noindent Herr, ich höre die Kunde,~\GreStar{}~\nopagebreak

ich sehe, Herr, was du früher getan hast. 

\noindent Lass es in diesen Jahren wieder geschehen,~\GreStar{}~\nopagebreak

offenbare es in diesen Jahren! 

\noindent Auch wenn du zürnst,~\GreStar{}~\nopagebreak

denk an dein Erbarmen!

\noindent Gott kommt von Teman her,~\GreStar{}~\nopagebreak

der Heilige kommt vom Gebirge Paran. 

\noindent Seine Hoheit überstrahlt den Himmel,~\GreStar{}~\nopagebreak

sein Ruhm erfüllt die Erde.

\noindent Er leuchtet wie das Licht der Sonne,~†~\nopagebreak

ein Kranz von Strahlen umgibt ihn,~\GreStar{}~\nopagebreak

in ihnen verbirgt sich seine Macht.

\noindent Du ziehst aus, um dein Volk zu retten,~\GreStar{}~\nopagebreak

um deinem Gesalbten zu helfen.

\noindent Du bahnst mit deinen Rossen den Weg durch das Meer,~\GreStar{}~\nopagebreak

durch das gewaltig schäumende Wasser.

\noindent Ich zitterte am ganzen Leib, als ich es hörte,~\GreStar{}~\nopagebreak

ich vernahm den Lärm, und ich schrie. 

\noindent Fäulnis befällt meine Glieder,~\GreStar{}~\nopagebreak

und es wanken meine Schritte. 

\noindent Doch in Ruhe erwarte ich den Tag der Not,~\GreStar{}~\nopagebreak

der dem Volk bevorsteht, das über uns herfällt.

\noindent Zwar blüht der Feigenbaum nicht,~\GreStar{}~\nopagebreak

an den Reben ist nichts zu ernten, 

\noindent der Ölbaum bringt keinen Ertrag,~\GreStar{}~\nopagebreak

die Kornfelder tragen keine Frucht; 

\noindent im Pferch sind keine Schafe,~\GreStar{}~\nopagebreak

im Stall steht kein Rind mehr.

\noindent Dennoch will ich jubeln über den Herrn~\GreStar{}~\nopagebreak

und mich freuen über Gott, meinen Retter.

\noindent Gott, der Herr, ist meine Kraft.~†~\nopagebreak

Er macht meine Füße schnell wie die Füße der Hirsche~\GreStar{}~\nopagebreak

und lässt mich schreiten auf den Höhen.

\noindent Ehre sei dem Vater und dem Sohn~\GreStar{}~\nopagebreak

und dem Heiligen Geist.

\noindent Wie im Anfang, so auch jetzt und alle Zeit~\GreStar{}~\nopagebreak

und in Ewigkeit. Amen.

\vspace{10pt}

\noindent \textsc{Ant.} Jesus Christus hat uns geliebt und durch sein Blut von unseren Sünden befreit.

\vspace{10pt}

\noindent \textsc{3 Ant.} Dein Kreuz, o Herr, verehren wir, und deine heilige Auferstehung preisen und rühmen wir; denn siehe, durch das Holz des Kreuzes kam Freude in alle Welt.

\subsubsection{Psalm 47,12--20}

\noindent Jerusalem, preise den Herrn,~\GreStar{}~\nopagebreak

lobsinge, Zion, deinem Gott!

\noindent Denn er hat die Riegel deiner Tore festgemacht,~\GreStar{}~\nopagebreak

die Kinder in deiner Mitte gesegnet;

\noindent er verschafft deinen Grenzen Frieden~\GreStar{}~\nopagebreak

und sättigt dich mit bestem Weizen.

\noindent Er sendet sein Wort zur Erde,~\GreStar{}~\nopagebreak

rasch eilt sein Befehl dahin.

\noindent Er spendet Schnee wie Wolle,~\GreStar{}~\nopagebreak

streut den Reif aus wie Asche.

\noindent Eis wirft er herab in Brocken,~\GreStar{}~\nopagebreak

vor seiner Kälte erstarren die Wasser.

\noindent Er sendet sein Wort aus, und sie schmelzen,~\GreStar{}~\nopagebreak

er lässt den Wind wehen, dann rieseln die Wasser.

\noindent Er verkündet Jakob sein Wort,~\GreStar{}~\nopagebreak

Israel seine Gesetze und Rechte.

\noindent An keinem andern Volk hat er so gehandelt,~\GreStar{}~\nopagebreak

keinem sonst seine Rechte verkündet.

\noindent Ehre sei dem Vater und dem Sohn~\GreStar{}~\nopagebreak

und dem Heiligen Geist.

\noindent Wie im Anfang, so auch jetzt und alle Zeit~\GreStar{}~\nopagebreak

und in Ewigkeit. Amen.

\vspace{10pt}

\noindent \textsc{Ant.} Dein Kreuz, o Herr, verehren wir, und deine heilige Auferstehung preisen und rühmen wir; denn siehe, durch das Holz des Kreuzes kam Freude in alle Welt.
                 \newpage
 \subsubsection{Kurzlesung}     \hfill Jes 52,13-15        \lettrine[lines=3]{S}{}eht, mein Knecht hat Erfolg, er wird groß sein und hoch erhaben. Viele haben sich über ihn entsetzt, so entstellt sah er aus, nicht mehr wie ein Mensch, seine Gestalt war nicht mehr die eines Menschen. Jetzt aber setzt er viele Völker in Staunen, Könige müssen vor ihm verstummen. Denn was man ihnen noch nie erzählt hat, das sehen sie nun; was sie niemals hörten, das erfahren sie jetzt.
 \subsubsection{Responsorium}  \greannotation{V} \index[Responsorium]{Responsorium} \label{Responsorium (Responsorium)} \grecommentary[0pt]{} \gresetinitiallines{1} \grechangestyle{initial}{\fontsize{36}{36}\selectfont} \grechangedim{maxbaroffsettextleft@nobar}{12 cm}{scalable} \grechangedim{spaceabovelines}{0.5cm}{scalable} \gresetlyriccentering{syllable}   \gregorioscore{chants/gr--christus_factus_est--dominican--id_6737--only-respond}   \vspace{5pt} \emph{Christus war für uns gehorsam bis zum Tod, bis zum Tod am Kreuze.}
\subsubsection{Benedictus} \noindent \textsc{Benedictus-Ant.} Mit Sehnsucht habe ich danach verlangt, dieses Ostermahl mit euch zu halten, bevor ich leide.

\subsubsection{Benedictus}

\paragraph{Lk 1, 68-79}

\gabcsnippet{(f3) (e f hr0 fR ,3) (hr0 iRr1 fR : hr0 e f gRr1 fR ::)}

\noindent Gepriesen sei der Herr, der Gott \uline{I}sraels!~$\star$~\nopagebreak

Denn er hat sein Volk besucht und ihm Erl\uline{ö}sung geschaffen;

\noindent er hat uns einen starken Retter erw\uline{e}ckt~$\star$~\nopagebreak

im Hause seines Kn\uline{e}chtes David.

\noindent So hat er verheißen von \uline{a}lters her~$\star$~\nopagebreak

durch den Mund seiner heilig\uline{e}n Propheten.

\noindent Er hat uns errettet vor unseren F\uline{ei}nden~$\star$~\nopagebreak

und aus der Hand aller, d\uline{ie} uns hassen;

\noindent er hat das Erbarmen mit den Vätern an uns vollendet~†~\nopagebreak

und an seinen heiligen B\uline{u}nd gedacht~$\star$~\nopagebreak

an den Eid, den er unserm Vater Abrah\uline{a}m geschworen hat;

\noindent er hat uns geschenkt, dass wir, aus Feindeshand befreit,~†~\nopagebreak

ihm furchtlos dienen in Heiligkeit und Ger\uline{e}chtigkeit~$\star$~\nopagebreak

vor seinem Angesicht all \uline{u}nsre Tage.

\noindent Und du, Kind, wirst Prophet des Höchst\uline{en hei}ßen;~†~\nopagebreak

denn du wirst dem Herrn vor\uline{a}ngehen~$\star$~\nopagebreak

und ihm den W\uline{e}g bereiten.

\noindent Du wirst sein Volk mit der Erfahrung des Heils b\uline{e}schenken~$\star$~\nopagebreak

in der Vergeb\uline{u}ng der Sünden. 

\noindent Durch die barmherzige Liebe unseres G\uline{o}ttes~$\star$~\nopagebreak

wird uns besuchen das aufstrahlende Licht \uline{au}s der Höhe,

\noindent um allen zu leuchten, die in Finsternis sitzen und im Schatten des T\uline{o}des,~$\star$~\nopagebreak

und unsre Schritte zu lenken auf den W\uline{e}g des Friedens.

\noindent Ehre sei dem Vater und dem S\uline{o}hn~$\star$~\nopagebreak

und dem H\uline{ei}ligen Geist.

\noindent Wie im Anfang, so auch jetzt und \uline{a}lle Zeit~$\star$~\nopagebreak

und in Ew\uline{i}gkeit. Amen.

\vspace{5pt}

\noindent \textsc{Ant.} Mit Sehnsucht habe ich danach verlangt, dieses Ostermahl mit euch zu halten, bevor ich leide.
 \subsubsection{Preces}  \greannotation{} \index[Preces]{Holy Thursday} \label{Holy Thursday (Preces)} \grecommentary[3pt]{} \gresetinitiallines{1} \grechangestyle{initial}{\fontsize{36}{36}\selectfont} \grechangedim{maxbaroffsettextleft@nobar}{12 cm}{scalable} \grechangedim{spaceabovelines}{0.7cm}{scalable} \gresetlyriccentering{vowel}   \gregorioscore{chants/misc.versus_litanici_in_cantu_feria_vi--dominican-de-rubrics}
 \subsubsection{Vater Unser}   \index[Vater Unser]{Vater Unser} \label{Vater Unser (Vater Unser)}             \vspace{10pt}
 \subsubsection{Oration}   \index[Oration]{Oration} \label{Oration (Oration)}         \lettrine[lines=3]{H}{}err, unser Gott, sieh herab auf deine Familie, für die unser Herr Jesus Christus sich willig den Händen der Frevler überliefert und die Marter des Kreuzes auf sich genommen hat. Er, der in der Einheit des Heiligen Geistes mit dir lebt und herrscht in alle Ewigkeit.
\par \Rbar.~Amen.

 \subsubsection{Schlußsegen}   \index[Schlußsegen]{Schlußsegen} \label{Schlußsegen (Schlußsegen)}         Der Herr sei mit euch.

\Rbar. Und mit deinem Geiste.

Es segne euch der allmächtige Gott, + \par der Vater und der Sohn und der Heilige Geist.

\Rbar. Amen.

Gehet hin in Frieden.

\Rbar. Dank sei Gott, dem Herrn.    

  \end{document}
