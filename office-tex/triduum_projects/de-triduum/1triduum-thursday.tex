\documentclass[11pt,twoside]{book}

%%Page Size (rev. 08/19/2016)
%\usepackage[inner=0.5in, outer=0.5in, top=0.5in, bottom=0.5in, papersize={6in,9in}, head=12pt, headheight=30pt, headsep=5pt]{geometry}
\usepackage[inner=0.5in, outer=0.5in, top=0.5in, bottom=0.5in, papersize={5.5in,8.5in}, head=12pt, headheight=30pt, headsep=5pt]{geometry}
%% width of textblock = 324 pt / 4.5in
%% A5 = 5.8 x 8.3 inches -- if papersize is A5, then margins should be [inner=0.75in, outer=0.55in, top=0.4in, bottom=0.4in]


%%Header (rev. 4/11/2011)
\usepackage{fancyhdr}
 \pagestyle{fancy}
\renewcommand{\chaptermark}[1]{\markboth{#1}{}}
\renewcommand{\sectionmark}[1]{\markright{#1}}
 \fancyhf{}
\fancyhead[LE,RO]{\thepage}
\fancyhead[CE]{\leftmark}
\fancyhead[CO]{\rightmark}
 \fancypagestyle{plain}{ %
\fancyhf{} % remove everything
\renewcommand{\headrulewidth}{0pt} % remove lines as well
\renewcommand{\footrulewidth}{0pt}}



\usepackage[autocompile,allowdeprecated=false]{gregoriotex}
\usepackage{gregoriosyms}
\gresetgregoriofont[op]{greciliae}

%%Titles (rev. 9/4/2011) -- TOCLESS --- lets you have sections that don't appear in the table of contents

\setcounter{secnumdepth}{-1}

\usepackage[compact,nobottomtitles*]{titlesec}
\titlespacing*{\chapter}{0pt}{-30pt}{0pt}
\titlespacing*{\section}{0pt}{*0}{*1}
\titlespacing*{\subsubsection}{0pt}{*0}{*0}
\titlespacing*{\subsubsubsection}{0pt}{10pt}{*0}
\titleformat{\part} {\normalfont\Huge\sc\center}{\thechapter}{1em}{}
\titleformat{\chapter} {\normalfont\LARGE\sc\center}{\thechapter}{1em}{}
\titleformat{\section} {\normalfont\Large\sc\center}{\thesection}{1em}{}
\titleformat{\subsection} {\normalfont\Large\sc\center}{\thesubsubsection}{1em}{}
\titleformat{\subsubsection}{\normalfont\large\sc\center}{\thesubsubsubsection}{1em}{}
\titleformat{\paragraph}{\normalfont\normalsize\sc\center}{\thesubsubsubsection}{1em}{}

\newcommand{\nocontentsline}[3]{}
\newcommand{\tocless}[2]{\bgroup\let\addcontentsline=\nocontentsline#1{#2}\egroup} %% lets you have sections that don't appear in the table of contents


%%%

%%Index (rev. December 11, 2013)
\usepackage[noautomatic,nonewpage]{imakeidx}


\makeindex[name=incipit,title=Index]
\indexsetup{level=\section,toclevel=section,noclearpage}

\usepackage[indentunit=8pt,rule=.5pt,columns=2]{idxlayout}


%%Table of Contents (rev. May 16, 2011)

%\usepackage{multicol}
%\usepackage{ifthen}
%\usepackage[toc]{multitoc}

%% General settings (rev. January 19, 2015)

\usepackage[normalem]{ulem}

\usepackage[latin,german]{babel}
\usepackage{lettrine}

\usepackage{paracol}

\usepackage{fontspec}

\setmainfont[Ligatures=TeX,BoldFont=MinionPro-Bold,ItalicFont=MinionPro-It, BoldItalicFont=MinionPro-BoldIt]{MinionPro-Regular-Modified.otf}

%% Style for translation line
\grechangestyle{translation}{\fontsize{10}{10}\it\selectfont}
\grechangestyle{annotation}{\fontsize{10}{10}\selectfont}
\grechangestyle{commentary}{\textnormal\selectfont}
\gresetcustosalteration{invisible}

%\grechangedim{annotationseparation}{0.1cm}{scalable}

%\GreLoadSpaceConf{smith-four}

\frenchspacing

\usepackage{indentfirst} %%%indents first line after a section

\usepackage{graphicx}
%\usepackage{tocloft}

%%Hyperref (rev. August 20, 2011)
%\usepackage[colorlinks=false,hyperindex=true,bookmarks=true]{hyperref}
\usepackage{hyperref}
\hypersetup{pdftitle={Vesperale O.P. 2017}}
\hypersetup{pdfauthor={Order of Preachers}}
\hypersetup{pdfsubject={Liturgy}}
\hypersetup{pdfkeywords={Dominican, Liturgy, Order of Preachers, Dominican Rite, Liturgia Horarum, Divine Office}}

\newlength{\drop}



\begin{document}


\raggedbottom

\newcommand{\lectio}[3]{%
  \makebox[0pt][l]{#1}%
  \makebox[\textwidth][c]{#2}%
  \makebox[0pt][r]{\normalsize{\textnormal{#3}}}}


%%Combination
\chapter{Trauermette am Gründonnerstag}
\section{Officium Lectionis}
    \index[Varia]{Herr offne} \label{Herr offne (Varia)} \grecommentary[0pt]{} \gresetinitiallines{1} \grechangestyle{initial}{\fontsize{36}{36}\selectfont} \grechangedim{maxbaroffsettextleft@nobar}{12 cm}{scalable} \grechangedim{spaceabovelines}{0.5cm}{scalable} \gresetlyriccentering{syllable}  \grechangedim{maxbaroffsettextleft}{0 cm}{scalable} \gregorioscore{chants/herr_offne}
 \subsubsection{Invitatorium}   \index[Invitatorium]{Invitatorium} \label{Invitatorium (Invitatorium)}         \vspace{5pt} \par \gresetlyriccentering{syllable}

\emph{Cantor:}~
\nopagebreak
\greannotation{I}
\gabcsnippet{(c4) Im(d) Kreuz(h_) Je(g)su(h) Chri(j_)sti(h_) fin(g)den(f) wir(e_) Heil.(d_) (::)} \vspace{5pt}

\emph{Alle:}~
\nopagebreak
\gabcsnippet{(c4) Im(d) Kreuz(h_) Je(g)su(h) Chri(j_)sti(h_) fin(g)den(f) wir(e_) Heil.(d_) (::)} \vspace{5pt}

\emph{Cantor:}~
\nopagebreak
\gabcsnippet{Kommt(dh_), lasst(h) uns(h) ju(i)beln(h) vor(h) dem(g) Herrn(h) (,) und(j) zu(h)jauch(h)zen(g) dem(f) Fels(g) uns(g)res(h) Hei(f)les! (f) (:) 
Lasst(f) uns(f) mit(g) Lob(ixi_) sei(h)-nem(h) An(h)ge(h)sicht(f) na(gh)hen, (g) (,) vor(e) ihm(e) jau(g)chzen(g) mit(h) Lie(fe)dern! (d) <sp>R/</sp> (::)} \vspace{5pt}


\emph{Cantor:}~
\nopagebreak
\gabcsnippet{(c4)Denn(d) der(d) Herr(h_) ist(h) ein(h) gro(i)ßer(g) Gott,(h) (,) ein(h) gro(h)ßer(j) Kö(hi)nig(h) üb(h)er(h) all(g)en(h) Gö(f)ttern.(f) (:)  In(d) sei(f)ner(f) Hand(g_) sind(g) die(g) Tie(h)fen(g) der(h) Er(f)de,(g) (,) sein(f) sind(f) die(f) Gi(g)pfel(g) der(f) Ber(gh)ge.(h) (:) Sein(h) ist(g) das(h) Meer,(ixi_) das(h) er(h) ge(f)macht(g) hat,(g) (,) das(g) tro(h)cke(g)ne(f) Land,(g_) das(g) sei(g)ne(e) Hän(g)de(g)  ge(h)bil(fe)det.(d) (::)} \vspace{5pt}

\emph{Alle:}~
\nopagebreak
\gabcsnippet{(c4) Im(d) Kreuz(h_) Je(g)su(h) Chri(j_)sti(h_) fin(g)den(f) wir(e_) Heil.(d_) (::)} \vspace{5pt}

\emph{Cantor:}~
\nopagebreak

\gabcsnippet{(c4)Kommt(d_), lasst(d) uns(d) nie(h)der(h)fal(h)len(h), uns(h) vor(h) ihm(i) ver(g)nei(hi)gen,(h) (,) lasst(h) uns(j) nie(h)der(h)knien(h) vor(g) dem(f) Herrn,(g) un(g)serm(h) Schö(f)pfer!(f) (:) Denn(f) er(g) ist(g) un(f)ser(g) Gott,(h) (,) wir(h) sind(h) das(h) Volk(ixi_) sei(h)ner(f) Wei(gh)de,(g) (,) die(g) Her(fg)de,(g) von(e) sei(g)ner(h) Hand(f) ge(e)führt.(d) 
(::)}

 \vspace{5pt}


\emph{Alle:}~
\nopagebreak
\gabcsnippet{(c4) Im(d) Kreuz(h_) Je(g)su(h) Chri(j_)sti(h_) fin(g)den(f) wir(e_) Heil.(d_) (::)}

 \vspace{5pt}

\emph{Cantor:}~
\nopagebreak
\gabcsnippet{(c4) Ach,(d_) wür(d)det(d) ihr(d) doch(d) heu(dh)te(h) auf(h) sei(h)ne(h) Sti(i)mme(g) hö(hi)ren!(h) (;) „Ver(h)här(h)tet(h) eu(h)er(h) Herz(g) nicht(h) wie(h) in(h) Me(f)rí(g)ba,(g) (,) wie(h) in(h) der(j) Wü(hi)ste(h) am(h) Tag(g) von(h) Ma(f)ssa!(f) (:) Dort(f) ha(f)ben(f) eu(f)re(f) Vä(g)ter(g) mich(f) ver(g)sucht,(h) (,) sie(h) ha(h)ben(h) mich(h) auf(h) die(h) Pro(ixi)be(h) ge(f)stellt(g) und(g) hat(g)ten(g) doch(g) mein(e) Tun(g) ge(h)se(fe)hen.(d) 
(::)}

 \vspace{5pt}


\emph{Alle:}~
\nopagebreak
\gabcsnippet{(c4) Im(d) Kreuz(h_) Je(g)su(h) Chri(j_)sti(h_) fin(g)den(f) wir(e_) Heil.(d_) (::)}

 \vspace{5pt}

\emph{Cantor:}~\nopagebreak

\gabcsnippet{(c4) Vier(d)zig(d) Jah(dh)re(h) war(h) mir(h) dies(h) Ge(h)schlecht(i) zu(g)wi(hi)der,(h) (;) und(h) ich(f) sa(gh)gte:(g) Sie(h) sind(h) ein(j) Volk(h), de(h)ssen(h) Herz(h) in(g) die(f) Ir(g)re(g) geht;(g) (,) denn(h) mei(h)ne(j) We(hi)ge(h) ken(h)nen(g) sie(h) nicht.(f) (:) Da(f)rum(f) ha(f)be(f) ich(f) in(f) mei(f)nem(f) Zorn(g) ge(f)schwo(gh)ren:(h) (,) Sie(h) sol(ixi)len(h) nicht(f) kom(gh)men(g) in(g) das(e) Land(g) mei(g)ner(h) Ru(fe)he.”(d) (::)}

 \vspace{5pt}


\emph{Alle:}~\nopagebreak

\gabcsnippet{(c4) Im(d) Kreuz(h_) Je(g)su(h) Chri(j_)sti(h_) fin(g)den(f) wir(e_) Heil.(d_) (::)}

 \vspace{5pt}

\emph{Cantor:}~
\nopagebreak

\gabcsnippet{(c4) Eh(d)re(d) dem(d) Va(dh)ter,(h) (,) Eh(i)re(h) dem(g) Sohn,(h) (,) Eh(h)re(h) dem(j) Hei(h)li(g)gen(h) Geist.(f) (:) Wie(f) im(g) An(ixi)fang,(h) (,) so(h) auch(h) jetzt(h) und(f) al(g)le(h) Zeit(g) und(g) in(e) E(g)wig(g)keit.(h) A(fvED)men.(d) (::)}

 \vspace{5pt}


\emph{Alle:}~
\nopagebreak
\gabcsnippet{(c4) Im(d) Kreuz(h_) Je(g)su(h) Chri(j_)sti(h_) fin(g)den(f) wir(e_) Heil.(d_) (::)}
\subsubsection{Hymnus}  \greannotation{I} \index[Hymnus]{Hymnus} \label{Hymnus (Hymnus)} \grecommentary[0pt]{} \gresetinitiallines{1} \grechangestyle{initial}{\fontsize{36}{36}\selectfont} \grechangedim{maxbaroffsettextleft@nobar}{12 cm}{scalable} \grechangedim{spaceabovelines}{0.5cm}{scalable} \gresetlyriccentering{syllable}   

\gabcsnippet{
(c4) Heil(d)ig(e) Kreuz,(gh) du(hg) Baum(h) der(j) Treu(ji)e,(g) (,)
ed(i)ler(j) Baum,(kl) dem(hg) kein(j)er(i) gleich,(h) (;)
kein(hj)er(i) so(g) an(e) Laub(f) und(dc) Blü(de)te,(e) (,)
kein(d)er(h) so(hg) an(ed) Früch(f)ten(e) reich:(d) (;)
Sü(hj)ßes(i) Holz,(g) o(e) sü(f)ße(dc) Nä(de)gel,(e) (,)
wel(d)che(h) sü(hg)ße(ed) Last(f) an(e) euch.(d) (::)}

\gresetinitiallines{0}

\vspace{7pt}

\gabcsnippet{2. Beu(d)ge,(e) ho(gh)her(hg) Baum,(h) die(j) Zwei(ji)ge,(g) (,)
wer(i)de(j) weich(kl) an(hg) Stamm(j) und(i) Ast,(h) (;)
denn(hj) dein(i) har(g)tes(e) Holz(f) muß(dc) tra(de)gen(e) (,)
ei(d)ne(h) kö(hg)nig(ed)li(f)che(e) Last,(d) (;)
gib(hj) den(i) Glie(g)dern(e) dei(f)nes(dc) Schöp(de)fers(e) (,)
an(d) dem(h) Stam(hg)me(ed) lin(f)de(e) Rast.(d) (::)}

\vspace{7pt}

\gabcsnippet{3. Du(d) al(e)lein(gh) warst(hg) wert,(h) zu(j) tra(ji)gen(g) (,)
al(i)ler(j) Sün(kl)den(hg) Lö(j)se(i)geld,(h) (;)
du,(hj) die(i) Plan(g)ke,(e) die(f) uns(dc) ret(de)tet(e) (,)
aus(d) dem(h) Schiff(hg)bruch(ed) die(f)ser(e) Welt.(d) (;)
Du,(hj) ge(i)salbt(g) vom(e) Blut(f) des(dc) Lam(de)mes,(e) (,)
Pfos(d)ten,(h) der(hg) den(ed) Tod(f) ab(e)hält.(d) (::)}

\vspace{7pt}

\gabcsnippet{4. Lob(d) und(e) Ruhm(gh) sei(hg) oh(h)ne(j) En(ji)de(g) (,)
Gott,(i) dem(j) höchs(kl)ten(hg) Herrn,(j) ge(i)weiht.(h) (;)
Preis(hj) dem(i) Va(g)ter(e) und(f) dem(dc) Soh(de)ne(e) (,)
und(d) dem(h) Geist(hg) der(ed) Hei(f)lig(e)keit.(d) (;)
Ei(hj)nen(i) Gott(g) in(e) drei(f) Per(dc)so(de)nen(e) (,)
lo(d)be(h) al(hg)le(ed) Welt(f) und(e) Zeit.(d) (::)
A(ded)men.(cd) (::)}                 \newpage
\subsection{Psalmodie} \noindent \textsc{1 Ant.} Du hast uns gerettet, Herr, wir preisen deinen Namen auf ewig.

\subsubsection{Psalm 44 (43)} \paragraph{I}

\noindent Gott, wir hörten es mit eigenen Ohren,~$\star$~\nopagebreak

unsere Väter erzählten uns

\noindent von dem Werk, das du in ihren Tagen vollbracht hast,~$\star$~\nopagebreak

in den Tagen der Vorzeit.

\noindent Mit eigener Hand hast du Völker vertrieben,~$\star$~\nopagebreak

sie aber eingepflanzt.

\noindent Du hast Nationen zerschlagen,~$\star$~\nopagebreak

sie aber ausgesät.

\noindent Denn sie gewannen das Land nicht mit ihrem Schwert,~$\star$~\nopagebreak

noch verschaffte ihr Arm ihnen den Sieg;

\noindent nein, deine Rechte war es, dein Arm und dein leuchtendes Angesicht;~$\star$~\nopagebreak

denn du hattest an ihnen Gefallen.

\noindent Du, mein König und mein Gott,~$\star$~\nopagebreak

du bist es, der Jakob den Sieg verleiht.

\noindent Mit dir stoßen wir unsere Bedränger nieder,~$\star$~\nopagebreak

in deinem Namen zertreten wir unsere Gegner.

\noindent Denn ich verlasse mich nicht auf meinen Bogen,~$\star$~\nopagebreak

noch kann mein Schwert mir helfen;

\noindent nein, du hast uns vor unsern Bedrängern gerettet;~$\star$~\nopagebreak

alle, die uns hassen, bedeckst du mit Schande.

\noindent Wir rühmen uns Gottes den ganzen Tag~$\star$~\nopagebreak

und preisen deinen Namen auf ewig.

\noindent Ehre sei dem Vater und dem Sohn~$\star$~\nopagebreak

und dem Heiligen Geist.

\noindent Wie im Anfang, so auch jetzt und alle Zeit~$\star$~\nopagebreak

und in Ewigkeit. Amen.

\vspace{10pt}

\noindent \textsc{Ant.} Du hast uns gerettet, Herr, wir preisen deinen Namen auf ewig. 

%\vspace{10pt}

\newpage

\noindent \textsc{2 Ant.} Verschone dein Volk, o Herr; gib dein Erbe nicht der Schande preis.

\paragraph{II}

\noindent Doch nun hast du uns verstoßen und mit Schmach bedeckt,~$\star$~\nopagebreak

du ziehst nicht mit unserm Heer in den Kampf. 

 \noindent Du lässt uns vor unsern Bedrängern fliehen~$\star$~\nopagebreak

und Menschen, die uns hassen, plündern uns aus. 

 \noindent Du gibst uns preis wie Schlachtvieh,~$\star$~\nopagebreak

unter die Völker zerstreust du uns. 

 \noindent Du verkaufst dein Volk um ein Spottgeld~$\star$~\nopagebreak

und hast an dem Erlös keinen Gewinn. 

 \noindent Du machst uns zum Schimpf für die Nachbarn,~$\star$~\nopagebreak

zu Spott und Hohn bei allen, die rings um uns wohnen.

\noindent Du machst uns zum Spottlied der Völker,~$\star$~\nopagebreak

die Heiden zeigen uns nichts als Verachtung.

\noindent Meine Schmach steht mir allzeit vor Augen~$\star$~\nopagebreak

und Scham bedeckt mein Gesicht

\noindent wegen der Worte des lästernden Spötters,~$\star$~\nopagebreak

wegen der rachgierigen Blicke des Feindes.

\noindent Ehre sei dem Vater und dem Sohn~$\star$~\nopagebreak

und dem Heiligen Geist.

\noindent Wie im Anfang, so auch jetzt und alle Zeit~$\star$~\nopagebreak

und in Ewigkeit. Amen.

\vspace{10pt}

\noindent \textsc{Ant.} Verschone dein Volk, o Herr; gib dein Erbe nicht der Schande preis. \vspace{10pt}

\noindent \textsc{3 Ant.} Steh auf und hilf uns, Herr; in deiner Huld erlöse uns.

\paragraph{III}

\noindent Das alles ist über uns gekommen~†~\nopagebreak

und doch haben wir dich nicht vergessen,~$\star$~\nopagebreak

uns von deinem Bund nicht treulos abgewandt.

\noindent Unser Herz ist nicht von dir gewichen,~$\star$~\nopagebreak

noch hat unser Schritt deinen Pfad verlassen.

\noindent Doch du hast uns verstoßen an den Ort der Schakale~$\star$~\nopagebreak

und uns bedeckt mit Finsternis.

\noindent Hätten wir den Namen unseres Gottes vergessen~$\star$~\nopagebreak

und zu einem fremden Gott die Hände erhoben,

\noindent würde Gott das nicht ergründen?~$\star$~\nopagebreak

Denn er kennt die heimlichen Gedanken des Herzens.

\noindent Nein, um deinetwillen werden wir getötet Tag für Tag,~$\star$~\nopagebreak

behandelt wie Schafe, die man zum Schlachten bestimmt hat. 

 \noindent Wach auf! Warum schläfst du, Herr?~$\star$~\nopagebreak

Erwache, verstoß nicht für immer! 

 \noindent Warum verbirgst du dein Gesicht,~$\star$~\nopagebreak

vergisst unsere Not und Bedrängnis? 

 \noindent Unsere Seele ist in den Staub hinabgebeugt,~$\star$~\nopagebreak

unser Leib liegt am Boden.

\noindent Steh auf und hilf uns!~$\star$~\nopagebreak

In deiner Huld erlöse uns!

\noindent Ehre sei dem Vater und dem Sohn~$\star$~\nopagebreak

und dem Heiligen Geist.

\noindent Wie im Anfang, so auch jetzt und alle Zeit~$\star$~\nopagebreak

und in Ewigkeit. Amen.

\vspace{10pt}

\noindent \textsc{Ant.} Steh auf und hilf uns, Herr; in deiner Huld erlöse uns.

 \subsubsection{Versiculum}  \greannotation{} \index[Versiculum]{Versiculum} \label{Versiculum (Versiculum)} \grecommentary[0pt]{} \gresetinitiallines{0} \grechangestyle{initial}{\fontsize{36}{36}\selectfont} \grechangedim{maxbaroffsettextleft@nobar}{12 cm}{scalable} \grechangedim{spaceabovelines}{0.5cm}{scalable} \gresetlyriccentering{syllable}   \gregorioscore{chants/versiculum_thursday}    \newpage
\subsection{Lesungen}
 \subsubsection{Erste Lesung} \vspace{5pt} \emph{Anfang der Klagelieder des Propheten Jeremia. Aleph. Weh, wie einsam sitzt da die einst so volkreiche Stadt. Einer Witwe wurde gleich die Große unter den Völkern. Die Fürstin über die Länder ist zur Fron erniedrigt. Beth. Sie weint und weint des Nachts, Tränen auf ihren Wangen. Keinen hat sie als Tröster von all ihren Geliebten. Untreu sind all ihre Freunde, sie sind ihr zu Feinden geworden. Ghimel. Gefangen ist Juda im Elend, in harter Knechtschaft. Nun weilt sie unter den Völkern und findet nicht Ruhe. All ihre Verfolger holten sie ein mitten in der Bedrängnis. - Jerusalem, Jerusalem, bekehre dich zum Herrn, Deinem Gott.} \vspace{5pt} \greannotation{} \index[Erste Lesung]{Erste Lesung} \label{Erste Lesung (Erste Lesung)} \grecommentary[0pt]{Klgl 1, 1--3} \gresetinitiallines{1} \grechangestyle{initial}{\fontsize{36}{36}\selectfont} \grechangedim{maxbaroffsettextleft@nobar}{12 cm}{scalable} \grechangedim{spaceabovelines}{0.5cm}{scalable} \gresetlyriccentering{vowel}   \gregorioscore{chants/va--incipit_lamentatio_ieremiae_prophetae--dominican}
 \subsubsection{II} \vspace{5pt} \emph{Daleth. Die Wege nach Zion trauern, niemand pilgert zum Fest, verödet sind all ihre Tore. Ihre Priester seufzen, ihre Jungfrauen sind voll Gram, sie selbst trägt Weh und Kummer. He. Ihre Bedränger sind an der Macht, ihre Feinde im Glück. Denn Trübsal hat der Herr ihr gesandt wegen ihrer vielen Sünden. Ihre Kinder zogen fort, gefangen, vor dem Bedränger. Gewichen ist von der Tochter Zion all ihre Pracht. Vau. Ihre Fürsten sind wie Hirsche geworden, die keine Weide finden. Kraftlos zogen sie dahin vor ihren Verfolgern. - Jerusalem, Jerusalem, bekehre dich zum Herrn, deinem Gott.} \vspace{5pt} \greannotation{} \index[II]{II} \label{II (II)} \grecommentary[0pt]{Klgl 1, 4--6} \gresetinitiallines{1} \grechangestyle{initial}{\fontsize{36}{36}\selectfont} \grechangedim{maxbaroffsettextleft@nobar}{12 cm}{scalable} \grechangedim{spaceabovelines}{0.5cm}{scalable} \gresetlyriccentering{vowel}   \gregorioscore{chants/va--daleth_viae_sion_lugent--dominican}
 \subsubsection{III} \vspace{5pt} \emph{Zain. Jerusalem denkt an die Tage ihres Elends, ihrer Unrast, an all ihre Kostbarkeiten, die sie einst besessen, als ihr Volk in Feindeshand fiel und keiner ihr beistand. Die Feinde sahen sie an, lachten über ihre Vernichtung. Heth. Schwer gesündigt hatte Jerusalem, deshalb ist sie zum Abscheu geworden. All ihre Verehrer verachten sie, weil sie ihre Blöße gesehen. Sie selbst aber seufzt und wendet sich ab (von ihnen). Teth. Ihre Unreinheit klebt an ihrer Schleppe, ihr Ende bedachte sie nicht. Entsetzlich ist sie gesunken, keinen hat sie als Tröster. Sieh doch mein Elend, o Herr, denn die Feinde prahlen. - Jerusalem, Jerusalem, bekehre dich zum Herrn, deinem Gott.}  \vspace{5pt} \greannotation{} \index[III]{III} \label{III (III)} \grecommentary[0pt]{Klgl 1, 7--9} \gresetinitiallines{1} \grechangestyle{initial}{\fontsize{36}{36}\selectfont} \grechangedim{maxbaroffsettextleft@nobar}{12 cm}{scalable} \grechangedim{spaceabovelines}{0.5cm}{scalable} \gresetlyriccentering{vowel}   \gregorioscore{chants/va--zain_recordata_est_ierusalem--dominican}    \newpage
 \subsubsection{Zweite Lesung}             \vspace{5pt}

Meliton von Sardes († vor 190), Aus einer Osterpredigt

\vspace{5pt}

%\par \hfill(Nn. 65071: SC 123, 95-101)

\lettrine[lines=3]{D}{}ie Propheten haben vieles vorausverkündigt über das Paschamysterium, das Christus ist, „dem Ehre sei in alle Ewigkeit. Amen“. Er kam vom Himmel auf die Erde wegen des leidenden Menschen; den leidenden Menschen zog er wie ein Kleid an im Schoß der Jungfrau und ging hervor als Mensch; durch einen Leib, der dem Leiden ausgesetzt war, nahm er die Leiden des leidenden Menschen auf sich und vernichtete die Leiden des Fleisches. Durch den Geist aber, der nicht sterben konnte, tötete er den Mörder Tod.

Er wurde zum Schlachten geführt wie ein Lamm und getötet wie ein Schaf. Wie aus einem Ägypten erlöste er uns aus dem Dienst der Welt. Er rettete uns aus der Knechtschaft des Teufels wie aus der Hand des Pharaos; er besiegelte unsere Seelen mit seinem eigenen Geist und die Glieder unseres Leibes mit seinem Blut. Er ist es, der Verwirrung über den Tod brachte und den Teufel in Trauer versetzte wie Mose den Pharao. Er schlug die Bosheit und verdammte die Ungerechtigkeit zur Unfruchtbarkeit wie Mose Ägypten.

Er ist es, der uns der Knechtschaft entrissen und uns befreit hat, der uns aus der Finsternis zum Licht führte, vom Tod zum Leben, von der Gewaltherrschaft zu ewigem Königtum, der uns zu einer neuen Priesterschaft machte, zu einem erwählten und ewigen Volk. Er ist das Paschalamm unseres Heils. Er ertrug in vielen vieles: Er wurde in Abel gemordet. 

In Isaak wurden ihm die Füße gefesselt, in Jakob mußte er auswandern. In Josef wurde er verkauft, in Mose ausgesetzt, im Lamm geschlachtet, in David verfolgt, in den Propheten geschmäht. Er wurde Mensch in der Jungfrau, ans Holz gehängt, in das Grab der Erde gesenkt. Er erstand von den Toten und stieg empor zur Höhe des Himmels. Er ist das Lamm, das verstummt, aus der Herde geholt, zum Schlachten geführt, am Abend geopfert und in der Nacht begraben. Am Holz zerbrach man ihn nicht und im Grab verweste er nicht. Er stand von den Toten auf und erweckte den Menschen aus dem Grab der Unterwelt.    \newpage
\section{Laudes}
\subsubsection{Psalmodie} \noindent \textsc{1 Ant.} Sieh her, mein Gott, verbirg nicht dein Gesicht, denn mir ist angst, erhöre mich bald.

\subsubsection{Psalm 80}

\noindent Du Hirte Israels, höre,~$\star$~\nopagebreak

der du Josef weidest wie eine Herde!

\noindent Der du auf den Kerubim thronst, erscheine~$\star$~\nopagebreak

vor Efraim, Benjamin und Manasse!

\noindent Biete deine gewaltige Macht auf~$\star$~\nopagebreak

und komm uns zu Hilfe!

\noindent Gott, richte uns wieder auf!~$\star$~\nopagebreak

Lass dein Angesicht leuchten, dann ist uns geholfen.

\noindent Herr, Gott der Heerscharen, wie lange noch zürnst du,~$\star$~\nopagebreak

während dein Volk zu dir betet?

\noindent Du hast sie gespeist mit Tränenbrot,~$\star$~\nopagebreak

sie überreich getränkt mit Tränen.

\noindent Du machst uns zum Spielball der Nachbarn~$\star$~\nopagebreak

und unsere Feinde verspotten uns.

\noindent Gott der Heerscharen, richte uns wieder auf!~$\star$~\nopagebreak

Lass dein Angesicht leuchten, dann ist uns geholfen.

\noindent Du hobst in Ägypten einen Weinstock aus,~$\star$~\nopagebreak

du hast Völker vertrieben, ihn aber eingepflanzt.

\noindent Du schufst ihm weiten Raum;~$\star$~\nopagebreak

er hat Wurzeln geschlagen und das ganze Land erfüllt.

\noindent Sein Schatten bedeckte die Berge,~$\star$~\nopagebreak

seine Zweige die Zedern Gottes.

\noindent Seine Ranken trieb er hin bis zum Meer~$\star$~\nopagebreak

und seine Schößlinge bis zum Eufrat.

\noindent Warum rissest du seine Mauern ein?~$\star$~\nopagebreak

Alle, die des Weges kommen, plündern ihn aus.

\noindent Der Eber aus dem Wald wühlt ihn um,~$\star$~\nopagebreak

die Tiere des Feldes fressen ihn ab.

\noindent Gott der Heerscharen, wende dich uns wieder zu!~$\star$~\nopagebreak

Blick vom Himmel herab, und sieh auf uns!

\noindent Sorge für diesen Weinstock~$\star$~\nopagebreak

und für den Garten, den deine Rechte gepflanzt hat.

\noindent Die ihn im Feuer verbrannten wie Kehricht,~$\star$~\nopagebreak

sie sollen vergehen vor deinem drohenden Angesicht.

\noindent Deine Hand schütze den Mann zu deiner Rechten,~$\star$~\nopagebreak

den Menschensohn, den du für dich groß und stark gemacht.

\noindent Erhalt uns am Leben!~$\star$~\nopagebreak

Dann wollen wir deinen Namen anrufen und nicht von dir weichen. 

\noindent Herr, Gott der Heerscharen, richte uns wieder auf!~$\star$~\nopagebreak

Lass dein Angesicht leuchten, dann ist uns geholfen.

\noindent Ehre sei dem Vater und dem Sohn~$\star$~\nopagebreak

und dem Heiligen Geist.

\noindent Wie im Anfang, so auch jetzt und alle Zeit~$\star$~\nopagebreak

und in Ewigkeit. Amen.

\vspace{10pt}

\noindent \textsc{Ant.} Sieh her, mein Gott, verbirg nicht dein Gesicht, denn mir ist angst, erhöre mich bald. 

\vspace{10pt}

\noindent \textsc{2 Ant.} Gott ist mein Retter; ihm will ich vertrauen und niemals verzagen. \nopagebreak

\subsubsection{Canticum}

\paragraph{Jes 12,1-6}

\noindent Ich danke dir, Herr.~†~\nopagebreak

Du hast mir gezürnt, doch dein Zorn hat sich gewendet~$\star$~\nopagebreak

und du hast mich getröstet.

\noindent Ja, Gott ist meine Rettung;~$\star$~\nopagebreak

ihm will ich vertrauen und niemals verzagen.

\noindent Denn meine Stärke und mein Lied ist der Herr.~$\star$~\nopagebreak

Er ist für mich zum Retter geworden.

\noindent Ihr werdet Wasser schöpfen voll Freude~$\star$~\nopagebreak

aus den Quellen des Heils.

\noindent An jenem Tag werdet ihr sagen:~$\star$~\nopagebreak

Dankt dem Herrn! Ruft seinen Namen an!

\noindent Macht seine Taten unter den Völkern bekannt,~$\star$~\nopagebreak

verkündet: Sein Name ist groß und erhaben!

\noindent Preist den Herrn, denn herrliche Taten hat er vollbracht;~$\star$~\nopagebreak

auf der ganzen Erde soll man es wissen.

\noindent Jauchzt und jubelt, ihr Bewohner von Zion,~$\star$~\nopagebreak

denn groß ist in eurer Mitte der Heilige Israels.

\noindent Ehre sei dem Vater und dem Sohn~$\star$~\nopagebreak

und dem Heiligen Geist.

\noindent Wie im Anfang, so auch jetzt und alle Zeit~$\star$~\nopagebreak

und in Ewigkeit. Amen.

\vspace{10pt}

\noindent \textsc{Ant.} Gott ist mein Retter; ihm will ich vertrauen und niemals verzagen. 

\vspace{10pt}

\noindent \textsc{3 Ant.} Mit bestem Weizen nährt uns der Herr und sättigt uns mit Honig aus dem Felsen.

\subsubsection{Psalm 81}

\noindent Jubelt Gott zu, er ist unsre Zuflucht;~$\star$~\nopagebreak

jauchzt dem Gott Jakobs zu!

\noindent Stimmt an den Gesang, schlagt die Pauke,~$\star$~\nopagebreak

die liebliche Laute, dazu die Harfe! 

\noindent Stoßt in die Posaune am Neumond~$\star$~\nopagebreak

und zum Vollmond, am Tag unsres Festes!

\noindent Denn das ist Satzung für Israel,~$\star$~\nopagebreak

Entscheid des Gottes Jakobs.

\noindent Das hat er als Gesetz für Josef erlassen,~$\star$~\nopagebreak

als Gott gegen Ägypten auszog.

\noindent Eine Stimme höre ich, die ich noch nie vernahm:~†~\nopagebreak

Seine Schulter hab ich von der Bürde befreit,~$\star$~\nopagebreak

seine Hände kamen los vom Lastkorb.

\noindent Du riefst in der Not~$\star$~\nopagebreak

und ich riss dich heraus;

\noindent ich habe dich aus dem Gewölk des Donners erhört,~$\star$~\nopagebreak

an den Wassern von Meríba geprüft.

\noindent Höre, mein Volk, ich will dich mahnen!~$\star$~\nopagebreak

Israel, wolltest du doch auf mich hören!

\noindent Für dich gibt es keinen andern Gott.~$\star$~\nopagebreak

Du sollst keinen fremden Gott anbeten.

\noindent Ich bin der Herr, dein Gott,~†~\nopagebreak

der dich heraufgeführt hat aus Ägypten.~$\star$~\nopagebreak

Tu deinen Mund auf! Ich will ihn füllen.

\noindent Doch mein Volk hat nicht auf meine Stimme gehört;~$\star$~\nopagebreak

Israel hat mich nicht gewollt.

\noindent Da überließ ich sie ihrem verstockten Herzen~$\star$~\nopagebreak

und sie handelten nach ihren eigenen Plänen. 

\noindent Ach dass doch mein Volk auf mich hörte,~$\star$~\nopagebreak

dass Israel gehen wollte auf meinen Wegen!

\noindent Wie bald würde ich seine Feinde beugen,~$\star$~\nopagebreak

meine Hand gegen seine Bedränger wenden. 

\noindent Alle, die den Herrn hassen, müssten Israel schmeicheln~$\star$~\nopagebreak

und das sollte für immer so bleiben. 

\noindent Ich würde es nähren mit bestem Weizen~$\star$~\nopagebreak

und mit Honig aus dem Felsen sättigen.

\noindent Ehre sei dem Vater und dem Sohn~$\star$~\nopagebreak

und dem Heiligen Geist.

\noindent Wie im Anfang, so auch jetzt und alle Zeit~$\star$~\nopagebreak

und in Ewigkeit. Amen.

\vspace{10pt}

\noindent \textsc{Ant.} Mit bestem Weizen nährt uns der Herr und sättigt uns mit Honig aus dem Felsen.
 \subsubsection{Kurzlesung}     \hfill Hebr 2, 9b--10        \lettrine[lines=3]{W}{}ir sehen Jesus um seines Todesleidens willen mit Herrlichkeit und Ehre gekrönt; es war nämlich Gottes gnädiger Wille, dass er für alle den Tod erlitt. Denn es war angemessen, dass Gott, für den und durch den das All ist und der viele Söhne zur Herrlichkeit führen wollte, den Urheber ihres Heils durch Leiden vollendete.
 \subsubsection{Responsorium}             \Rbar. Mit deinem heiligen Blute \GreStar{} hast du uns losgekauft.

\Rbar. Mit deinem heiligen Blute \GreStar{} hast du uns losgekauft.

\Vbar. Aus allen Stämmen und Sprachen. 

\Rbar. Hast du uns losgekauft.

\Vbar. Ehre sei dem Vater und dem Sohn und dem Heiligen Geist.

\Rbar. Mit deinem heiligen Blute \GreStar{} hast du uns losgekauft.
    \newpage
\subsubsection{Benedictus} \noindent \textsc{Benedictus-Ant.} Mit Sehnsucht habe ich danach verlangt, dieses Ostermahl mit euch zu halten, bevor ich leide.

\subsubsection{Benedictus}

\paragraph{Lk 1, 68-79}

\gabcsnippet{(f3) (e f hr0 fR ,3) (hr0 iRr1 fR : hr0 e f gRr1 fR ::)}

\noindent Gepriesen sei der Herr, der Gott \uline{I}sraels!~$\star$~\nopagebreak

Denn er hat sein Volk besucht und ihm Erl\uline{ö}sung geschaffen;

\noindent er hat uns einen starken Retter erw\uline{e}ckt~$\star$~\nopagebreak

im Hause seines Kn\uline{e}chtes David.

\noindent So hat er verheißen von \uline{a}lters her~$\star$~\nopagebreak

durch den Mund seiner heilig\uline{e}n Propheten.

\noindent Er hat uns errettet vor unseren F\uline{ei}nden~$\star$~\nopagebreak

und aus der Hand aller, d\uline{ie} uns hassen;

\noindent er hat das Erbarmen mit den Vätern an uns vollendet~†~\nopagebreak

und an seinen heiligen B\uline{u}nd gedacht~$\star$~\nopagebreak

an den Eid, den er unserm Vater Abrah\uline{a}m geschworen hat;

\noindent er hat uns geschenkt, dass wir, aus Feindeshand befreit,~†~\nopagebreak

ihm furchtlos dienen in Heiligkeit und Ger\uline{e}chtigkeit~$\star$~\nopagebreak

vor seinem Angesicht all \uline{u}nsre Tage.

\noindent Und du, Kind, wirst Prophet des Höchst\uline{en hei}ßen;~†~\nopagebreak

denn du wirst dem Herrn vor\uline{a}ngehen~$\star$~\nopagebreak

und ihm den W\uline{e}g bereiten.

\noindent Du wirst sein Volk mit der Erfahrung des Heils b\uline{e}schenken~$\star$~\nopagebreak

in der Vergeb\uline{u}ng der Sünden. 

\noindent Durch die barmherzige Liebe unseres G\uline{o}ttes~$\star$~\nopagebreak

wird uns besuchen das aufstrahlende Licht \uline{au}s der Höhe,

\noindent um allen zu leuchten, die in Finsternis sitzen und im Schatten des T\uline{o}des,~$\star$~\nopagebreak

und unsre Schritte zu lenken auf den W\uline{e}g des Friedens.

\noindent Ehre sei dem Vater und dem S\uline{o}hn~$\star$~\nopagebreak

und dem H\uline{ei}ligen Geist.

\noindent Wie im Anfang, so auch jetzt und \uline{a}lle Zeit~$\star$~\nopagebreak

und in Ew\uline{i}gkeit. Amen.

\vspace{5pt}

\noindent \textsc{Ant.} Mit Sehnsucht habe ich danach verlangt, dieses Ostermahl mit euch zu halten, bevor ich leide.
 \subsubsection{Preces}  \greannotation{} \index[Preces]{Holy Thursday} \label{Holy Thursday (Preces)} \grecommentary[3pt]{} \gresetinitiallines{1} \grechangestyle{initial}{\fontsize{36}{36}\selectfont} \grechangedim{maxbaroffsettextleft@nobar}{12 cm}{scalable} \grechangedim{spaceabovelines}{0.7cm}{scalable} \gresetlyriccentering{vowel}   \gregorioscore{chants/misc.versus_litanici_in_cantu_feria_v}
 \subsubsection{Vater Unser}   \index[Vater Unser]{Vater Unser} \label{Vater Unser (Vater Unser)}             \vspace{10pt}
 \subsubsection{Oration}   \index[Oration]{Oration} \label{Oration (Oration)}         \lettrine[lines=3]{G}{}ott, es ist würdig und recht, dich über alles zu lieben. Mehre in uns den Reichtum deiner Gnade. Durch den Tod deines Sohnes lässt du uns erhoffen, was wir glauben. Gib, dass wir durch seine Auferstehung erlangen, was wir ersehnen. Darum bitten wir durch ihn, Jesus Christus.
\par \Rbar.~Amen.

 \subsubsection{Schlußsegen}   \index[Schlußsegen]{Schlußsegen} \label{Schlußsegen (Schlußsegen)}         Der Herr sei mit euch.

\Rbar. Und mit deinem Geiste.

Es segne euch der allmächtige Gott, + \par der Vater und der Sohn und der Heilige Geist.

\Rbar. Amen.

Gehet hin in Frieden.

\Rbar. Dank sei Gott, dem Herrn.

  \end{document}
