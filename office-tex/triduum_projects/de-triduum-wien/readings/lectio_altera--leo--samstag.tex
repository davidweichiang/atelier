\vspace{10pt}

Leo der Große († 461), Aus einer Predigt über die Passion des Herrn

\vspace{10pt}

%\par \hfill(Nn. 65071: SC 123, 95-101)

\lettrine[lines=3]{M}{}it allen Kräften unseres Geistes und unseres Leibes müssen wir darnach trachten, unzertrennlich mit dem Geheimnis des Leidens Christi verbunden zu bleiben; denn der Herr sagt: „Wer nicht sein Kreuz auf sich nimmt und mir nachfolgt, ist meiner nicht würdig.“ Und der Apostel spricht: „Wenn wir mit ihm leiden, werden wir auch mit ihm verherrlicht werden.“ Wer anders erweist also nach diesen Worten dem wahrhaft leidenden, sterbenden und auferstehenden Christus seine Verehrung, als wer mit ihm leidet, stirbt und aufersteht? Diese Teilnahme an dem Leiden des Herrn hat bei allen Kindern der Kirche schon mit ihrer wunderbaren Wiedergeburt begonnen: Durch die Tilgung der Sünde ersteht hier der Mensch zu neuem Leben, und durch das dreimalige Untertauchen wird der dreitägige Tod des Herrn versinnbildet. Bei der Taufe wird gleichsam die Erddecke von einem Grabe entfernt. Mit unserem alten Menschen steigen wir in den Taufquell hinab, und neugeboren kommen wir aus ihm hervor. Was aber durch dieses Sakrament mit uns begonnen wurde, das müssen wir durch Taten vollenden. Die ganze Lebenszeit, die den im Heiligen Geiste Wiedergeborenen noch übrigbleibt, muß ein beständiges Tragen des Kreuzes sein. Obgleich nämlich durch die Macht des Leidens Christi dem starken und grausamen Feinde unseres Geschlechtes die „Gefäße der alten Erbeutung“ entrissen wurden und „der Herrscher dieser Welt“ über die Herzen der Erlösten keine Gewalt mehr hat, verfolgt er doch die Menschen selbst nach ihrer Rechtfertigung immer noch mit seiner alten Bosheit. Auf mancherlei Art greift er die an, in denen er nicht mehr herrscht, um nachlässige und sorglose Seelen aufs neue mit noch grausameren Banden an sich zu ketten, um sie aus dem Paradies der Kirche zu vertreiben und sie zu Genossen seiner Verdammnis zu machen. Wenn darum jemand merkt, daß er die Grenzen der christlichen Gebote überschreitet und daß seine Begierden auf Dinge gerichtet sind, die ihn vom rechten Wege abbringen könnten, so nehme er seine Zuflucht zum Kreuze des Herrn und kreuzige sein sündhaftes Wollen und Wünschen auf dem Baume des Lebens.
