\vspace{10pt}

Johannes Chrysostomus († 407), Aus einer Katechese


\vspace{10pt}



\lettrine[lines=3]{W}{}illst du erfahren, welche Kraft das Blut Christi besitzt? Dann laß uns zurückgehen bis zu dem Vorausbild. Auf das frühe Vorausbild wollen wir uns besinnen und die Niederschrift aus der Vergangenheit erzählen. Mose sagt: „Tötet ein einjähriges Lamm und bestreicht mit seinem Blut die Tür.“ Was sagst du da, Mose? Kann denn das Blut eines Lammes den vernunftbegabten Menschen befreien? Gewiß, sagt er, weil es auf das Blut des Herrn verweist. Wenn der Feind nicht das Blut des Vorbildes an Pfosten, sondern auf den Lippen der Glaubenden das kostbare Blut der Wahrheit leuchten sieht, mit dem der Tempel Christi geweiht ist, dann weicht er viel weiter zurück. 

Willst du der Kraft dieses Blutes noch weiter nachforschen? Dann schau bitte, woher es kommt und aus welcher Quelle es entspringt. Vom Kreuz Christi kam es zuerst, aus der Seite Christi nahm es den Anfang. Denn das Evangelium berichtet: Als Jesus tot war und noch am Kreuz hing, kam ein Soldat herbei und stieß die Seite auf. Da floß Wasser und Blut heraus: Symbol der Taufe das eine, Symbol des Mysteriums (der Eucharistie) das andere. Der Soldat hat die Seite geöffnet und die Wand des Tempels aufgetan. Ich habe den herrlichen Schatz gefunden und bin glücklich, den glanzvollen Reichtum entdeckt zu haben. So war es auch mit dem Lamm: Die Juden haben es geschlachtet, und ich erfahre die Frucht des Opfers. 

Blut und Wasser aus der Seite. Lieber Hörer, bitte geh nicht eilig an dem verborgenen Mysterium vorbei. Denn ich muß noch mystische und geheime Dinge aussprechen: Ich sagte, dieses Wasser und Blut seien Sinnzeichen für die Taufe und das Mysterium. Daraus ist die heilige Kirche aufgebaut, durch die Wiedergeburt aus dem Wasser und die Erneuerung des Heiligen Geistes, ich sage euch: durch die Taufe und das Mysterium, das aus seiner Seite hervorging. Aus seiner Seite nämlich baute Christus die Kirche, wie aus der Seite Adams Eva, die Gattin, kam. 

Dafür ist auch Paulus Zeuge, wenn er sagt: „Wir sind Glieder seines Leibes“, von seinem Gebein genommen, womit er die Seite meint. Denn wie Gott aus der Seite des Adam die Frau schuf, so gab uns Christus aus seiner Seite Wasser und Blut, wodurch die Kirche erbaut werden sollte. Wie Gott die Seite öffnete, während Adam im Schlaf ruhte, so schenkte er uns jetzt nach dem Tode Christi aus seiner Seite das Wasser und das Blut.