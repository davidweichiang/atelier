\documentclass[11pt,twoside]{book}

%%Page Size (rev. 08/19/2016)
%\usepackage[inner=0.5in, outer=0.5in, top=0.5in, bottom=0.5in, papersize={6in,9in}, head=12pt, headheight=30pt, headsep=5pt]{geometry}
\usepackage[inner=0.5in, outer=0.5in, top=0.5in, bottom=0.5in, papersize={5.5in,8.5in}, head=12pt, headheight=30pt, headsep=5pt]{geometry}
%% width of textblock = 324 pt / 4.5in
%% A5 = 5.8 x 8.3 inches -- if papersize is A5, then margins should be [inner=0.75in, outer=0.55in, top=0.4in, bottom=0.4in]


%%Header (rev. 4/11/2011)
\usepackage{fancyhdr}
 \pagestyle{fancy}
\renewcommand{\chaptermark}[1]{\markboth{#1}{}}
\renewcommand{\sectionmark}[1]{\markright{#1}}
 \fancyhf{}
\fancyhead[LE,RO]{\thepage}
\fancyhead[CE]{\leftmark}
\fancyhead[CO]{\rightmark}
 \fancypagestyle{plain}{ %
\fancyhf{} % remove everything
\renewcommand{\headrulewidth}{0pt} % remove lines as well
\renewcommand{\footrulewidth}{0pt}}



\usepackage[autocompile,allowdeprecated=false]{gregoriotex}
\usepackage{gregoriosyms}
\gresetgregoriofont[op]{greciliae}




%%Titles (rev. 9/4/2011) -- TOCLESS --- lets you have sections that don't appear in the table of contents

\setcounter{secnumdepth}{-1}

\usepackage[compact,nobottomtitles*]{titlesec}
\titlespacing*{\chapter}{0pt}{-30pt}{0pt}
\titlespacing*{\section}{0pt}{*0}{*1}
\titlespacing*{\subsubsection}{0pt}{*0}{*0}
\titlespacing*{\subsubsubsection}{0pt}{10pt}{*0}
\titleformat{\part} {\normalfont\Huge\sc\center}{\thechapter}{1em}{}
\titleformat{\chapter} {\normalfont\LARGE\sc\center}{\thechapter}{1em}{}
\titleformat{\section} {\normalfont\Large\sc\center}{\thesection}{1em}{}
\titleformat{\subsection} {\normalfont\Large\sc\center}{\thesubsubsection}{1em}{}
\titleformat{\subsubsection}{\normalfont\large\sc\center}{\thesubsubsubsection}{1em}{}
\titleformat{\paragraph}{\normalfont\normalsize\sc\center}{\thesubsubsubsection}{1em}{}

\newcommand{\nocontentsline}[3]{}
\newcommand{\tocless}[2]{\bgroup\let\addcontentsline=\nocontentsline#1{#2}\egroup} %% lets you have sections that don't appear in the table of contents


%%%

%%Index (rev. December 11, 2013)
\usepackage[noautomatic,nonewpage]{imakeidx}


\makeindex[name=incipit,title=Index]
\indexsetup{level=\section,toclevel=section,noclearpage}

\usepackage[indentunit=8pt,rule=.5pt,columns=2]{idxlayout}


%%Table of Contents (rev. May 16, 2011)

%\usepackage{multicol}
%\usepackage{ifthen}
%\usepackage[toc]{multitoc}

%% General settings (rev. January 19, 2015)

\usepackage[normalem]{ulem}

\usepackage[latin,english]{babel}
\usepackage{lettrine}

\usepackage{paracol}

\usepackage{fontspec}

\setmainfont[Ligatures=TeX,BoldFont=MinionPro-Bold,ItalicFont=MinionPro-It, BoldItalicFont=MinionPro-BoldIt]{MinionPro-Regular-Modified.otf}


%% Style for translation line
\grechangestyle{translation}{\fontsize{10}{10}\it\selectfont}
\grechangestyle{annotation}{\fontsize{10}{10}\selectfont}
\grechangestyle{commentary}{\textnormal\selectfont}
\gresetcustosalteration{invisible}

%\grechangedim{annotationseparation}{0.1cm}{scalable}

%\GreLoadSpaceConf{smith-four}

\frenchspacing

\usepackage{indentfirst} %%%indents first line after a section

\usepackage{graphicx}
%\usepackage{tocloft}

%%Hyperref (rev. August 20, 2011)
%\usepackage[colorlinks=false,hyperindex=true,bookmarks=true]{hyperref}
\usepackage{hyperref}
\hypersetup{pdftitle={Vesperale O.P. 2016}}
\hypersetup{pdfauthor={Order of Preachers}}
\hypersetup{pdfsubject={Liturgy}}
\hypersetup{pdfkeywords={Dominican, Liturgy, Order of Preachers, Dominican Rite, Liturgia Horarum, Divine Office}}

\newlength{\drop}



\begin{document}


\raggedbottom

\newcommand{\lectio}[3]{%
  \makebox[0pt][l]{#1}%
  \makebox[\textwidth][c]{#2}%
  \makebox[0pt][r]{\normalsize{\textnormal{#3}}}}


%%Combination

\chapter{8\textsuperscript{th} Sunday in Ordinary Time (Year C)}
\section{Second Vespers}
    \index[Varia]{Deus in adiutorium} \label{Deus in adiutorium (Varia)} \grecommentary[0pt]{} \gresetinitiallines{1} \grechangedim{spaceabovelines}{0.25 cm}{scalable} \grechangedim{maxbaroffsettextleft}{0 cm}{scalable} \gregorioscore{chants/misc.deus_in_adjutorium-T} \grechangedim{spaceabovelines}{0 cm}{scalable} \grechangedim{maxbaroffsettextleft}{0.6 cm}{scalable}
 \subsubsection{Hymnus}  \greannotation{VIII} \index[Hymnus]{Lucis creator} \label{Lucis creator (Hymnus)} \grecommentary[10pt]{\emph{Lucis creator optime}} \gresetinitiallines{1} \grechangestyle{initial}{\fontsize{36}{36}\selectfont} \grechangedim{maxbaroffsettextleft@nobar}{12 cm}{scalable} \grechangedim{spaceabovelines}{0.5cm}{scalable} \gresetlyriccentering{syllable}   \gregorioscore{chants/hy--lucis-creator-english}
 \subsubsection{Antiphona}  \greannotation{VII d} \index[Antiphona]{Dixit Dominus} \label{Dixit Dominus (Antiphona)} \grecommentary[0pt]{Ps 109:1} \gresetinitiallines{1} \gresetlyriccentering{vowel} \grechangedim{maxbaroffsettextleft}{0 cm}{scalable} \gregorioscore{chants/an--dixit_dominus_domino_meo--dominican-mss}  \grechangedim{maxbaroffsettextleft}{0.6 cm}{scalable}
 \subsubsection{Psalm 109} \paragraph{The Messiah, king and priest}  \index[Psalmus]{Psalm 109} \label{Psalm 109 (Psalmus)} \emph{Christ’s reign will last until all his enemies are made subject to him (1~Cor 15:25).}    \vspace{5pt} \par %%underlines for psalm tones with three movements in the second and three movements in the third sections.

\noindent The Lord’s revelation to my \uline{Master:}~†~\nopagebreak

“Sit \uline{on} my right:~$\star$~\nopagebreak

your foes I will put be\uline{neath} your feet.”

\noindent The Lord will wield from \uline{Zion}~†~\nopagebreak

your scep\uline{ter} of power:~$\star$~\nopagebreak

rule in the midst of \uline{all} your foes.

\noindent A prince from the day of your \uline{birth}~†~\nopagebreak

on the \uline{ho}ly mountains;~$\star$~\nopagebreak

from the womb before the dawn \uline{I} begot you.

\noindent The Lord has sworn an oath he will not \uline{change}.~†~\nopagebreak

“You are a \uline{priest} for ever,~$\star$~\nopagebreak

a priest like Melchize\uline{dek} of old.”

\noindent The Master standing at \uline{your} right hand~$\star$~\nopagebreak

will shatter kings in the day of \uline{his} great wrath.

\noindent He shall drink from the stream \uline{by} the wayside~$\star$~\nopagebreak

and therefore he shall lift \uline{up} his head.

\noindent Glory to the Father, and \uline{to} the Son,~$\star$~\nopagebreak

and to the \uline{Ho}ly Spirit:

\noindent as it was in the begin\uline{ning}, is now,~$\star$~\nopagebreak

and will be for ev\uline{er}. Amen.

 \subsubsection{Antiphona}  \greannotation{IV E} \index[Antiphona]{Psalm 111} \label{Psalm 111 (Antiphona)} \grecommentary[0pt]{Ps 111:1} \gresetinitiallines{1} \gresetlyriccentering{vowel} \grechangedim{maxbaroffsettextleft}{0 cm}{scalable} \gregorioscore{chants/an--in_mandatis--dominican}  \grechangedim{maxbaroffsettextleft}{0.6 cm}{scalable}
 \subsubsection{Psalm 111} \paragraph{The happiness of the just man}  \index[Psalmus]{Psalm 111} \label{Salus et gloria (Psalmus)} \emph{Live as children born of the light. Light produces every kind of goodness and justice and truth (Eph~5:8--9).}    \vspace{5pt} \par \noindent Happy the man who \uline{fears} the Lord,~$\star$~\nopagebreak

who takes delight in all \uline{his} commands.

\noindent His sons will be power\uline{ful} on earth;~$\star$~\nopagebreak

the children of the up\uline{right} are blessed.

\noindent Riches and wealth are \uline{in} his house;~$\star$~\nopagebreak

his justice stands \uline{firm} for ever.

\noindent He is a light in the darkness \uline{for} the upright:~$\star$~\nopagebreak

he is generous, merci\uline{ful} and just.

\noindent The good man takes pi\uline{ty} and lends,~$\star$~\nopagebreak

he conducts his af\uline{fairs} with honor.

\noindent The just man will \uline{nev}er waver:~$\star$~\nopagebreak

he will be remem\uline{bered} for ever.

\noindent He has no fear of \uline{ev}il news;~$\star$~\nopagebreak

with a firm heart he trusts \uline{in} the Lord.

\noindent With a steadfast heart he \uline{will} not fear;~$\star$~\nopagebreak

he will see the downfall \uline{of} his foes.

\noindent Open-handed, he gives to the \uline{poor;}~†~\nopagebreak

his justice stands \uline{firm} for ever.~$\star$~\nopagebreak

His head will be \uline{raised} in glory.

\noindent The wicked man sees and is \uline{angry,}~†~\nopagebreak

grinds his teeth and \uline{fades} away;~$\star$~\nopagebreak

the desire of the wicked \uline{leads} to doom.

\noindent Glory to the Father, and \uline{to} the Son,~$\star$~\nopagebreak

and to the \uline{Ho}ly Spirit:

\noindent as it was in the begin\uline{ning}, is now,~$\star$~\nopagebreak

and will be for ev\uline{er}. Amen.


 %%This is the version for Week II and IV
  \subsubsection{Canticum} \paragraph{The wedding of the Lamb} \greannotation{} \index[Canticum]{Salus et gloria} \label{Salus et gloria (Canticum)} \grecommentary[5pt]{Cf. Ap 19:1--2, 5--7 [AR]} \gresetinitiallines{1} \gresetlyriccentering{syllable}  \gregorioscore{chants/canticle--salus-et-honor--d--english} \newpage

 \subsubsection{Lectio brevis}     \hfill Heb 12:22--24    \lettrine[lines=3]{Y}{}ou have drawn near to Mount Zion and the city of the living God, the heavenly Jerusalem, to myriads of angels in festal gathering, to the assembly of the first-born enrolled in heaven, to God the judge of all, to the spirits of just men made perfect, to Jesus, the mediator of a new covenant, and to the sprinkled blood which speaks more eloquently than that of Abel.


 \subsubsection{Responsorium brevis}  \greannotation{VI} \index[Responsorium brevis]{Magnus Dominus noster} \label{Magnus Dominus noster (Responsorium brevis)} \grecommentary[0pt]{Ps 146:5 [AR]} \gresetinitiallines{1} \gresetlyriccentering{vowel}  \gregorioscore{chants/rb--magnus_dominus_noster--solesmes} \newpage

  \subsubsection{Antiphona ad Magnificat}  \greannotation{VIII \textsc{g}} \index[Antiphona ad Magnificat]{Non potest arbor bona} \label{Non potest arbor bona (Antiphona ad Magnificat)} \grecommentary[0pt]{Mt 7:18 [AR]} \gresetinitiallines{1}  \grechangestyle{initial}{\fontsize{36}{36}\selectfont} \grechangedim{maxbaroffsettextleft@nobar}{12 cm}{scalable} \grechangedim{spaceabovelines}{0.5cm}{scalable} \gresetlyriccentering{vowel}   \gregorioscore{chants/an--non_potest_arbor_bona--solesmes-op} \vspace{5pt} \emph{A good tree cannot bear bad fruit, nor can a bad tree bear good fruit.}

 \subsubsection{Canticum Evangelicum} \paragraph{The soul rejoices in the Lord} \greannotation{VII a} \index[Canticum Evangelicum]{} \label{ (Canticum Evangelicum)} \grecommentary[2pt]{Lc 1:46--55} \gresetinitiallines{1} \gresetlyriccentering{vowel}  \gregorioscore{chants/magnificat8G-c3} \vspace{10pt} 

\emph{My soul proclaims the greatness of the Lord, my spirit rejoices in God my Savior for he has looked with favor on his lowly servant. From this day all generations will call me blessed: the Almighty has done great things for me, and holy is his Name. He has mercy on those who fear him in every generation. He has shown the strength of his arm, he has scattered the proud in their conceit. He has cast down the mighty from their thrones, and has lifted up the lowly. He has filled the hungry with good things, and the rich he has sent away empty. He has come to the help of his servant Israel for he has remembered his promise of mercy, the promise he made to our fathers, to Abraham and his children for ever. Glory to the Father, and to the Son, and to the Holy Spirit: as it was in the beginning, is now, and will be for ever. Amen.}


 \subsubsection{Preces}   \index[Preces]{Week IV, Sunday, Second Vespers} \label{Week IV, Sunday, Second Vespers (Preces)}     \lettrine[loversize=0.15,lines=2]{R}{}ejoicing in the Lord, from whom all good things come, let us pray: \par Lord, hear our prayer.
\Rbar. Lord, hear our prayer.

Father and Lord of all, you sent your Son into the world, that your name might be glorified in every place,
– strengthen the witness of your Church among the nations.
\par \Rbar. Lord, hear our prayer.

Make us obedient to the teachings of your apostles,
– and bound to the truth of our faith.
\par \Rbar. Lord, hear our prayer.

As you love the innocent,
– render justice to those who are wronged.
\par \Rbar. Lord, hear our prayer.

Free those in bondage and give sight to the blind,
– raise up the fallen and protect the stranger.
\par \Rbar. Lord, hear our prayer.

Fulfill your promise to those who already sleep in your peace,
– through your Son grant them a blessed resurrection.
\par \Rbar. Lord, hear our prayer.


 \newpage

 \subsubsection{Pater noster}   \index[Pater noster]{Pater noster} \label{Pater noster (Pater noster)}   \grechangedim{spaceabovelines}{0.4 cm}{scalable}  \gregorioscore{chants/or--pater_noster_a--solesmes-T} \grechangedim{spaceabovelines}{0 cm}{scalable}

 \newpage

 \subsubsection{Oratio conclusiva}     \lettrine[lines=3]{G}{}rant us, O Lord, we pray,
that the course of our world
may be directed by your peaceful rule
and that your Church may rejoice,
untroubled in her devotion.
Through our Lord Jesus Christ, your Son,
who lives and reigns with you in the unity of the Holy Spirit,
one God, for ever and ever. \par \Rbar.~Amen.

 \subsubsection{Ritus conclusionis}         \par \Vbar.~The Lord be with you. \par \Rbar.~And with your spirit. \par \Vbar.~May almighty God bless you, the Father, and the Son, and the Holy Spirit. \par \Rbar.~Amen.
 \subsubsection{Benedicamus Domino}   \index[Benedicamus Domino]{Sundays} \label{Sundays (Benedicamus Domino)}      \gregorioscore{chants/misc.benedicamus.dominio.4-T}




 \end{document}
