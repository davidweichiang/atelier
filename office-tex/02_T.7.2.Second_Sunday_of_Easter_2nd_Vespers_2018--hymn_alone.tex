\documentclass[11pt,twoside]{book}

%%Page Size (rev. 08/19/2016)
%\usepackage[inner=0.5in, outer=0.5in, top=0.5in, bottom=0.5in, papersize={6in,9in}, head=12pt, headheight=30pt, headsep=5pt]{geometry}
\usepackage[inner=0.5in, outer=0.5in, top=0.5in, bottom=0.5in, papersize={5.5in,8.5in}, head=12pt, headheight=30pt, headsep=5pt]{geometry}
%% width of textblock = 324 pt / 4.5in
%% A5 = 5.8 x 8.3 inches -- if papersize is A5, then margins should be [inner=0.75in, outer=0.55in, top=0.4in, bottom=0.4in]


%%Header (rev. 4/11/2011)
\usepackage{fancyhdr}
 \pagestyle{fancy}
\renewcommand{\chaptermark}[1]{\markboth{#1}{}}
\renewcommand{\sectionmark}[1]{\markright{#1}}
 \fancyhf{}
\fancyhead[LE,RO]{\thepage}
\fancyhead[CE]{\leftmark}
\fancyhead[CO]{\rightmark}
 \fancypagestyle{plain}{ %
\fancyhf{} % remove everything
\renewcommand{\headrulewidth}{0pt} % remove lines as well
\renewcommand{\footrulewidth}{0pt}}



\usepackage[autocompile,allowdeprecated=false]{gregoriotex}
\usepackage{gregoriosyms}
\gresetgregoriofont[op]{greciliae}




%%Titles (rev. 9/4/2011) -- TOCLESS --- lets you have sections that don't appear in the table of contents

\setcounter{secnumdepth}{-1}

\usepackage[compact,nobottomtitles*]{titlesec}
\titlespacing*{\chapter}{0pt}{-30pt}{0pt}
\titlespacing*{\section}{0pt}{*0}{*1}
\titlespacing*{\subsubsection}{0pt}{*0}{*0}
\titlespacing*{\subsubsubsection}{0pt}{10pt}{*0}
\titleformat{\part} {\normalfont\Huge\sc\center}{\thechapter}{1em}{}
\titleformat{\chapter} {\normalfont\LARGE\sc\center}{\thechapter}{1em}{}
\titleformat{\section} {\normalfont\Large\sc\center}{\thesection}{1em}{}
\titleformat{\subsection} {\normalfont\Large\sc\center}{\thesubsubsection}{1em}{}
\titleformat{\subsubsection}{\normalfont\large\sc\center}{\thesubsubsubsection}{1em}{}
\titleformat{\paragraph}{\normalfont\normalsize\sc\center}{\thesubsubsubsection}{1em}{}

\newcommand{\nocontentsline}[3]{}
\newcommand{\tocless}[2]{\bgroup\let\addcontentsline=\nocontentsline#1{#2}\egroup} %% lets you have sections that don't appear in the table of contents


%%%

%%Index (rev. December 11, 2013)
\usepackage[noautomatic,nonewpage]{imakeidx}


\makeindex[name=incipit,title=Index]
\indexsetup{level=\section,toclevel=section,noclearpage}

\usepackage[indentunit=8pt,rule=.5pt,columns=2]{idxlayout}


%%Table of Contents (rev. May 16, 2011)

%\usepackage{multicol}
%\usepackage{ifthen}
%\usepackage[toc]{multitoc}

%% General settings (rev. January 19, 2015)

\usepackage{ulem}

\usepackage[latin,english]{babel}
\usepackage{lettrine}

\usepackage{paracol}

\usepackage{fontspec}

\setmainfont[Ligatures=TeX,BoldFont=MinionPro-Bold,ItalicFont=MinionPro-It, BoldItalicFont=MinionPro-BoldIt]{MinionPro-Regular-Modified.otf}


\usepackage{unicode-math}
\setmathfont{LatinModernMath-Regular}
\AtBeginDocument{\grelatexsimpledefbarredsymbol{V}{0.1em}{0.12em}{0.14em}{0.18em}}


%% Style for translation line
\grechangestyle{translation}{\fontsize{10}{10}\it\selectfont}
\grechangestyle{annotation}{\fontsize{10}{10}\selectfont}
\grechangestyle{commentary}{\textnormal\selectfont}
\gresetcustosalteration{invisible}

%\grechangedim{annotationseparation}{0.1cm}{scalable}

%\GreLoadSpaceConf{smith-four}

\frenchspacing

\usepackage{indentfirst} %%%indents first line after a section

\usepackage{graphicx}
%\usepackage{tocloft}

%%Hyperref (rev. August 20, 2011)
%\usepackage[colorlinks=false,hyperindex=true,bookmarks=true]{hyperref}
\usepackage{hyperref}
\hypersetup{pdftitle={Vesperale O.P. 2017}}
\hypersetup{pdfauthor={Order of Preachers}}
\hypersetup{pdfsubject={Liturgy}}
\hypersetup{pdfkeywords={Dominican, Liturgy, Order of Preachers, Dominican Rite, Liturgia Horarum, Divine Office}}

\newlength{\drop}



\begin{document}


\raggedbottom

\newcommand{\lectio}[3]{%
  \makebox[0pt][l]{#1}%
  \makebox[\textwidth][c]{#2}%
  \makebox[0pt][r]{\normalsize{\textnormal{#3}}}}


%%Combination
\chapter[2\textsuperscript{nd} Sunday of Easter]{2\textsuperscript{nd} Sunday of Easter\\Solemn Choral Vespers}
\section{Second Vespers}
 

 \subsubsection{Hymnus}  \greannotation{I} \index[Hymnus]{Ad cenam Agni providi} \label{Ad cenam Agni providi (Hymnus)} \grecommentary[10pt]{\emph{Polyphonic verses by Tomás Luis de Victoria (c. 1548-1611)}} \gresetinitiallines{1} \gresetlyriccentering{vowel} \grechangedim{spaceabovelines}{0.5 cm}{scalable} \gregorioscore{chants/hy--ad_cenam_agni--latin--even_verses_in_italics--new_lines}

 \vspace{5pt}



 \emph{The Lamb's high banquet we await, In snow-white robes of royal state; And now, the Red Sea's channel past, To Christ our Prince we sing at last.}

 \emph{2. Upon the altar of the Cross His body has redeemed our loss; And tasting of his crimson blood Our life is hid with him in God.}

 \emph{3. That Paschal eve God's arm was bared, From devastating angel spared; By strength of hand our hosts went free From Pharaoh's ruthless tyranny.}

 \emph{4. Now Christ, our Paschal Lamb, is slain, The Lamb of God that knows no stain; The true oblation offer'd here, Our own unleavened bread sincere.}

 \emph{5. O great, O worthy Victim true, The pow'r of hell is crushed by you; A captive people, ransomed, live, And endless life once more you give.}

 \emph{6. For Christ, arising from the dead, From conquered hell triumphant, sped; He thrust the tyrant down enchained, And Paradise for man regained.}

 \emph{7. Creator great, be you our guide In this, the joy of Easter-tide: Whene'er assaults of death impend, Your people strengthen and defend.}

 \emph{8. All glory, Lord, to you we pay, Arisen from the dead today; With Father and the Spirit be All glory yours eternally. Amen.}


  \end{document}
