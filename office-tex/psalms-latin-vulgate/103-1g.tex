2. Confessiónem, et decórem \uline{i}ndu\uline{í}sti:~* amíctus lúmine sicut v\uuline{e}stim\uline{é}nto.\par 
3. Exténdens cælum s\uline{i}cut p\uline{e}llem:~* qui tegis aquis superi\uuline{ó}ra \uline{e}jus.\par 
4. Qui ponis nubem asc\uline{é}nsum t\uline{u}um:~* qui ámbulas super penn\uuline{a}s vent\uline{ó}rum.\par 
5. Qui facis ángelos t\uline{u}os, sp\uline{í}ritus:~* et minístros tuos ign\uuline{e}m ur\uline{é}ntem.\par 
6. Qui fundásti terram super stabilit\uline{á}tem s\uline{u}am:~* non inclinábitur in sǽc\uuline{u}lum s\uline{ǽ}culi.\par 
7. Abýssus, sicut vestiméntum, am\uline{í}ctus \uline{e}jus:~* super montes st\uuline{a}bunt \uline{a}quæ.\par 
8. Ab increpatióne t\uline{u}a f\uline{ú}gient:~* a voce tonítrui tui f\uuline{o}rmid\uline{á}bunt.\par 
9. Ascéndunt montes: et desc\uline{é}ndunt c\uline{a}mpi~* in locum, quem fund\uuline{á}sti \uline{e}is.\par 
10. Términum posuísti, quem non transgr\uline{e}di\uline{é}ntur:~* neque converténtur oper\uuline{í}re t\uline{e}rram.\par 
11. Qui emíttis fontes \uline{i}n conv\uline{á}llibus:~* inter médium móntium pertrans\uuline{í}bunt \uline{a}quæ.\par 
12. Potábunt omnes b\uline{é}stiæ \uline{a}gri:~* exspectábunt ónagri in s\uuline{i}ti s\uline{u}a.\par 
13. Super ea vólucres cæli h\uline{a}bit\uline{á}bunt:~* de médio petrárum d\uuline{a}bunt v\uline{o}ces.\par 
14. Rigans montes de superi\uline{ó}ribus s\uline{u}is:~* de fructu óperum tuórum satiáb\uuline{i}tur t\uline{e}rra:\par 
15. Prodúcens f\uline{œ}num jum\uline{é}ntis:~* et herbam servit\uuline{ú}ti h\uline{ó}minum:\par 
16. Ut edúcas p\uline{a}nem de t\uline{e}rra:~* et vinum lætífic\uuline{e}t cor h\uline{ó}minis:\par 
17. Ut exhílaret fáci\uline{e}m in \uline{ó}leo:~* et panis cor hómin\uuline{i}s conf\uline{í}rmet.\par 
18. Saturabúntur ligna campi, et cedri Líbani, \uline{qua}s plant\uline{á}vit:~* illic pásseres nid\uuline{i}fic\uline{á}bunt.\par 
19. Heródii domus dux \uline{e}st e\uline{ó}rum:~* montes excélsi cervis: petra refúgium h\uuline{e}rin\uline{á}ciis.\par 
20. Fecit l\uline{u}nam in t\uline{é}mpora:~* sol cognóvit occ\uuline{á}sum s\uline{u}um.\par 
21. Posuísti ténebras, et f\uline{a}cta \uline{e}st nox:~* in ipsa pertransíbunt omnes bést\uuline{i}æ s\uline{i}lvæ.\par 
22. Cátuli leónum rugi\uline{é}ntes, ut r\uline{á}piant:~* et quærant a Deo \uuline{e}scam s\uline{i}bi.\par 
23. Ortus est sol, et c\uline{o}ngreg\uline{á}ti sunt:~* et in cubílibus suis coll\uuline{o}cab\uline{ú}ntur.\par 
24. Exíbit homo ad \uline{o}pus s\uline{u}um:~* et ad operatiónem suam us\uuline{que} ad v\uline{é}sperum.\par 
25. Quam magnificáta sunt ópera t\uline{u}a, D\uline{ó}mine!~* ómnia in sapiéntia fecísti: impléta est terra possessi\uuline{ó}ne t\uline{u}a.\par 
26. Hoc mare magnum, et spati\uline{ó}sum m\uline{á}nibus:~* illic reptília, quorum n\uuline{o}n est n\uline{ú}merus.\par 
27. Animália pus\uline{í}lla cum m\uline{a}gnis:~* illic naves p\uuline{e}rtrans\uline{í}bunt.\par 
28. Draco iste, quem formásti ad illud\uline{é}ndum \uline{e}i:~* ómnia a te exspéctant ut des illis esc\uuline{a}m in t\uline{é}mpore.\par 
29. Dante te \uline{i}llis, c\uline{ó}lligent:~* aperiénte te manum tuam, ómnia implebúntur b\uuline{o}nit\uline{á}te.\par 
30. Averténte autem te fáciem, t\uline{u}rbab\uline{ú}ntur:~* áuferes spíritum eórum, et defícient, et in púlverem suum r\uuline{e}vert\uline{é}ntur.\par 
31. Emíttes spíritum tuum, et cr\uline{e}ab\uline{ú}ntur:~* et renovábis fác\uuline{i}em t\uline{e}rræ.\par 
32. Sit glória Dómin\uline{i} in s\uline{ǽ}culum:~* lætábitur Dóminus in opér\uuline{i}bus s\uline{u}is:\par 
33. Qui réspicit terram, et facit \uline{e}am tr\uline{é}mere:~* qui tangit mont\uuline{e}s, et f\uline{ú}migant.\par 
34. Cantábo Dómino in v\uline{i}ta m\uline{e}a:~* psallam Deo meo, \uuline{quá}mdi\uline{u} sum.\par 
35. Jucúndum sit ei el\uline{ó}quium m\uline{e}um:~* ego vero delectáb\uuline{o}r in D\uline{ó}mino.\par 
36. Defíciant peccatóres a terra, et iníqui \uline{i}ta ut n\uline{o}n sint:~* bénedic, ánima m\uuline{e}a, D\uline{ó}mino.\par 
37. Glória P\uline{a}tri, et F\uline{í}lio,~* et Spirít\uuline{u}i S\uline{a}ncto.\par 
38. Sicut erat in princípio, et n\uline{u}nc, et s\uline{e}mper,~* et in sǽcula sæcul\uuline{ó}rum. \uline{A}men.\par 
