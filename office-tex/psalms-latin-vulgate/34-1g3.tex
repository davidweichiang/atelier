2. Apprehénde \uline{a}rma et sc\uline{u}tum:~* et exsúrge in adjutór\uuline{i}um m\uline{i}hi.\par 
3. Effúnde frámeam, et conclúde advérsus eos, qui p\uline{e}rse\uline{quú}ntur me:~* dic ánimæ meæ: Salus t\uuline{u}a \uline{e}go sum.\par 
4. Confundántur et rev\uline{e}re\uline{á}ntur,~* quæréntes án\uuline{i}mam m\uline{e}am.\par 
5. Avertántur retrórsum, et c\uline{o}nfund\uline{á}ntur~* cogitántes m\uuline{i}hi m\uline{a}la.\par 
6. Fiant tamquam pulvis ante f\uline{á}ciem v\uline{e}nti:~* et Angelus Dómini co\uuline{á}rctans \uline{e}os.\par 
7. Fiat via illórum ténebr\uline{æ} et l\uline{ú}bricum:~* et Angelus Dómini pérs\uuline{e}quens \uline{e}os.\par 
8. Quóniam gratis abscondérunt mihi intéritum l\uline{á}quei s\uline{u}i:~* supervácue exprobravérunt án\uuline{i}mam m\uline{e}am.\par 
9. Véniat illi láqueus, quem i\uline{gnó}rat:~† et cáptio, quam abscóndit, appreh\uline{é}ndat \uline{e}um:~* et in láqueum cad\uuline{a}t in \uline{i}psum.\par 
10. Anima autem mea exsult\uline{á}bit in D\uline{ó}mino:~* et delectábitur super salut\uuline{á}ri s\uline{u}o.\par 
11. Omnia ossa m\uline{e}a d\uline{i}cent:~* Dómine, quis sím\uuline{i}lis t\uline{i}bi?\par 
12. Erípiens ínopem de manu forti\uline{ó}rum \uline{e}jus:~* egénum et páuperem a diripiént\uuline{i}bus \uline{e}um.\par 
13. Surgéntes t\uline{e}stes in\uline{í}qui,~* quæ ignorábam int\uuline{e}rrog\uline{á}bant me.\par 
14. Retribuébant mihi m\uline{a}la pro b\uline{o}nis:~* sterilitátem án\uuline{i}mæ m\uline{e}æ.\par 
15. Ego autem cum mihi mol\uline{é}sti \uline{e}ssent,~* induéb\uuline{a}r cil\uline{í}cio.\par 
16. Humiliábam in jejúnio \uline{á}nimam m\uline{e}am:~* et orátio mea in sinu meo c\uuline{o}nvert\uline{é}tur.\par 
17. Quasi próximum, et quasi fratrem nostrum, sic c\uline{o}mplac\uline{é}bam:~* quasi lugens et contristátus, sic hum\uuline{i}li\uline{á}bar.\par 
18. Et advérsum me lætáti sunt, et c\uline{o}nven\uline{é}runt:~* congregáta sunt super me flagélla, et \uuline{i}gnor\uline{á}vi.\par 
19. Dissipáti sunt, nec com\uline{pún}cti,~† tentavérunt me, subsannavérunt me subsann\uline{a}ti\uline{ó}ne:~* frenduérunt super me dént\uuline{i}bus s\uline{u}is.\par 
20. Dómine, \uline{qua}ndo resp\uline{í}cies?~* restítue ánimam meam a malignitáte eórum, a leónibus ún\uuline{i}cam m\uline{e}am.\par 
21. Confitébor tibi in eccl\uline{é}sia m\uline{a}gna,~* in pópulo grav\uuline{i} laud\uline{á}bo te.\par 
22. Non supergáudeant mihi qui adversántur m\uline{i}hi in\uline{í}que:~* qui odérunt me gratis et ánn\uuline{u}unt \uline{ó}culis.\par 
23. Quóniam mihi quidem pacífice l\uline{o}queb\uline{á}ntur:~* et in iracúndia terræ loquéntes, dolos c\uuline{o}git\uline{á}bant.\par 
24. Et dilatavérunt super m\uline{e} os s\uline{u}um:~* dixérunt: Euge, euge, vidérunt óc\uuline{u}li n\uline{o}stri.\par 
25. Vidísti, Dómin\uline{e}, ne s\uline{í}leas:~* Dómine, ne disc\uuline{é}das \uline{a} me.\par 
26. Exsúrge et inténde jud\uline{í}cio m\uline{e}o:~* Deus meus, et Dóminus meus in c\uuline{au}sam m\uline{e}am.\par 
27. Júdica me secúndum justítiam tuam, Dómine, D\uline{e}us m\uline{e}us,~* et non supergáud\uuline{e}ant m\uline{i}hi.\par 
28. Non dicant in córdibus \uline{su}is:~† Euge, euge, \uline{á}nimæ n\uline{o}stræ:~* nec dicant: Devoráv\uuline{i}mus \uline{e}um.\par 
29. Erubéscant et revere\uline{á}ntur s\uline{i}mul,~* qui gratulántur m\uuline{a}lis m\uline{e}is.\par 
30. Induántur confusióne et r\uline{e}ver\uline{é}ntia~* qui magna lo\uuline{quú}ntur s\uline{u}per me.\par 
31. Exsúltent et læténtur qui volunt just\uline{í}tiam m\uline{e}am:~* et dicant semper: Magnificétur Dóminus qui volunt pacem s\uuline{e}rvi \uline{e}jus.\par 
32. Et lingua mea meditábitur just\uline{í}tiam t\uline{u}am,~* tota die l\uuline{au}dem t\uline{u}am.\par 
33. Glória P\uline{a}tri, et F\uline{í}lio,~* et Spirít\uuline{u}i S\uline{a}ncto.\par 
34. Sicut erat in princípio, et n\uline{u}nc, et s\uline{e}mper,~* et in sǽcula sæcul\uuline{ó}rum. \uline{A}men.\par 
