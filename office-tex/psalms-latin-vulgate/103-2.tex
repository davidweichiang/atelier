2. Confessiónem, et decórem indu\uline{í}sti:~* amíctus lúmine sicut vest\uuline{i}m\uline{é}nto.\par 
3. Exténdens cælum sicut p\uline{e}llem:~* qui tegis aquis superiór\uuline{a} \uline{e}jus.\par 
4. Qui ponis nubem ascénsum t\uline{u}um:~* qui ámbulas super pennas v\uuline{e}nt\uline{ó}rum.\par 
5. Qui facis ángelos tuos, sp\uline{í}ritus:~* et minístros tuos ignem \uuline{u}r\uline{é}ntem.\par 
6. Qui fundásti terram super stabilitátem s\uline{u}am:~* non inclinábitur in sǽculum s\uuline{ǽ}c\uline{u}li.\par 
7. Abýssus, sicut vestiméntum, amíctus \uline{e}jus:~* super montes stab\uuline{u}nt \uline{a}quæ.\par 
8. Ab increpatióne tua f\uline{ú}gient:~* a voce tonítrui tui form\uuline{i}d\uline{á}bunt.\par 
9. Ascéndunt montes: et descéndunt c\uline{a}mpi~* in locum, quem fundást\uuline{i} \uline{e}is.\par 
10. Términum posuísti, quem non transgredi\uline{é}ntur:~* neque converténtur operír\uuline{e} t\uline{e}rram.\par 
11. Qui emíttis fontes in conv\uline{á}llibus:~* inter médium móntium pertransíb\uuline{u}nt \uline{a}quæ.\par 
12. Potábunt omnes béstiæ \uline{a}gri:~* exspectábunt ónagri in sit\uuline{i} s\uline{u}a.\par 
13. Super ea vólucres cæli habit\uline{á}bunt:~* de médio petrárum dab\uuline{u}nt v\uline{o}ces.\par 
14. Rigans montes de superióribus s\uline{u}is:~* de fructu óperum tuórum satiábit\uuline{u}r t\uline{e}rra:\par 
15. Prodúcens fœnum jum\uline{é}ntis:~* et herbam servitúti h\uuline{ó}m\uline{i}num:\par 
16. Ut edúcas panem de t\uline{e}rra:~* et vinum lætíficet cor h\uuline{ó}m\uline{i}nis:\par 
17. Ut exhílaret fáciem in \uline{ó}leo:~* et panis cor hóminis c\uuline{o}nf\uline{í}rmet.\par 
18. Saturabúntur ligna campi, et cedri Líbani, quas plant\uline{á}vit:~* illic pásseres nidif\uuline{i}c\uline{á}bunt.\par 
19. Heródii domus dux est e\uline{ó}rum:~* montes excélsi cervis: petra refúgium herin\uuline{á}c\uline{i}is.\par 
20. Fecit lunam in t\uline{é}mpora:~* sol cognóvit occás\uuline{u}m s\uline{u}um.\par 
21. Posuísti ténebras, et facta \uline{e}st nox:~* in ipsa pertransíbunt omnes bésti\uuline{æ} s\uline{i}lvæ.\par 
22. Cátuli leónum rugiéntes, ut r\uline{á}piant:~* et quærant a Deo esc\uuline{a}m s\uline{i}bi.\par 
23. Ortus est sol, et congreg\uline{á}ti sunt:~* et in cubílibus suis colloc\uuline{a}b\uline{ú}ntur.\par 
24. Exíbit homo ad opus s\uline{u}um:~* et ad operatiónem suam usque ad v\uuline{é}sp\uline{e}rum.\par 
25. Quam magnificáta sunt ópera tua, D\uline{ó}mine!~* ómnia in sapiéntia fecísti: impléta est terra possessión\uuline{e} t\uline{u}a.\par 
26. Hoc mare magnum, et spatiósum m\uline{á}nibus:~* illic reptília, quorum non est n\uuline{ú}m\uline{e}rus.\par 
27. Animália pusílla cum m\uline{a}gnis:~* illic naves pertr\uuline{a}ns\uline{í}bunt.\par 
28. Draco iste, quem formásti ad illudéndum \uline{e}i:~* ómnia a te exspéctant ut des illis escam in t\uuline{é}mp\uline{o}re.\par 
29. Dante te illis, c\uline{ó}lligent:~* aperiénte te manum tuam, ómnia implebúntur bon\uuline{i}t\uline{á}te.\par 
30. Averténte autem te fáciem, turbab\uline{ú}ntur:~* áuferes spíritum eórum, et defícient, et in púlverem suum rev\uuline{e}rt\uline{é}ntur.\par 
31. Emíttes spíritum tuum, et creab\uline{ú}ntur:~* et renovábis fáci\uuline{e}m t\uline{e}rræ.\par 
32. Sit glória Dómini in s\uline{ǽ}culum:~* lætábitur Dóminus in opérib\uuline{u}s s\uline{u}is:\par 
33. Qui réspicit terram, et facit eam tr\uline{é}mere:~* qui tangit montes, et f\uuline{ú}m\uline{i}gant.\par 
34. Cantábo Dómino in vita m\uline{e}a:~* psallam Deo meo, quámd\uuline{i}\uline{u} sum.\par 
35. Jucúndum sit ei elóquium m\uline{e}um:~* ego vero delectábor in D\uuline{ó}m\uline{i}no.\par 
36. Defíciant peccatóres a terra, et iníqui ita ut n\uline{o}n sint:~* bénedic, ánima mea, D\uuline{ó}m\uline{i}no.\par 
37. Glória Patri, et F\uline{í}lio,~* et Spirítu\uuline{i} S\uline{a}ncto.\par 
38. Sicut erat in princípio, et nunc, et s\uline{e}mper,~* et in sǽcula sæculór\uuline{u}m. \uline{A}men.\par 
