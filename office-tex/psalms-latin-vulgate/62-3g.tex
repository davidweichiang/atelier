2. Sitívit in te \uline{á}nima m\uline{e}a,~* quam multiplíciter tibi c\uline{a}ro m\uline{e}a.\par 
3. In terra desérta, et ínvia, et ina\uline{quó}sa:~† sic in sancto app\uline{á}rui t\uline{i}bi,~* ut vidérem virtútem tuam, et gl\uline{ó}riam t\uline{u}am.\par 
4. Quóniam mélior est misericórdia tua s\uline{u}per v\uline{i}tas:~* lábia m\uline{e}a laud\uline{á}bunt te.\par 
5. Sic benedícam te in v\uline{i}ta m\uline{e}a:~* et in nómine tuo levábo m\uline{a}nus m\uline{e}as.\par 
6. Sicut ádipe et pinguédine repleátur \uline{á}nima m\uline{e}a:~* et lábiis exsultatiónis laud\uline{á}bit os m\uline{e}um.\par 
7. Si memor fui tui super stratum \uline{me}um,~† in matutínis medit\uline{á}bor \uline{i}n te:~* quia fuísti adj\uline{ú}tor m\uline{e}us.\par 
8. Et in velaménto alárum tuárum exsul\uline{tá}bo,~† adhǽsit ánima m\uline{e}a p\uline{o}st te:~* me suscépit d\uline{é}xtera t\uline{u}a.\par 
9. Ipsi vero in vanum quæsiérunt ánimam \uline{me}am,~† introíbunt in inferi\uline{ó}ra t\uline{e}rræ:~* tradéntur in manus gládii, partes v\uline{ú}lpium \uline{e}runt.\par 
10. Rex vero lætábitur in \uline{De}o,~† laudabúntur omnes qui j\uline{u}rant in \uline{e}o:~* quia obstrúctum est os loquénti\uline{u}m in\uline{í}qua.\par 
11. Glória Patri, et F\uline{í}l\uline{i}o,~* et Spir\uline{í}tui S\uline{a}ncto.\par 
12. Sicut erat in princípio, et n\uline{u}nc, et s\uline{e}mper,~* et in sǽcula sæcul\uline{ó}rum. \uline{A}men.\par 
