2. Quia ipse super mária f\uuline{u}ndávit \uline{e}um:~* et super flúmina præparáv\uuline{i}t \uline{e}um.\par 
3. Quis ascéndet \uuline{i}n montem D\uline{ó}mini?~* aut quis stabit in loco sanct\uuline{o} \uline{e}jus?\par 
4. Innocens mánibus \uuline{e}t mundo c\uline{o}rde,~* qui non accépit in vano ánimam suam, nec jurávit in dolo próxim\uuline{o} s\uline{u}o.\par 
5. Hic accípiet benedicti\uuline{ó}nem a D\uline{ó}mino:~* et misericórdiam a Deo, salutár\uuline{i} s\uline{u}o.\par 
6. Hæc est generátio quær\uuline{é}ntium \uline{e}um,~* quæréntium fáciem De\uuline{i} J\uline{a}cob.\par 
7. Attóllite portas, príncipes, \uline{ve}stras,~† et elevámini, port\uuline{æ} ætern\uline{á}les:~* et introíbit R\uuline{e}x gl\uline{ó}riæ.\par 
8. Quis est \uuline{i}ste Rex gl\uline{ó}riæ?~* Dóminus fortis et potens: Dóminus potens \uuline{i}n pr\uline{ǽ}lio.\par 
9. Attóllite portas, príncipes, \uline{ve}stras,~† et elevámini, port\uuline{æ} ætern\uline{á}les:~* et introíbit R\uuline{e}x gl\uline{ó}riæ.\par 
10. Quis est \uuline{i}ste Rex gl\uline{ó}riæ?~* Dóminus virtútum ipse est R\uuline{e}x gl\uline{ó}riæ.\par 
11. Glória P\uuline{a}tri, et F\uline{í}lio,~* et Spirítu\uuline{i} S\uline{a}ncto.\par 
12. Sicut erat in princípio, \uuline{e}t nunc, et s\uline{e}mper,~* et in sǽcula sæculór\uuline{u}m. \uline{A}men.\par 
