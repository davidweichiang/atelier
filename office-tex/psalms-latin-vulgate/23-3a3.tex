2. Quia ipse super mária fund\uline{á}vit \uline{e}um:~* et super flúmina præpar\uline{á}vit \uline{e}um.\par 
3. Quis ascéndet in montem D\uline{ó}m\uline{i}ni?~* aut quis stabit in loco s\uline{a}ncto \uline{e}jus?\par 
4. Innocens mánibus et m\uline{u}ndo c\uline{o}rde,~* qui non accépit in vano ánimam suam, nec jurávit in dolo pr\uline{ó}ximo s\uline{u}o.\par 
5. Hic accípiet benedictiónem a D\uline{ó}m\uline{i}no:~* et misericórdiam a Deo, salut\uline{á}ri s\uline{u}o.\par 
6. Hæc est generátio quær\uline{é}ntium \uline{e}um,~* quæréntium fáciem D\uline{e}i J\uline{a}cob.\par 
7. Attóllite portas, príncipes, \uline{ve}stras,~† et elevámini, portæ \uline{æ}tern\uline{á}les:~* et intro\uline{í}bit Rex gl\uline{ó}riæ.\par 
8. Quis est iste Rex gl\uline{ó}r\uline{i}æ?~* Dóminus fortis et potens: Dóminus p\uline{o}tens in pr\uline{ǽ}lio.\par 
9. Attóllite portas, príncipes, \uline{ve}stras,~† et elevámini, portæ \uline{æ}tern\uline{á}les:~* et intro\uline{í}bit Rex gl\uline{ó}riæ.\par 
10. Quis est iste Rex gl\uline{ó}r\uline{i}æ?~* Dóminus virtútum ipse \uline{e}st Rex gl\uline{ó}riæ.\par 
11. Glória Patri, et F\uline{í}l\uline{i}o,~* et Spir\uline{í}tui S\uline{a}ncto.\par 
12. Sicut erat in princípio, et n\uline{u}nc, et s\uline{e}mper,~* et in sǽcula sæcul\uline{ó}rum. \uline{A}men.\par 
